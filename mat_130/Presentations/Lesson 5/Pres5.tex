\documentclass{beamer}
\usepackage[utf8]{inputenc}
\usepackage{amsmath,amsfonts,amsthm,amstext,amssymb, xcolor, tikz, pgf}

% ----------------------------------------------------------
% Theme Setup

% Use Metropolis Theme
\usetheme[numbering=fraction]{metropolis}
\setbeamertemplate{blocks}[rounded][shadow=false]
\makeatletter
\setlength{\metropolis@titleseparator@linewidth}{1pt}
\makeatother

% Define Colors
\definecolor{chargerblue}{HTML}{002764}
\definecolor{chargerred}{HTML}{e02034}
\definecolor{bggray}{HTML}{d0d3d4}

% Set Colors
\setbeamercolor{title}{fg=chargerblue}
\setbeamercolor{background canvas}{bg=white}
\setbeamercolor{title separator}{fg=chargerred}
\setbeamercolor{structure}{fg=chargerblue}
\setbeamercolor{frametitle}{fg=white, bg=chargerblue}
\setbeamercolor*{normal text}{fg=chargerblue}
\setbeamercolor*{block body}{bg=bggray}
\setbeamercolor*{block title}{bg=chargerblue, fg=white}
% ----------------------------------------------------------

% ----------------------------------------------------------
% Custom Definitions, Commands, Environments, etc.

% Sets of numbers
\def\R{\mathbb{R}} % The reals
\def\N{\mathbb{N}} % The naturals
\def\Z{\mathbb{Z}} % The integers
\def\Q{\mathbb{Q}} % The rationals

% Blank space
\newcommand{\blank}[1]{\underline{\hspace{#1}}} % Blank space

% Fitted inclusion symbols
\newcommand{\fp}[1]{\left({#1}\right)} % Fitted parentheses around content
\newcommand{\fb}[1]{\left[{#1}\right]} % Fitted brackets
\newcommand{\set}[1]{\left\{{#1}\right\}} % Fitted braces (useful for sets)
\newcommand{\av}[1]{\left|{#1}\right|} % Fitted absolute value bars

% Degree symbol
\def\deg{^\circ}

% Coordinate Plane (Four-Quadrant)
\def\coordplane {
	\begin{tikzpicture}		\draw[step=0.25cm,black,very thin,opacity=0.25] (-2.5cm, -2.5cm) grid (2.5cm, 2.5cm);
		\draw[<->,thick,black] (-2.5cm, 0) -- (2.5cm, 0) node[anchor=north west,pos=0.94,font=\scriptsize]{$x$};
		\draw[<->,thick,black] (0,-2.5cm) -- (0, 2.5cm) node[anchor=south east,font=\scriptsize,pos=0.94]{$y$};
	\end{tikzpicture}
}

% Coordinate Plane (One-Quadrant)
\def\onequad {
	\begin{tikzpicture}
		\draw[step=0.25cm, black, very thin, opacity=0.25] (0,0) grid (7.5cm,5cm);
		\draw[->, thick, black] (0,0) -- (7.5cm, 0) node[anchor=north west,font=\scriptsize,pos=0.94]{$x$};
		\draw[->, black, thick] (0,0) -- (0,5cm) node[anchor=south east,font=\scriptsize,pos=0.94]{$y$};
	\end{tikzpicture}
}
% ----------------------------------------------------------

% ----------------------------------------------------------
% Presentation Information 
\title[1.3 and 1.4]{Linear Models; Quadratic Equations}
\subtitle{Sections 1.3 and 1.4}
\author{Jacob Ayers}
\institute{Lesson \#5}
\date{MAT 130}
% ----------------------------------------------------------

\begin{document}

% Slide 1 (Title Slide)
\begin{frame}
\titlepage
\end{frame}

% Slide 2 (Objectives)
\begin{frame}[t]{Objectives}
\begin{itemize}
	\item Solve problems involving linear models
	\item Solve quadratic equations by factoring
	\item Solve quadratic equations by extracting square roots
	\item Solve quadratic equations using the Quadratic Formula
\end{itemize}
\end{frame}

\begin{frame}[t]{Linear Models}
Many real-life situations can be modeled using linear equations.

Example: You accept a job with an annual income of \$58,400. This includes your salary and a \$1200 year-end bonus. You are paid weekly. What is your salary per pay period?

\begin{flalign*}
\onslide<2->{I &= N\cdot P + B & \\}
\onslide<3->{58400 &= 52P + 1200 & \\}
\onslide<4->{57200 &= 52P & \\}
\onslide<5>{\$P &= 1100}
\end{flalign*}
\end{frame}

\begin{frame}[t]{Linear Models}
You buy a stock at \$15 per share. You sell the stock at \$18 per share. What is the percent increase in the stock's value?

\begin{flalign*}
\onslide<2->{R &= P\cdot O & \\}
\onslide<3->{3 &= P\cdot 15 & \\}
\onslide<4->{P &= 0.2}
\end{flalign*}

\onslide<5>{The stock's price increased by 20\%.}
\end{frame}

\begin{frame}[t]{Linear Models}
A boat travels at a constant speed to an island 14 miles away. It takes 0.5 hour to travel the first 5 miles. How long does the entire trip take?

\pause

We can use the equation $d = rt$ to solve this equation.

\pause

First, find the rate of the boat: $5 = r(0.5) \Rightarrow r = 10$ mph.

\pause

Next, use this rate to determine how long it will take to travel 14 miles.

\pause

$14 = 10t \Rightarrow t = 1.4$

\pause

So it will take 1.4 hours, or 1 hour and 24 minutes, for the entire trip.
\end{frame}

\begin{frame}[t]{Common Formulas}
On page 95 of the text, there are two boxes filled with common formulas that can be used to solve applied problems.

You should know most of these already, but make sure you spend time getting to know them (with the exception of the interest formulas; we'll cover interest later). 
\end{frame}

\begin{frame}[t]{Using Common Formulas}
A cylinder has a volume of 250 cubic cm and a diameter of 10 inches. How tall is the can?

\pause

We use the formula for the volume of a cylinder, with $V = 250$ and $r = 5$.

\pause

\begin{flalign*}
V &= \pi r^2 h & \\
250 &= \pi (5)^2 h & \\
\dfrac{250}{25\pi} &= h \\
h &\approx 3.18 \text{ cm}
\end{flalign*}
\end{frame}

\begin{frame}[t]{Linear Models}
It is currently $65\deg$ F outside. What is the temperature in degrees Celsius?

\pause

We use the formula $C = \dfrac59\fp{F - 32}$, with $F = 65$.

\pause

\begin{flalign*}
C &= \dfrac59\fp{65 - 32} & \\
&= \dfrac59\fp{33} \\
&\approx 18.33\deg \text{ C}
\end{flalign*}
\end{frame}

\begin{frame}[t]{Quadratic Equations}
\begin{block}{Definition}
A \textit{quadratic equation} in $x$ is an equation that can be written in the form $$ax^2 + bx + c = 0$$ with $a,b,c\in\R$ and $a \neq 0$.
\end{block}

\pause

We have three main strategies for solving quadratic equations: \begin{itemize}
\pause \item Factoring
\pause \item Square Roots
\pause \item Quadratic Formula
\end{itemize}
\end{frame}

\begin{frame}[t]{Solving Quadratic Equations by Factoring}
\begin{block}{Strategy: Solve Quadratic Equations by Factoring}
\begin{enumerate}[1)]
\item Write the equation in the form $ax^2 + bx + c = 0$
\item Factor the left-hand side of the equation
\item Use the zero product property to find solution(s)
\end{enumerate}
\end{block}

\onslide<2->{Example: Solve $x^2 - 4x = 60$}

\begin{flalign*}
\onslide<3->{x^2 - 4x - 60 &= 0 & \\}
\onslide<4->{(x-10)(x+6) &= 0 & \\}
\onslide<5>{x &= \set{-6, 10}}
\end{flalign*}
\end{frame}

\begin{frame}[t]{Solving Quadratic Equations by Factoring}
Solve $2x^2 - 3x + 1 = 6$

\begin{flalign*}
\onslide<2->{2x^2 - 3x - 5 &= 0 & \\}
\onslide<3->{(2x - 5)(x + 1) &= 0 & \\}
\onslide<4>{x &= \set{-1, \dfrac{5}{2}}}
\end{flalign*}
\end{frame}

\begin{frame}[t]{Extracting Square Roots}
We can solve equations of the form $ax^2 + c = 0$ by extracting square roots.

\pause

\begin{block}{Strategy: Solve Quadratic Equations by Extracting Square Roots}
\begin{enumerate}[1)]
\item Isolate the $x^2$ term
\item Take the square root of each side of the equation
\item Write solution(s) (if there are solutions, they'll involve a $\pm$)
\end{enumerate}
\end{block}
\end{frame}

\begin{frame}[t]{Extracting Square Roots}
Solve for $x$: $4x^2 = 12$

\pause

$4x^2 = 12 \Rightarrow x^2 = 3 \Rightarrow x = \pm \sqrt{3}$ \vspace{18pt}

\pause

Solve for $x$: $(x-1)^2 = 10$

\pause

$(x-1)^2 = 10 \Rightarrow x - 1 = \pm \sqrt{10} \Rightarrow x = 1 \pm \sqrt{10}$
\end{frame}

\begin{frame}[t]{The Quadratic Formula}
The Quadratic Formula is one of the most important formulas in all of algebra. The formula can be used to solve \textit{any} quadratic equation.

\pause

\begin{block}{The Quadratic Formula}
Given an equation of the form $ax^2 + bx + c = 0$, we can find its solution(s) using the formula $$x = \dfrac{-b\pm\sqrt{b^2 - 4ac}}{2a}$$ if there are any.
\end{block}
\end{frame}

\begin{frame}[t]{The Quadratic Formula}
Solve for $x$: $3x^2 + 2x = 10$

First, write in the form $ax^2 + bx + c = 0$: $3x^2 + 2x - 10 = 0$ \\
$a = 3$, $b = 2$, $c = -10$
\begin{flalign*}
\onslide<2->{x &= \dfrac{-b\pm\sqrt{b^2 - 4ac}}{2a} & \\}
\onslide<3->{&= \dfrac{-(2) \pm \sqrt{(2)^2 - 4(3)(-10)}}{2(3)} & \\}
\onslide<4->{&= \dfrac{-2\pm\sqrt{124}}{6} & \\}
\onslide<5->{&= \dfrac{-2\pm 2\sqrt{31}}{6} & \\}
\onslide<6>{&= -\dfrac13 \pm \dfrac{\sqrt{31}}{3}}
\end{flalign*}
\end{frame}

\begin{frame}[t]{The Quadratic Formula}
Solve for $x$: $x^2 - 12x + 9 = 0$

$a = 1$, $b = -12$, $c = 9$

\begin{flalign*}
\onslide<2->{x &= \dfrac{-b\pm\sqrt{b^2 - 4ac}}{2a} & \\}
\onslide<3->{&= \dfrac{-(-12)\pm\sqrt{(-12)^2 - 4(1)(9)}}{2(1)} & \\}
\onslide<4->{&= \dfrac{12\pm\sqrt{108}}{2} & \\}
\onslide<5->{&= \dfrac{12\pm 6\sqrt{3}}{2} & \\}
\onslide<6>{&= 6 \pm 3\sqrt{3}}
\end{flalign*}
\end{frame}

\begin{frame}[t]{Applications}
A rectangular kitchen is 6 feet longer than it is wide and has an area of 112 square feet. Find the dimensions of the kitchen.

\pause

$A = \ell w$ \\
$\ell = w + 6$
\pause
\begin{flalign*}
112 &= w(w+6) & \\
112 &= w^2 + 6w & \\
w^2 + 6w - 112 &= 0 & \\
(w + 14)(w - 8) &= 0 & \\
w &= \set{-14, 8}
\end{flalign*}

\pause

But the width can't be negative. So the dimensions of the kitchen are 8 feet wide by 14 feet long. 
\end{frame}

\begin{frame}[t]{Applications}
A construction worker accidentally drops a wrench from height of 150 feet. How long will it take the wrench to hit the ground? Use the equation $s = -16t^2 + v_0 t + s_0$.

\pause

We will assume that the wrench has no initial velocity (i.e. $v_0 = 0$). We know that $s = 0$ (since we want to know when the wrench hits the ground and that $s_0 = 150$.
\pause
\begin{flalign*}
0 &= -16t^2 + 0t + 150 \\
-150 &= -16t^2 \\
9.375 &= t^2 \\
t &\approx 3.06 \text{ sec}
\end{flalign*}
\end{frame}

\begin{frame}[t]{Next Steps}
\begin{itemize}
\item Post questions in the Lesson 5 Forum, if you have them.
\item Read Sections 1.5 and 1.6
\item Watch Video Lesson \#6
\item Complete Assignment \#3
\end{itemize}

\vfill

Thanks for watching!

\end{frame}
\end{document}