\documentclass[t]{beamer}
\usepackage{amsmath,amsfonts,amsthm,amstext,amssymb, xcolor, tikz, pgf}

% ----------------------------------------------------------
% Theme Setup

% Use Metropolis Theme
\usetheme[numbering=fraction]{metropolis}
\setbeamertemplate{blocks}[rounded][shadow=false]
\makeatletter
\setlength{\metropolis@titleseparator@linewidth}{1pt}
\makeatother

% Define Colors
\definecolor{chargerblue}{HTML}{002764}
\definecolor{chargerred}{HTML}{e02034}
\definecolor{bggray}{HTML}{d0d3d4}

% Set Colors
\setbeamercolor{title}{fg=chargerblue}
\setbeamercolor{background canvas}{bg=white}
\setbeamercolor{title separator}{fg=chargerred}
\setbeamercolor{structure}{fg=chargerblue}
\setbeamercolor{frametitle}{fg=white, bg=chargerblue}
\setbeamercolor*{normal text}{fg=chargerblue}
\setbeamercolor*{block body}{bg=bggray}
\setbeamercolor*{block title}{bg=chargerblue, fg=white}
% ----------------------------------------------------------

% ----------------------------------------------------------
% Custom Definitions, Commands, Environments, etc.

% Sets of numbers
\def\R{\mathbb{R}} % The reals
\def\N{\mathbb{N}} % The naturals
\def\Z{\mathbb{Z}} % The integers
\def\Q{\mathbb{Q}} % The rationals

% Change font colors
\newcommand{\cyan}[1]{{\color{cyan}{#1}}} % Changes font to cyan
\newcommand{\red}[1]{{\color{red}{#1}}} % Changes font to red
\newcommand{\magenta}[1]{{\color{magenta}{#1}}} % Changes font to magenta
\newcommand{\orange}[1]{{\color{orange}{#1}}} % Changes font to orange
\newcommand{\yellow}[1]{{\color{yellow}{#1}}} % Changes font to yellow
\newcommand{\violet}[1]{{\color{violet}{#1}}} % Changes font to violet
\newcommand{\green}[1]{{\color{green}{#1}}} % Changes font to green
\newcommand{\blue}[1]{{\color{blue}{#1}}} % Changes font to blue
\newcommand{\white}[1]{{\color{white}{#1}}} % Changes font to white

% Blank space
\newcommand{\blank}[1]{\underline{\hspace{#1}}} % Blank space

% Fitted inclusion symbols
\newcommand{\fp}[1]{\left({#1}\right)} % Fitted parentheses around content
\newcommand{\fb}[1]{\left[{#1}\right]} % Fitted brackets
\newcommand{\set}[1]{\left\{{#1}\right\}} % Fitted braces (useful for sets)
\newcommand{\av}[1]{\left|{#1}\right|} % Fitted absolute value bars

% Augmented Matrix Environment
\newenvironment{amatrix}[1]{%
	\left[\begin{array}{@{}*{#1}{c}|c@{}}
	}{%
	\end{array}\right]
}

% Miscellaneous
\def\then{\Rightarrow}

% Coordinate Plane (Four-Quadrant)
\def\coordplane {
	\begin{tikzpicture}		\draw[step=0.25cm,black,very thin,opacity=0.25] (-2.5cm, -2.5cm) grid (2.5cm, 2.5cm);
	\draw[<->,thick,black] (-2.5cm, 0) -- (2.5cm, 0) node[anchor=north west,pos=0.94,font=\scriptsize]{$x$};
	\draw[<->,thick,black] (0,-2.5cm) -- (0, 2.5cm) node[anchor=south east,font=\scriptsize,pos=0.94]{$y$};
	\end{tikzpicture}
}

% Coordinate Plane (One-Quadrant)
\def\onequad {
	\begin{tikzpicture}
	\draw[step=0.25cm, black, very thin, opacity=0.25] (0,0) grid (7.5cm,5cm);
	\draw[->, thick, black] (0,0) -- (7.5cm, 0) node[anchor=north west,font=\scriptsize,pos=0.94]{$x$};
	\draw[->, black, thick] (0,0) -- (0,5cm) node[anchor=south east,font=\scriptsize,pos=0.94]{$y$};
	\end{tikzpicture}
}
% ----------------------------------------------------------

% ----------------------------------------------------------
% Presentation Information 
\title[Abbr]{Operations with Matrices}
\subtitle{Section 10.2}
\author{Jacob Ayers}
\institute{Lesson \#26}
\date{MAT 130}
% ----------------------------------------------------------

\begin{document}
	
	% Slide 1 (Title Slide)
	\begin{frame}
		\titlepage
	\end{frame}
	
	% Slide 2 (Objectives)
	\begin{frame}{Objectives}
		\begin{itemize}
			\item Determine whether two matrices are equal
			\item Perform addition, subtraction, and scalar multiplication of matrices
			\item Multiply two matrices
		\end{itemize}
	\end{frame}

	\begin{frame}{Matrix Equality}
		\begin{block}{Representation of Matrices}
			\begin{enumerate}[1)]
				\item A matrix can be denoted by a capital letter such as $A$, $B$, or $C$.
				\item A matrix can be denoted by a representative entry closed in brackets, such as $\fb{a_{ij}}$, $\fb{b_{ij}}$, or $\fb{c_{ij}}$.
				\item A matrix can be denoted as a rectangular array of numbers like we saw in the previous two lessons.
			\end{enumerate}
		\end{block} \pause
	
		\begin{block}{Definition}
			Given matrices $A$ and $B$, we say that $A = B$ if the following conditions are met: \begin{enumerate}[1)]
				\item $A$ and $B$ have the same dimension ($m\times n$)
				\item All of their corresponding entries are equal (i.e. $a_{ij} = b_{ij}$ for all $1\leq i\leq m$ and $1 \leq j \leq n$)
			\end{enumerate}
		\end{block}
	\end{frame}

	\begin{frame}{Matrix Equality}
		Solve the equation below:
		
		$\begin{bmatrix}
		a_{11} & a_{12} \\ a_{21} & a_{22}
		\end{bmatrix} = \begin{bmatrix}
		6 & 3 \\ 2 & -4
		\end{bmatrix}$ \pause \vspace{18pt}
		
		In order for two matrices to be equal, their corresponding elements must be equal. Therefore, \pause
		
		$a_{11} = 6$ \\ $a_{12} = 3$ \\ $a_{21} = 2$ \\ $a_{22} = -4$
	\end{frame}

	\begin{frame}{Addition, Subtraction, and Scalar Multiplication of Matrices}
		\begin{block}{Definition}
			If $A = \fb{a_{ij}}$ and $B = \fb{b_{ij}}$ are matrices of dimension $m\times n$, then their sum is $A + B = \fb{a_{ij} + b_{ij}}$ and their difference is $A - B = \fb{a_{ij} - b_{ij}}$. \vspace{12pt}
			
			If $A$ and $B$ do not have the same dimension, then their sum and difference are undefined.
		\end{block} \pause
	
		\begin{block}{Defintion}
			If $A = \fb{a_{ij}}$ is an $m\times n$ matrix and $c$ is a constant, then the \textit{scalar} multiple of $A$ by $c$ is the $m\times n$ matrix $$cA = \fb{ca_{ij}}$$
		\end{block}
	\end{frame}

	\begin{frame}{Addition, Subtraction, and Scalar Multiplication of Matrices}
		Given $A = \begin{bmatrix}
		7 & -3 \\ -2 & 0
		\end{bmatrix}$ and $B = \begin{bmatrix}
		12 & 8 \\ 6 & -11
		\end{bmatrix}$, find \begin{enumerate}[i)]
			\item $A + B$
			\item $A - B$
			\item $B - A$
			\item $2A - 5B$
		\end{enumerate}
	
		\onslide<2->{i) $A + B$} \begin{flalign*}
		\onslide<2->{\begin{bmatrix}
			7 & -3 \\ -2 & 0
			\end{bmatrix} + \begin{bmatrix}
			12 & 8 \\ 6 & -11
			\end{bmatrix} &= \begin{bmatrix}
			7 + 12 & -3 + 8 \\ -2 + 6 & 0 + (-11)
			\end{bmatrix} & \\}
		\onslide<3->{&= \begin{bmatrix}
			19 & 5 \\ 4 & -11
			\end{bmatrix}}
		\end{flalign*}
	\end{frame}

	\begin{frame}{Addition, Subtraction, and Scalar Multiplication of Matrices}
		$A = \begin{bmatrix}
		7 & -3 \\ -2 & 0
		\end{bmatrix}$ and $B = \begin{bmatrix}
		12 & 8 \\ 6 & -11
		\end{bmatrix}$
		
		ii) $A - B$
		\begin{flalign*}
		\onslide<2->{\begin{bmatrix}
			7 & -3 \\ -2 & 0
			\end{bmatrix} - \begin{bmatrix}
			12 & 8 \\ 6 & -11
			\end{bmatrix} &= \begin{bmatrix}
			7 - 12 & -3 - 8 \\ -2 - 6 & 0 - (-11)
			\end{bmatrix} & \\}
		\onslide<3->{&= \begin{bmatrix}
			-5 & -11 \\ -8 & 11
			\end{bmatrix}}
		\end{flalign*}
		iii) $B - A$
		\begin{flalign*}
		\onslide<4->{\begin{bmatrix}
			12 & 8 \\ 6 & -11
			\end{bmatrix} - \begin{bmatrix}
			7 & -3 \\ -2 & 0
			\end{bmatrix} &= \begin{bmatrix}
			12 - 7 & 8 - (-3) \\ 6 - (-2) & -11 - 0
			\end{bmatrix} & \\}
		\onslide<5->{&= \begin{bmatrix}
			5 & 11 \\ 8 & -11
			\end{bmatrix}}
		\end{flalign*}
	\end{frame}

	\begin{frame}{Addition, Subtraction, and Scalar Multiplication of Matrices}
		$A = \begin{bmatrix}
		7 & -3 \\ -2 & 0
		\end{bmatrix}$ and $B = \begin{bmatrix}
		12 & 8 \\ 6 & -11
		\end{bmatrix}$
		
		iv) $2A - 5B$
		\begin{flalign*}
		\onslide<2->{2\begin{bmatrix}
			7 & -3 \\ -2 & 0
			\end{bmatrix} - 5\begin{bmatrix}
			12 & 8 \\ 6 & -11
			\end{bmatrix} &= \begin{bmatrix}
			2(7) & 2(-3) \\ 2(-2) & 2(0)
			\end{bmatrix} - \begin{bmatrix}
			5(12) & 5(8) \\ 5(6) & 5(-11)
			\end{bmatrix} & \\}
		\onslide<3->{&= \begin{bmatrix}
			14 & -6 \\ -4 & 0
			\end{bmatrix} - \begin{bmatrix}
			60 & 40 \\ 30 & -55
			\end{bmatrix} & \\}
		\onslide<4->{&= \begin{bmatrix}
			14 - 60 & -6 - 40 \\ -4 - 30 & 0 - (-55)
			\end{bmatrix} & \\}
		\onslide<5->{&= \begin{bmatrix}
			-46 & -46 \\ -34 & 55
			\end{bmatrix}}
		\end{flalign*}
	\end{frame}

	\begin{frame}{Addition, Subtraction, and Scalar Multiplication of Matrices}
		Given that $A = \begin{bmatrix}
		4 & 8 \\ -8 & -4
		\end{bmatrix}$ and $B = \begin{bmatrix}
		1 & 2 & 3 \\ 4 & 5 & 6
		\end{bmatrix}$, find \begin{enumerate}[i)]
			\item $A + B$
			\item $-4B$
		\end{enumerate}
	
		i) \onslide<2->{$A + B$ is undefined since $A$ is a $2 \times 2$ matrix and $B$ is a $2 \times 3$ matrix.}
		
		ii) $-4B$ \begin{flalign*}
		\onslide<3->{-4\begin{bmatrix}
			1 & 2 & 3 \\ 4 & 5 & 6
			\end{bmatrix} &= \begin{bmatrix}
			-4(1) & -4(2) & -4(3) \\ -4(4) & -4(5) & -4(6)
			\end{bmatrix} & \\}
		\onslide<4->{&= \begin{bmatrix}
			-4 & -8 & -12 \\ -16 & -20 & -24
			\end{bmatrix}}
		\end{flalign*}
	\end{frame}

	\begin{frame}{Addition, Subtraction, and Scalar Multiplication of Matrices}
		Solve for $X$ in the equation $2X - A = B$ if $A = \begin{bmatrix}
		6 & 1 \\ 0 & 3
		\end{bmatrix}$ and $B = \begin{bmatrix}
		4 & -1 \\ -2 & 5
		\end{bmatrix}$
		\begin{flalign*}
		\onslide<2->{2X - \begin{bmatrix}
			6 & 1 \\ 0 & 3
			\end{bmatrix} &= \begin{bmatrix}
			4 & -1 \\ -2 & 5
			\end{bmatrix} & \\}
		\onslide<3->{2X &= \begin{bmatrix}
			4 & -1 \\ -2 & 5
			\end{bmatrix} + \begin{bmatrix}
			6 & 1 \\ 0 & 3
			\end{bmatrix} & \\}
		\onslide<4->{X &= \dfrac12 \fp{\begin{bmatrix}
				10 & 0 \\ -2 & 8
				\end{bmatrix}} & \\}
		\onslide<5->{&= \begin{bmatrix}
			5 & 0 \\ -1 & 4
			\end{bmatrix}}
		\end{flalign*}
	\end{frame}

	\begin{frame}{Matrix Multiplication}
		The definition of matrix multiplication isn't what you'd assume it is.
		
		\begin{block}{Definition}
			If $A = \fb{a_{ij}}$ is an $m \times n$ matrix and $B = \fb{b_{ij}}$ is an $n \times p$ matrix, then the product $AB$ is an $m \times p$ matrix given by $AB = \fb{c_{ij}}$, where $$c_{ij} = a_{i1}b_{1j} a_{i2}b_{2j} + a_{i3}b_{3j} + \cdots + a_{in}b_{nj}$$
		\end{block} \pause
	
		There is an excellent figure showing how matrix multiplication works on page 718 of the textbook.
	\end{frame}

	\begin{frame}{Matrix Multiplication}
		Here are some key points regarding matrix multiplication: \begin{itemize}
			\item The two matrices do NOT have to have the same dimension
			\item The number of \textit{columns} in the first matrix must be equal to the number of \textit{rows} in the second matrix
			\item The product matrix will have the same number of \textit{rows} as the first matrix and the same number of \textit{columns} as the second matrix
			\item To find an entry $c_{ij}$ in the product matrix, multiply the $i$th \textit{row} of the first matrix by the $j$th \textit{column} of the second matrix
			\item Matrix multiplication is NOT commutative (that is, $AB$ is not necessarily equal to $BA$)
		\end{itemize}
	\end{frame}

	\begin{frame}{Matrix Multiplication}
		Find $AB$ and $BA$ if $A = \begin{bmatrix}
		-1 & 3 \\ 4 & -2 \\ 5 & 0
		\end{bmatrix}$ and $B = \begin{bmatrix}
		-3 & 2 \\ -4 & 1
		\end{bmatrix}$
		
		\pause Matrix $A$ is $3\times 2$ and matrix $B$ is a $2 \times 2$. 
		
		Since the number of columns in $A$ (2) is equal to the number of rows in $B$, we can compute $AB$. It will be a $3 \times 2$ matrix since there are 3 rows in the first matrix and 2 columns in the second matrix. \pause
		
		Since the number of columns in $B$ (2) is not equal to the number of rows in $A$ (3), $BA$ is undefined. \pause
		
		Now, let's compute $AB$.
	\end{frame}

	\begin{frame}{Matrix Multiplication}
		\begin{center}
			\only<6>{$\begin{bmatrix}
				-1 & 3 \\ 4 & -2 \\ \red{5} & \red{0}
				\end{bmatrix}\begin{bmatrix}
				-3 & \blue{2} \\ -4 & \blue{1}
				\end{bmatrix} = \begin{bmatrix}
				-9 & 1 \\ -4 & 6 \\ -15 & \magenta{10}
				\end{bmatrix}$}
			\only<5>{$\begin{bmatrix}
				-1 & 3 \\ 4 & -2 \\ \red{5} & \red{0}
				\end{bmatrix}\begin{bmatrix}
				\blue{-3} & 2 \\ \blue{-4} & 1
				\end{bmatrix} = \begin{bmatrix}
				-9 & 1 \\ -4 & 6 \\ \magenta{-15} &
				\end{bmatrix}$}
			\only<4>{$\begin{bmatrix}
				-1 & 3 \\ \red{4} & \red{-2} \\ 5 & 0
				\end{bmatrix}\begin{bmatrix}
				-3 & \blue{2} \\ -4 & \blue{1}
				\end{bmatrix} = \begin{bmatrix}
				-9 & 1 \\ -4 & \magenta{6} \\  &
				\end{bmatrix}$}
			\only<3>{$\begin{bmatrix}
				-1 & 3 \\ \red{4} & \red{-2} \\ 5 & 0
				\end{bmatrix}\begin{bmatrix}
				\blue{-3} & 2 \\ \blue{-4} & 1
				\end{bmatrix} = \begin{bmatrix}
				-9 & 1 \\ \magenta{-4} &  \\  &
				\end{bmatrix}$}
			\only<2>{$\begin{bmatrix}
				\red{-1} & \red{3} \\ 4 & -2 \\ 5 & 0
				\end{bmatrix}\begin{bmatrix}
				-3 & \blue{2} \\ -4 & \blue{1}
				\end{bmatrix} = \begin{bmatrix}
				-9 & \magenta{1} \\ &  \\  &
				\end{bmatrix}$}
			\only<1>{$\begin{bmatrix}
				\red{-1} & \red{3} \\ 4 & -2 \\ 5 & 0
				\end{bmatrix}\begin{bmatrix}
				\blue{-3} & 2 \\ \blue{-4} & 1
				\end{bmatrix} = \begin{bmatrix}
				\magenta{-9} &  \\ &  \\  &
				\end{bmatrix}$}
		\end{center}
		
		\vspace{12pt}
		
		\onslide<1->{\only<1>{\color{magenta}}{$c_{11} = -1(-3) + 3(-4) = 3 - 12 = -9$ \\}}
		\onslide<2->{\only<2>{\color{magenta}}$c_{12} = -1(2) + 3(1) = -2 + 3 = 1$ \\}
		\onslide<3->{\only<3>{\color{magenta}}$c_{21} = 4(-3) + (-2)(-4) = -12 + 8 = -4$ \\}
		\onslide<4->{\only<4>{\color{magenta}}$c_{22} = 4(2) + (-2)(1) = 8 - 2 = 6$ \\}
		\onslide<5->{\only<5>{\color{magenta}}$c_{31} = 5(-3) + 0(-4) = -15 + 0 = -15$ \\}
		\onslide<6->{\only<6>{\color{magenta}}$c_{32} = 5(2) + 0(1) = 10 + 0 = 10$ \\}
	
	\end{frame}

	\begin{frame}{Matrix Multiplication}
		Find $AB$ if $A = \begin{bmatrix}
		0 & 4 & -3 \\ 2 & 1 & 7 \\ 3 & -2 & 1
		\end{bmatrix}$ and $B = \begin{bmatrix}
		-2 & 0 \\ 0 & -4 \\ 1 & 2
		\end{bmatrix}$ \pause
		
		Matrix $A$ is $3\times 3$ and matrix $B$ is $3 \times 2$.
		
		Since the number of columns in $A$ is equal to the number of rows in $B$, we can compute $AB$. It will be a $3 \times 2$ matrix since $A$ has 3 rows and $B$ has 2 columns.
	\end{frame}

	\begin{frame}{Matrix Multiplication}
		\begin{center}
			\only<6>{$\begin{bmatrix}
				0 & 4 & -3 \\ 2 & 1 & 7 \\ \red{3} & \red{-2} & \red{1}
				\end{bmatrix}\begin{bmatrix}
				-2 & \blue{0} \\ 0 & \blue{-4} \\ 1 & \blue{2}
				\end{bmatrix} = \begin{bmatrix}
				-3 & -22 \\ 3 & 10 \\ -5 & \magenta{10}
				\end{bmatrix}$}
			\only<5>{$\begin{bmatrix}
				0 & 4 & -3 \\ 2 & 1 & 7 \\ \red{3} & \red{-2} & \red{1}
				\end{bmatrix}\begin{bmatrix}
				\blue{-2} & 0 \\ \blue{0} & -4 \\ \blue{1} & 2
				\end{bmatrix} = \begin{bmatrix}
				-3 & -22 \\ 3 & 10 \\ \magenta{-5} &
				\end{bmatrix}$}
			\only<4>{$\begin{bmatrix}
				0 & 4 & -3 \\ \red{2} & \red{1} & \red{7} \\ 3 & -2 & 1
				\end{bmatrix}\begin{bmatrix}
				-2 & \blue{0} \\ 0 & \blue{-4} \\ 1 & \blue{2}
				\end{bmatrix} = \begin{bmatrix}
				-3 & -22 \\ 3 & \magenta{10} \\  &
				\end{bmatrix}$}
			\only<3>{$\begin{bmatrix}
				0 & 4 & -3 \\ \red{2} & \red{1} & \red{7} \\ 3 & -2 & 1
				\end{bmatrix}\begin{bmatrix}
				\blue{-2} & 0 \\ \blue{0} & -4 \\ \blue{1} & 2
				\end{bmatrix} = \begin{bmatrix}
				-3 & -22 \\ \magenta{3} & \\  &
				\end{bmatrix}$}
			\only<2>{$\begin{bmatrix}
				\red{0} & \red{4} & \red{-3} \\ 2 & 1 & 7 \\ 3 & -2 & 1
				\end{bmatrix}\begin{bmatrix}
				-2 & \blue{0} \\ 0 & \blue{-4} \\ 1 & \blue{2}
				\end{bmatrix} = \begin{bmatrix}
				-3 & \magenta{-22} \\  & \\  &
				\end{bmatrix}$}
			\only<1>{$\begin{bmatrix}
				\red{0} & \red{4} & \red{-3} \\ 2 & 1 & 7 \\ 3 & -2 & 1
				\end{bmatrix}\begin{bmatrix}
				\blue{-2} & 0 \\ \blue{0} & -4 \\ \blue{1} & 2
				\end{bmatrix} = \begin{bmatrix}
				\magenta{-3} & \\  & \\  &
				\end{bmatrix}$}
		\end{center}
	
		\vspace{12pt}
		
		\onslide<1->{\only<1>{\color{magenta}}{$c_{11} = 0(-2) + 4(0) + (-3)(1) = -3$ \\}}
		\onslide<2->{\only<2>{\color{magenta}}$c_{12} = 0(0) + 4(-4) + (-3)(2) = -22$ \\}
		\onslide<3->{\only<3>{\color{magenta}}$c_{21} = 2(-2) + 1(0) + 7(1) = 3$ \\}
		\onslide<4->{\only<4>{\color{magenta}}$c_{22} = 2(0) + 1(-4) + 7(2) = 10$ \\}
		\onslide<5->{\only<5>{\color{magenta}}$c_{31} = 3(-2) + -2(0) + 1(1) = -5$ \\}
		\onslide<6->{\only<6>{\color{magenta}}$c_{32} = 3(0) + -2(-4) + 1(2) = 10$ \\}
	\end{frame}

	\begin{frame}{Using Technology}
		In the previous lesson, you learned how to use your calculator to enter matrices. \pause
		
		Your graphing calculator is capable of performing all the computations we did in this lesson. \pause
		
		Any problem on Assignment \#13 that is marked with an asterisk (*) may be worked with a calculator (i.e. no work required).
	\end{frame}

	\begin{frame}{Next Steps}
		\begin{itemize}
			\item Post questions in Lesson 26 Forum, if you have any
			\item Read 10.4
			\item Watch Video Lesson \#27
			\item Complete Assignment \#13
		\end{itemize}
	\end{frame}
	
\end{document}