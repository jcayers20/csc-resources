\documentclass[t, aspectratio=169]{beamer}
\usepackage{amsmath,amsfonts,amsthm,amstext,amssymb, xcolor, tikz, pgf}

% ----------------------------------------------------------
% Theme Setup

% Use Metropolis Theme
\usetheme[numbering=fraction]{metropolis}
\setbeamertemplate{blocks}[rounded][shadow=false]
\makeatletter
\setlength{\metropolis@titleseparator@linewidth}{1pt}
\makeatother

% Define Colors
\definecolor{chargerblue}{HTML}{002764}
\definecolor{chargerred}{HTML}{e02034}
\definecolor{bggray}{HTML}{d0d3d4}

% Set Colors
\setbeamercolor{title}{fg=chargerblue}
\setbeamercolor{background canvas}{bg=white}
\setbeamercolor{title separator}{fg=chargerred}
\setbeamercolor{structure}{fg=chargerblue}
\setbeamercolor{frametitle}{fg=white, bg=chargerblue}
\setbeamercolor*{normal text}{fg=chargerblue}
\setbeamercolor*{block body}{bg=bggray}
\setbeamercolor*{block title}{bg=chargerblue, fg=white}
% ----------------------------------------------------------

% ----------------------------------------------------------
% Custom Definitions, Commands, Environments, etc.

% Sets of numbers
\def\R{\mathbb{R}} % The reals
\def\N{\mathbb{N}} % The naturals
\def\Z{\mathbb{Z}} % The integers
\def\Q{\mathbb{Q}} % The rationals

% Blank space
\newcommand{\blank}[1]{\underline{\hspace{#1}}} % Blank space

% Fitted inclusion symbols
\newcommand{\fp}[1]{\left({#1}\right)} % Fitted parentheses around content
\newcommand{\fb}[1]{\left[{#1}\right]} % Fitted brackets
\newcommand{\set}[1]{\left\{{#1}\right\}} % Fitted braces (useful for sets)
\newcommand{\av}[1]{\left|{#1}\right|} % Fitted absolute value bars

% Miscellaneous
\def\then{\Rightarrow}

% Coordinate Plane (Four-Quadrant)
\def\coordplane {
	\begin{tikzpicture}		\draw[step=0.25cm,black,very thin,opacity=0.25] (-2.5cm, -2.5cm) grid (2.5cm, 2.5cm);
	\draw[<->,thick,black] (-2.5cm, 0) -- (2.5cm, 0) node[anchor=north west,pos=0.94,font=\scriptsize]{$x$};
	\draw[<->,thick,black] (0,-2.5cm) -- (0, 2.5cm) node[anchor=south east,font=\scriptsize,pos=0.94]{$y$};
	\end{tikzpicture}
}

% Coordinate Plane (One-Quadrant)
\def\onequad {
	\begin{tikzpicture}
	\draw[step=0.25cm, black, very thin, opacity=0.25] (0,0) grid (7.5cm,5cm);
	\draw[->, thick, black] (0,0) -- (7.5cm, 0) node[anchor=north west,font=\scriptsize,pos=0.94]{$x$};
	\draw[->, black, thick] (0,0) -- (0,5cm) node[anchor=south east,font=\scriptsize,pos=0.94]{$y$};
	\end{tikzpicture}
}
% ----------------------------------------------------------

% ----------------------------------------------------------
% Presentation Information 
\title[9.1 and 9.2]{Systems of Two Equations}
\subtitle{Sections 9.1 and 9.2}
\author{Jacob Ayers}
\institute{Lesson \#22}
\date{MAT 130}
% ----------------------------------------------------------

\begin{document}
	
	% Slide 1 (Title Slide)
	\begin{frame}
		\titlepage
	\end{frame}
	
	% Slide 2 (Objectives)
	\begin{frame}{Objectives}
		\begin{itemize}
			\item Solve systems of equations by graphing
			\item Solve systems of equations using substitution
			\item Solve systems of equations using elimination
		\end{itemize}
	\end{frame}

	\begin{frame}{Systems of Equations - Definitions}
		A \textit{system of equations} is a set of two or more equations in two or more variables. \pause
		
		In this lesson, we'll focus on systems of two equations. \pause
		
		A \textit{solution} to a system of equations is a point that satisfies all of the equations.
	\end{frame}

	\begin{frame}{Solutions to Systems of Equations}
		Determine whether each point is a solution to the system $\begin{cases}
			2x + y &= 5 \\
			4x - 3y &= 5
		\end{cases}$
		
		a) $(3, -1)$ \\
		b) $(2,1)$ \\
		c) $(1,2)$ \pause
		
		a) $2(3) + (-1) = 5$ is true, but $4(3) - 3(-1) = 5$ is false. $(3, -1)$ is not a solution. \pause
		
		b) $2(2) + 1 = 5$ and $4(2) - 3(1) = 5$ are both true, so $(2, 1)$ is a solution. \pause
		
		c) $4(1) + 2 = 5$ is false, so $(1, 2)$ is not a solution.
	\end{frame}

	\begin{frame}{Systems of Equations - Graphing}
		The first method we can use to solve a system of equations in two variables is by graphing. \pause
		
		In this method, we graph both equations and then see where they intersect. \pause
	\end{frame}

	\begin{frame}{Systems of Equations - Graphing}
		Solve the system of equations by graphing.
		
		$\begin{cases}
			y = 2x + 1 \\
			y = -4x + 7
		\end{cases}$
		
		\pause You can graph these equations using your graphing calculator or online tools. Then it's just a matter of using the intersect tool to determine the exact point of intersection.
		
		\pause In this case, the two lines intersect at the point $(1, 3)$.
	\end{frame}

	\begin{frame}{Systems of Equations - Graphing}
		Solve the system of equations by graphing:
		
		$\begin{cases}
		x - 2y = -4 \\
		y = x^2 - 4x + 3
		\end{cases}$ \pause
		
		This time, we'll notice that there are two points of intersection.
		
		\pause Those points are approximately $(4.27, 4.13)$ and $(0.23, 2.12)$.
	\end{frame}

	\begin{frame}{Systems of Equations - Substitution}
		Graphing can work, but it is not perfect - especially if you don't use technology to do the graphing. \pause
		
		We also have two algebraic methods of solving systems of equations. The first is substitution. \pause
		
		\begin{block}{Method of Substitution}
			\begin{enumerate}[1)]
				\item Solve one of the equations for one variable in terms of the other.
				\item Substitute the expression found in Step 1 into the other equation to obtain an equation in one variable.
				\item Solve the equation.
				\item Back-substitute the value(s) obtained to find the other variable's value.
			\end{enumerate}
		\end{block}
	\end{frame}

	\begin{frame}{Systems of Equations - Substitution}
		Solve the system of equations using substitution:
		
		$\begin{cases}
		2x + y = 9 \\
		3x - 2y = 17
		\end{cases}$ \pause
		
		It looks like the easiest way to implement substitution is to get $y$ by itself in the top equation: $y = -2x + 9$. Now we substitute $-2x + 9$ in for $y$ in the bottom equation and solve for $x$: \pause \begin{flalign*}
		3x - 2(-2x + 9) &= 17 & \\
		7x - 18 &= 17 & \\
		7x &= 35 & \\
		x & = 5
		\end{flalign*} \pause
		So $y = -2(5) + 9 = -1$ and our solution is $(5, -1)$.
	\end{frame} 

	\begin{frame}{Systems of Equations - Substitution}
		Solve the system of equations using substitution:
		
		$\begin{cases}
		3x^2 + 4x - y = 7 \\
		2x - y = -1
		\end{cases}$ 
		
		\onslide<2->{We can isolate $y$ in the bottom equation: $y = 2x + 1$. Now, using substitution into the top equation:} \begin{flalign*}
		\onslide<3->{3x^2 + 4x - (2x + 1) &= 7 & \\}
		\onslide<4->{3x^2 + 2x - 1 &= 7 & \\}
		\onslide<5->{3x^2 + 2x - 8 &= 0 & \\}
		\onslide<6->{(3x - 4)(x+2) &= 0 & \\}
		\onslide<7->{x &= \set{-2, \dfrac{4}{3}}}
		\end{flalign*}
	\end{frame}

	\begin{frame}{Systems of Equations - Substitution}
		So there are two solutions to this equation: \pause
		
		$x = -2 \then y = 2(-2) + 1 = -3$ \pause
		
		$x = \dfrac{4}{3} \then y = 2\fp{\dfrac43} + 1 = \dfrac{11}{3}$
	\end{frame}

	\begin{frame}{Systems of Equations - Substitution}
		Solve the system of equations using substitution:
		
		$\begin{cases}
		2x - y = -3 \\
		2x^2 + 4x - y^2 = 0
		\end{cases}$
		
		\onslide<2->{We can isolate $y$ in the top equation: $y = 2x + 3$. Using substitution:} \begin{flalign*}
		\onslide<3->{2x^2 + 4x - (2x+3)^2 &= 0 & \\}
		\onslide<4->{2x^2 + 4x - (4x^2 + 12x + 9) &= 0 & \\}
		\onslide<5->{-2x^2 - 8x - 9 &= 0 & \\}
		\onslide<6->{x &= \dfrac{8 \pm \sqrt{-8}}{-4}}
		\end{flalign*}
		\onslide<7>{There are no real solutions to this system.}
	\end{frame}

	\begin{frame}{Systems of Equations - Elimination}
		Not every system we come across is going to have an equation that it's easy for us to isolate a variable in. \pause
		
		In this case, we can use elimination. Note that elimination only works when both equations are linear, though. \pause
		
		\begin{block}{Method of Elimination}
			\begin{enumerate}[1)]
				\item Multiply one (or both) of the equations so that when you add the two equations together one of the variables is cancelled out.
				\item Solve the resulting equation.
				\item Back-substitute the value into either of the original equations and find the value of the other variable.
			\end{enumerate}
		\end{block}
	\end{frame}

	\begin{frame}{Systems of Equations - Elimination}
		Solve the system of equations using elimination:
		
		$\begin{cases}
		2x - 4y = -7 \\
		5x + y = -1
		\end{cases}$ \pause
		
		Multiply the bottom equation by $4$ so that we can eliminate $y$. \pause
		
		$\begin{cases}
		2x - 4y = -7 \\
		20x + 4y = -4
		\end{cases}$ \pause
		
		Adding the equations together: $22x = -11 \then x = \dfrac12$ \pause
		
		Back substitute into the top equation to solve for $x$: $2\fp{-\dfrac12} - 4y = -7 \then y = \dfrac32$ \pause
		
		Our solution is $\fp{-\dfrac12, \dfrac32}$.
	\end{frame}

	\begin{frame}{Systems of Linear Equations - Elimination}
		Solve the system of equations using elimination:
		
		$\begin{cases}
		3x + 2y = 7 \\
		2x + 5y = 1
		\end{cases}$ \pause
		
		This time, we'll multiply both equations. Multiply the top equation by $2$ and the bottom equation by $-3$ to eliminate $x$: \pause
		
		$\begin{cases}
		6x + 4y = 14 \\
		-6x - 15y = -3
		\end{cases}$ \pause
		
		Adding the equations together: $-11y = 11 \then y = -1$ \pause
		
		Back-substituting into the original bottom equation: $2x + 5(-1) = 1 \then x = 3$ \pause
		
		Our solution is $(3, -1)$.
	\end{frame}

	\begin{frame}{Systems of Equations - Elimination}
		Solve the system of equations using elimination:
		
		$\begin{cases}
		0.03x + 0.04y = 0.75 \\
		0.02x + 0.06y = 0.90
		\end{cases}$ \pause
		
		First, we can multiply both equations by $100$ to get rid of decimals: \pause
		
		$\begin{cases}
		3x + 4y = 75 \\
		2x + 6y = 90
		\end{cases}$ \pause
		
		Multiply the top equation by $3$ and the bottom equation by $-2$: \pause
		
		$\begin{cases}
		9x + 12y = 225 \\
		-4x - 12y = -180
		\end{cases}$
	\end{frame}

	\begin{frame}{Systems of Equations - Elimination}
		$\begin{cases}
		9x + 12y = 225 \\
		-4x - 12y = -180
		\end{cases}$
		
		Add the two equations together and solve for $x$: $5x = 45 \then x = 9$ \pause
		
		Back-substitute into the equation $3x + 4y = 75$: $3(9) + 4y = 75 \then y = 12$ \pause
		
		Our solution is $(9, 12)$.
	\end{frame}

	\begin{frame}{An Application}
		An airplane flying into a headwind travels the 2000-mile flying distance between Tallahassee, FL and Los Angeles, CA in 4 hours and 24 minutes. On the return flight, the airplane travels the same distance in 4 hours and 6 minutes. Find the airspeed of the plane, and the speed of the wind, assuming both remain constant. \pause
		
		Let $p$ be the airspeed of the plane and $w$ be the speed of the wind. \pause
		
		When traveling against the wind, the plane travels $p - w$ mph. \\
		When traveling with the wind, the plane travels $p + w$ mph. \pause
		
		Using the equation $d = rt$ we can write a system of equations to solve.
	\end{frame}

	\begin{frame}{An Application}
		$\begin{cases}
		2000 = \fp{4 + \dfrac{24}{60}}\fp{p - w} \\
		2000 = \fp{4 + \dfrac{6}{60}}\fp{p + w}
		\end{cases}$ \pause
		
		Simplifying: $\begin{cases}
		2000 &= 4.4(p-w) \\
		2000 &= 4.1(p+w)
		\end{cases} \then \begin{cases}
			454.545454 = p - w \\
			487.804878 = p + w
		\end{cases} \pause$
		
		$\begin{cases}
		454.545454 = p - w \\
		487.804878 = p + w
		\end{cases}$ \pause
		
		Adding the equations together: $2p = 942.3503 \then p \approx 471.18$ mph \pause
		
		Back-substituting: $487.804878 = 471.175166 + w \then w \approx 16.63$ mph
	\end{frame}

	\begin{frame}{Next Steps:}
		\begin{itemize}
			\item Ask questions in Lesson 22 Forum, if you have any
			\item Watch Video Lesson \#23
			\item Complete Assignment \#11
		\end{itemize}
	
		\vfill
		
		Thanks for watching!
	\end{frame}
	
\end{document}