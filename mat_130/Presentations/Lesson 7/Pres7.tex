\documentclass{beamer}
\usepackage[utf8]{inputenc}
\usepackage{amsmath,amsfonts,amsthm,amstext,amssymb, xcolor, tikz, pgf}

% ----------------------------------------------------------
% Theme Setup

% Use Metropolis Theme
\usetheme[numbering=fraction]{metropolis}
\setbeamertemplate{blocks}[rounded][shadow=false]
\makeatletter
\setlength{\metropolis@titleseparator@linewidth}{1pt}
\makeatother

% Define Colors
\definecolor{chargerblue}{HTML}{002764}
\definecolor{chargerred}{HTML}{e02034}
\definecolor{bggray}{HTML}{d0d3d4}

% Set Colors
\setbeamercolor{title}{fg=chargerblue}
\setbeamercolor{background canvas}{bg=white}
\setbeamercolor{title separator}{fg=chargerred}
\setbeamercolor{structure}{fg=chargerblue}
\setbeamercolor{frametitle}{fg=white, bg=chargerblue}
\setbeamercolor*{normal text}{fg=chargerblue}
\setbeamercolor*{block body}{bg=bggray}
\setbeamercolor*{block title}{bg=chargerblue, fg=white}
% ----------------------------------------------------------

% ----------------------------------------------------------
% Custom Definitions, Commands, Environments, etc.

% Sets of numbers
\def\R{\mathbb{R}} % The reals
\def\N{\mathbb{N}} % The naturals
\def\Z{\mathbb{Z}} % The integers
\def\Q{\mathbb{Q}} % The rationals

% Blank space
\newcommand{\blank}[1]{\underline{\hspace{#1}}} % Blank space

% Fitted inclusion symbols
\newcommand{\fp}[1]{\left({#1}\right)} % Fitted parentheses around content
\newcommand{\fb}[1]{\left[{#1}\right]} % Fitted brackets
\newcommand{\set}[1]{\left\{{#1}\right\}} % Fitted braces (useful for sets)
\newcommand{\av}[1]{\left|{#1}\right|} % Fitted absolute value bars



% Coordinate Plane (Four-Quadrant)
\def\coordplane {
	\begin{tikzpicture}		\draw[step=0.25cm,black,very thin,opacity=0.25] (-2.5cm, -2.5cm) grid (2.5cm, 2.5cm);
		\draw[<->,thick,black] (-2.5cm, 0) -- (2.5cm, 0) node[anchor=north west,pos=0.94,font=\scriptsize]{$x$};
		\draw[<->,thick,black] (0,-2.5cm) -- (0, 2.5cm) node[anchor=south east,font=\scriptsize,pos=0.94]{$y$};
	\end{tikzpicture}
}

% Coordinate Plane (One-Quadrant)
\def\onequad {
	\begin{tikzpicture}
		\draw[step=0.25cm, black, very thin, opacity=0.25] (0,0) grid (7.5cm,5cm);
		\draw[->, thick, black] (0,0) -- (7.5cm, 0) node[anchor=north west,font=\scriptsize,pos=0.94]{$x$};
		\draw[->, black, thick] (0,0) -- (0,5cm) node[anchor=south east,font=\scriptsize,pos=0.94]{$y$};
	\end{tikzpicture}
}

% Font Colors
\newcommand{\cyan}[1]{{\color{cyan}{#1}}} % Changes font to cyan
\newcommand{\red}[1]{{\color{red}{#1}}} % Changes font to red
\newcommand{\magenta}[1]{{\color{magenta}{#1}}} % Changes font to magenta
\newcommand{\orange}[1]{{\color{orange}{#1}}} % Changes font to orange
\newcommand{\yellow}[1]{{\color{yellow}{#1}}} % Changes font to yellow
\newcommand{\violet}[1]{{\color{violet}{#1}}} % Changes font to violet
\newcommand{\green}[1]{{\color{green}{#1}}} % Changes font to green
\newcommand{\blue}[1]{{\color{blue}{#1}}} % Changes font to blue
\newcommand{\white}[1]{{\color{white}{#1}}} % Changes font to white
% ----------------------------------------------------------

% ----------------------------------------------------------
% Presentation Information 
\title[1.7 and 1.8]{Linear and Polynomial Inequalities}
\subtitle{Sections 1.7 and 1.8}
\author{Jacob Ayers}
\institute{Lesson \#7}
\date{MAT 130}
% ----------------------------------------------------------

\begin{document}

% Slide 1 (Title Slide)
\begin{frame}
\titlepage
\end{frame}

% Slide 2 (Objectives)
\begin{frame}[t]{Objectives}
\begin{itemize}
	\item Solve linear inequalities in one variable
	\item Solve polynomial inequalities
	\item Solve applied problems involving inequalities
\end{itemize}
\end{frame}

\begin{frame}[t]{Review: Intervals and Inequalities}
Recall: we write intervals using brackets (to represent possible equality) and parentheses (to represent strict inequality)

\pause

Write $[-1,3]$ as an inequality in $x$. \\ \pause
$-1 \leq x \leq 3$ \vspace{8pt}

\pause

Write $(-\infty, 6)$ as an inequality in $x$. \\ \pause
$x < 6$ \vspace{8pt}

\pause

Write $x \geq -4$ as an interval. \\ \pause
$[-4, \infty)$ \vspace{8pt}

\pause

Write $6 \leq x < 9$ as an interval. \\ \pause
$[6, 9)$
\end{frame}

\begin{frame}[t]{Properties of Inequalities}
Inequalities have many of the same properties as the equal sign, but there is one key difference.

\begin{block}{Properties of Inequalities}
\begin{enumerate}[1)]
\item Transitive Property: If $a < b$ and $b < c$, then $a < c$
\item Addition of Inequalities: If $a < b$ and $c < d$, then $a + c < b + d$
\item Addition of a Constant: If $a < b$, then $a + c < b + c$ for any $c \in \R$
\red{
\item Multiplication of a Constant: If $a < b$ and $c \in \R$, then \begin{itemize}
\red{\item $ac < bc$ if $c > 0$}
\red{\item $ac > bc$ if $c < 0$}
\end{itemize}
}
\end{enumerate}
\end{block}
\end{frame}

\begin{frame}[t]{Solving Linear Inequalities}
What those properties tell us is:

We can solve linear inequalities just like we did linear equations, except we must change the direction of the sign when we multiply (or divide) by a negative number.

\pause

Solve for $x$: $3(2 + x) \leq 2x - 9$
\pause
\begin{flalign*}
3(2 + x) &\leq 2x - 9 & \\
6 + 3x &\leq 2x - 9 & \\
6 + x &\leq -9 & \\
x &\leq -15
\end{flalign*}
\end{frame}

\begin{frame}[t]{Solving Linear Inequalities}
Solve for $x$: $2 - \dfrac53 x > x - 6$

\begin{flalign*}
\onslide<2->{2 - \dfrac53 x &> x - 6 & \\}
\onslide<3->{-\dfrac53 x &> x - 8 & \\}
\onslide<4->{-\dfrac83 x &> 8 & \\}
\onslide<5>{x &< 3 &}
\end{flalign*}
\end{frame}

\begin{frame}[t]{Solving Linear Inequalities Graphically}
We can also solve inequalities using a graphing calculator or a graphing utility.

I've posted brief supplemental videos showing you how to do this with the Casio fx-9750 GII, the TI 83/84 family, and GeoGebra.
\end{frame}

\begin{frame}[t]{Double Inequalities}
A double inequality is an inequality of the form $a < x < b$ (note: either sign could be $\leq$)

We solve double inequalities using the same methods as before, but we do everything to all three components of the inequality.

\onslide<2->{Solve for $x$: $-12 < 3x - 9 \leq 3$}
\begin{flalign*}
\onslide<3->{-12 &< 3x - 9 \leq 3 & \\}
\onslide<4->{-3 &< 3x \leq 12 & \\}
\onslide<5->{-1 &< x \leq 4}
\end{flalign*}
\end{frame}

\begin{frame}[t]{An Application of Linear Inequalities}
Yennefer has a choice of two internet providers (assume that they provide the same quality of service). Provider A will only charge \$20 for installation because her new home is already wired for their service, and they charge \$90 per month. Provider B is new to town and will have to wire the home, so they are charging \$300 for installation. But they only charge \$50 per month. After how long will it be a better deal to go with Provider B?

\pause

Cost Models: \\
Provider A: $90m + 20$ \\
Provider B: $50m + 300$
\end{frame}

\begin{frame}[t]{An Application of Linear Inequalities}
We want Provider B to be cheaper than Provider A. In other words,

\pause

Cost of Provider B < Cost of Provider A
\pause
\begin{flalign*}
50m + 300 &< 90m + 20 & \\
-40m &< -280 & \\
m &> 7
\end{flalign*}
\pause
So after 8 months, Provider B will be a better deal.
\end{frame}

\begin{frame}[t]{Polynomial Inequalities}
To solve polynomials, we make use of the fact that a polynomial can only change signs at its zeros.

\begin{block}{Test Intervals for a Polynomial Inequality}
\begin{enumerate}[1)]
\item Find all real zeros of the polynomial, and arrange them in increasing order. We will call these values \textit{key values}.
\item Use the key numbers to determine the test intervals.
\item Pick one value in each test interval and evaluate the polynomial at that value. When the value of the polynomial is negative, the polynomial has negative values for every $x$-value in that interval. When the value of the polynomial is negative, the polynomial has negative values for every $x$-value in that interval.
\end{enumerate}
\end{block}
\end{frame}

\begin{frame}[t]{Solving Polynomial Inequalities}
Solve the inequality $x^2 - x - 20 < 0$

\pause

First, find the zeros of the polynomial: \\
$x^2 - x - 20 = (x-5)(x+4) \Rightarrow x = \set{-4, 5}$

\pause

Next, write the test intervals: \\
Interval 1: $(-\infty, -4)$ \\
Interval 2: $(-4, 5)$ \\
Interval 3: $(5, \infty)$

\pause

Now, pick one $x$-value in each interval to plug in: \\
\pause Interval 1: $x = -5; \; \text{ } (-5)^2 - (-5) - 20 = 10$ \\
\pause Interval 2: $x = 0; \; \text{ } 0^2 - 0 - 20 = -20$ \\
\pause Interval 3: $x = 6; \; \text{ } 6^2 - 6 - 20 = 10$

\pause

Conclusion: The solution to the inequality $x^2 - x - 20 < 0$ is $(-4, 5)$.
\end{frame}

\begin{frame}[t]{Solving Polynomial Inequalities}
Solve $3x^3 - x^2 - 12x > -4$

\onslide<2->{First, we need to make sure the right-hand side is zero: $3x^3 - x^2 - 12x + 4 > 0$}

\onslide<3->{With that out of the way, we'll need to find our key values:}
\begin{flalign*}
\onslide<4->{3x^3 - x^2 - 12x + 4 &= 0 & \\}
\onslide<5->{x^2(3x - 1) - 4(3x - 1) &= 0 & \\}
\onslide<6->{\fp{x^2 - 4}(3x - 1) &= 0 & \\}
\onslide<7->{(x+2)(x-2)(3x-1) &= 0 & \\}
\onslide<8>{x &= \set{-2, \dfrac13, 2}}
\end{flalign*}
\end{frame}

\begin{frame}[t]{Solving Polynomial Inequalities}
Solve for $x$: $3x^3 - x^2 - 12x + 4 > 0$

Next, we write our test intervals; there are four this time.
$(-\infty, -2)$ \hspace{0.5in} $\fp{-2, \dfrac13}$ \hspace{0.5in} $\fp{\dfrac13, 2}$ \hspace{0.5in} $(2, \infty)$

\pause

Choose one value in each and plug it in to the polynomial:
\pause Interval 1: $-3; \; \text{ } 3(-3)^3 - (-3)^2 - 12(-3) + 4 = -50$ \\
\pause Interval 2: $0; \; \text{ } 3(0)^3 - 0^2 - 12(0) + 4 = 4$ \\
\pause Interval 3: $1; \; \text{ } 3(1)^3 - 1^2 - 12(1) + 4 = -6$ \\
\pause Interval 4: $3; \; \text{ } 3(3)^3 - 3^2 - 12(3) + 4 = 40$

\pause

So the solution to the inequality $3x^3 - x^2 - 12x > -4$ is $\fp{-2, \dfrac13} \cup (2, \infty)$
\end{frame}

\begin{frame}[t]{Solving Polynomial Inequalities - Unusual Solution Set}
Solve for $x$: $x^2 + 6x + 9 < 0$

\pause

First, find key values: \\
$x^2 + 6x + 9 = (x+3)^2 \Rightarrow x = -3$

\pause

Next, find test intervals: \\
$(-\infty, -3)$ and $(-3, \infty)$

\pause

Pick one value in each interval and plug it in: \\
\pause Interval 1: $-4; \; \text{ } (-4)^2 + 6(-4) + 9 = 1$ \\
\pause Interval 2: $0; \; \text{ } 0^2 + 6(0) + 9 = 9$

\pause

This polynomial is never negative. We write $\emptyset$ to indicate that there is no real solution to the inequality.
\end{frame}

\begin{frame}[t]{Application - Finding Domain}
Find the domain of $\sqrt{x^2 - 7x + 10}$.

\pause

Recall that this expression will only be defined when the expression under the square root is at least 0. We must solve the inequality $x^2 - 7x + 10 \geq 0$.

\pause

Key Values: $x = 2$ and $x = 5$ (you can confirm by factoring)

\pause

Test Intervals: $(-\infty, 2)$, $(2, 5)$, $(5, \infty)$

Chosen Values and Results: $0 \Rightarrow 10$, $3 \Rightarrow -2$, $6 \Rightarrow 4$

Conclusion: The domain of the expression is $(-\infty, 2] \cup [5, \infty)$. We used brackets in this case because the inequality we were asked to solve wasn't strict.
\end{frame}

\begin{frame}[t]{Next Steps}
\begin{itemize}
\item Post questions in the Lesson 7 Forum, if you have any
\item Read 2.1 and 2.2
\item Watch Video Lesson \#6
\item Complete Assignment \#3
\end{itemize}

\vfill

Thanks for watching!
\end{frame}

\end{document}