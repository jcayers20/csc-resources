\documentclass[t, aspectratio=169]{beamer}
\usepackage{amsmath,amsfonts,amsthm,amstext,amssymb, xcolor, tikz, pgf}

% ----------------------------------------------------------
% Theme Setup

% Use Metropolis Theme
\usetheme[numbering=fraction]{metropolis}
\setbeamertemplate{blocks}[rounded][shadow=false]
\makeatletter
\setlength{\metropolis@titleseparator@linewidth}{1pt}
\makeatother

% Define Colors
\definecolor{chargerblue}{HTML}{002764}
\definecolor{chargerred}{HTML}{e02034}
\definecolor{bggray}{HTML}{d0d3d4}

% Set Colors
\setbeamercolor{title}{fg=chargerblue}
\setbeamercolor{background canvas}{bg=white}
\setbeamercolor{title separator}{fg=chargerred}
\setbeamercolor{structure}{fg=chargerblue}
\setbeamercolor{frametitle}{fg=white, bg=chargerblue}
\setbeamercolor*{normal text}{fg=chargerblue}
\setbeamercolor*{block body}{bg=bggray}
\setbeamercolor*{block title}{bg=chargerblue, fg=white}
% ----------------------------------------------------------

% ----------------------------------------------------------
% Custom Definitions, Commands, Environments, etc.

% Sets of numbers
\def\R{\mathbb{R}} % The reals
\def\N{\mathbb{N}} % The naturals
\def\Z{\mathbb{Z}} % The integers
\def\Q{\mathbb{Q}} % The rationals

% Blank space
\newcommand{\blank}[1]{\underline{\hspace{#1}}} % Blank space

% Fitted inclusion symbols
\newcommand{\fp}[1]{\left({#1}\right)} % Fitted parentheses around content
\newcommand{\fb}[1]{\left[{#1}\right]} % Fitted brackets
\newcommand{\set}[1]{\left\{{#1}\right\}} % Fitted braces (useful for sets)
\newcommand{\av}[1]{\left|{#1}\right|} % Fitted absolute value bars



% Coordinate Plane (Four-Quadrant)
\def\coordplane {
	\begin{tikzpicture}		\draw[step=0.25cm,black,very thin,opacity=0.25] (-2.5cm, -2.5cm) grid (2.5cm, 2.5cm);
	\draw[<->,thick,black] (-2.5cm, 0) -- (2.5cm, 0) node[anchor=north west,pos=0.94,font=\scriptsize]{$x$};
	\draw[<->,thick,black] (0,-2.5cm) -- (0, 2.5cm) node[anchor=south east,font=\scriptsize,pos=0.94]{$y$};
	\end{tikzpicture}
}

% Coordinate Plane (One-Quadrant)
\def\onequad {
	\begin{tikzpicture}
	\draw[step=0.25cm, black, very thin, opacity=0.25] (0,0) grid (7.5cm,5cm);
	\draw[->, thick, black] (0,0) -- (7.5cm, 0) node[anchor=north west,font=\scriptsize,pos=0.94]{$x$};
	\draw[->, black, thick] (0,0) -- (0,5cm) node[anchor=south east,font=\scriptsize,pos=0.94]{$y$};
	\end{tikzpicture}
}
% ----------------------------------------------------------

% ----------------------------------------------------------
% Presentation Information 
\title[5.1]{Exponential Functions and Their Graphs}
\subtitle{Section 5.1}
\author{Jacob Ayers}
\institute{Lesson \#17}
\date{MAT 130}
% ----------------------------------------------------------

\begin{document}
	
	% Slide 1 (Title Slide)
	\begin{frame}
		\titlepage
	\end{frame}
	
	% Slide 2 (Objectives)
	\begin{frame}{Objectives}
		\begin{itemize}
			\item Evaluate exponential expressions and functions
			\item Graph exponential functions
			\item Solve exponential equations using the One-to-One Property
			\item Solve problems involving the natural base $e$
			\item Solve applications of exponential functions
		\end{itemize}
	\end{frame}

	\begin{frame}{Exponential Functions}
		\begin{block}{Definition}
			The \textit{exponential function $f$ with base $a$} is denoted by $$f(x) = a^x$$ where $a > 0$, $a \neq 1$, and $x \in \R$
		\end{block} \pause
	
		\begin{columns}
			\begin{column}{0.5\textwidth}
				Example: Use a calculator to evaluate each expression at the given value of $x$: \begin{enumerate}[a)]
					\item $f(x)=2^x$ at $x = -2.3$
					\item $f(x)=2^{-x}$ at $x = 4.92$
					\item $f(x)=0.6^x$ at $x = -\pi$
				\end{enumerate} 
			\end{column} \pause
			\begin{column}{0.5\textwidth}
				\begin{enumerate}[a)]
					\item $f(-2.3) \approx 0.2031$
					\item $f(4.92) \approx 0.03303$
					\item $f(-\pi) \approx 4.9769$
				\end{enumerate}
			\end{column}
		\end{columns}
	\end{frame}

	\begin{frame}{Graphs of Exponential Functions}
		We graph exponential functions using the Point-Plotting Method. \pause
		
		Example: Graph $f(x) = 3^x$ and $g(x) = 9^x$ on the same coordinate plane. \pause
		
		\begin{tabular}{c|cccccc}
			$x$ & $-3$ & $-2$ & $-1$ & $0$ & $1$ & $2$ \\ \hline
			$f(x)$ & $\frac{1}{27}$ & $\frac{1}{9}$ & $\frac{1}{3}$ & $1$ & $3$ & $9$ \\ \hline
			$g(x)$ & $\frac{1}{729}$ & $\frac{1}{81}$ & $\frac{1}{9}$ & $1$ & $9$ & $81$  
		\end{tabular} \pause
	
		These points will be enough for us to draw accurate graphs.
	\end{frame}

	\begin{frame}{Graphs of Exponential Functions}
		\includegraphics[width=4in]{Exp1.png}
	\end{frame}

	\begin{frame}{Graphs of Exponential Functions}
		Graph $f(x) = 3^{-x}$ and $g(x) = 9^{-x}$ on the same coordinate plane. \pause
		
		\begin{tabular}{c|cccccc}
			$x$ & $3$ & $2$ & $1$ & $0$ & $-1$ & $-2$ \\ \hline
			$f(x)$ & $\frac{1}{27}$ & $\frac{1}{9}$ & $\frac{1}{3}$ & $1$ & $3$ & $9$ \\ \hline
			$g(x)$ & $\frac{1}{729}$ & $\frac{1}{81}$ & $\frac{1}{9}$ & $1$ & $9$ & $81$  
		\end{tabular} \pause
	\end{frame}

	\begin{frame}{Graphs of Exponential Functions}
		\includegraphics[width=4in]{Exp2.png}
	\end{frame}

	\begin{frame}{Solving Basic Exponential Equations}
		\begin{block}{One-To-One Property of Exponents}
			For all $a > 0$ and $a \neq 1$, $a^x = a^y$ if and only if $x = y$.
		\end{block} \pause
	
		We can use this property to solve simple exponential equations. \pause
		
		Example: Solve for $x$: $3^{7x - 4} = 27$ \pause
		
		The important thing to notice here is that $27 = 3^3$. \pause \begin{flalign*}
		3^{7x - 4} &= 27 & \\
		3^{7x - 4} &= 3^3 & \\
		7x - 4 &= 3 & \\
		x &= 1
		\end{flalign*}
	\end{frame}

	\begin{frame}{Solving Basic Exponential Equations}
		Solve for $x$: $\fp{\dfrac14}^{-x} =  16$ \pause
		
		We can use properties of exponents to show that $\fp{\dfrac14}^{-x} = 4^x$ and we also know that $16 = 4^2$. \pause \begin{flalign*}
		\fp{\dfrac14}^{-x} &= 16 & \\
		4^x &= 4^2 & \\
		x &= 2
		\end{flalign*}
	\end{frame}

	\begin{frame}{The Natural Base $e$}
		The number $e \approx 2.71828...$ is called the natural base, and it is used in many applications. \pause
		
		$e$ is an irrational constant, much like $\pi$. Your calculator should have an $e$ and/or an $e^x$ button. \pause
		
		The exact value of $e$ is the limit of $\fp{1 + \dfrac{1}{m}}^m$ as $m$ increases without bound (you'll learn more about limits in Calculus). \pause
		
		Use a calculator to evaluate $f(x) = e^x$ at $x = 0.3$, $x = -1.2$, and $x = 6.2$ \pause
		
		$f(0.3) \approx 1.3499$ \\
		$f(-1.2) \approx 0.3012$ \\
		$f(6.2) \approx 492.7490$
	\end{frame}

	\begin{frame}{Applications of Exponential Functions}
		One of the most common applications of exponential functions is compound interest. \pause
		
		\begin{block}{Compound Interest}
			After $t$ years, the balance $A$ in an account with principal $P$ and annual interest rate $r$ (as a decimal) is given by one of these two formulas: \begin{enumerate}[1)]
				\item Compounded $n$ times per year: $A = P\fp{1 + \dfrac{r}{n}}^{nt}$ \\
				\item Compounded continuously: $A = Pe^{rt}$
			\end{enumerate} \pause
		\end{block}
	\end{frame}

	\begin{frame}{Applications of Exponential Functions}
		Some common values of $n$ are: \\
		``compounded annually": $n = 1$ \\
		``compounded semi-annually": $n = 2$ \\
		``compounded quarterly": $n = 4$ \\
		``compounded monthly": $n = 12$ \\
		``compounded daily": $n = 365$
	\end{frame}

	\begin{frame}{Applications of Exponential Functions}
		Jim places \$50,000 into a retirement account earning 8\%, compounded quarterly. How much will the account be worth when Jim retires in 35 years? \pause
		
		This is a compound interest problem with $P = 50000$, $r = 0.08$, $n = 4$, and $t = 35$ \pause \begin{flalign*}
		A &= P\fp{1+\dfrac{r}{n}}^{nt} & \\
		&= 50000\fp{1 + \dfrac{0.08}{4}}^{4(35)} & \\
		&= 50000(1.02)^{140} & \\
		&\approx \$799,823.30
		\end{flalign*}
	\end{frame}

	\begin{frame}{Applications of Exponential Functions}
		At the time of her graduation, Hailey has student loan debt totaling \$86,719.38 at an interest rate of 6.75\%, compounded continuously. She is not required to make any payment on the loan for six months. How much interest will she be charged during those six months? \pause
		
		To solve this continuously compounding interest problem, we'll find Hailey's balance after six month $t = 0.5$ and subtract her current balance from her new balance to figure out how much interest she was charged.
	\end{frame}

	\begin{frame}{Applications of Exponential Functions}
		$t = 0.5$, $r = 0.0725$, $P = 86719.38$ \pause \begin{flalign*}
		A &= Pe^{rt} & \\
		&= 86719.38e^{0.0725*0.5} & \\
		&\approx \$89920.63
		\end{flalign*} \pause
		Hailey was charged $\$89920.63 - \$86719.38 = \$3,201.25$ in interest.
	\end{frame}

	\begin{frame}{Applications of Exponential Functions}
		In 1986, a nuclear reactor accident occurred in Chernobyl in what was then the Soviet Union. The explosion spread highly toxic radioactive chemicals, such as plutonium $\fp{^{239}\text{Pu}}$, over hundreds of square miles, and the government evacuated the city and the surrounding area. To see why the city is now uninhabited, consider the model $$P = 10\fp{\dfrac{1}{2}}^{t/24000}$$ which represents the amount of Plutonium $P$ that remains (from an initial amount of 10 pounds) after $t$ years. How much of the 10 pounds will remain after 100 years? After 125,000 years?
	\end{frame}

	\begin{frame}{Applications of Exponential Functions}
		$P(t) = 10\fp{\dfrac{1}{2}}^{t/24100}$ \pause
		
		We need to find $P(100)$ and $P(125000)$.
		
		\begin{columns}
			\begin{column}{0.5\textwidth}
				\begin{flalign*}
				P(100) &= 10(0.5)^{100/24100} & \\
				&\approx 9.9713 \text{ lb}
				\end{flalign*}
			\end{column} \pause
			\begin{column}{0.5\textwidth}
				\begin{flalign*}
				P(125000) &= 10(0.5)^{125000/24100} & \\
				&\approx 0.2746 \text{ lb}
				\end{flalign*}
			\end{column}
		\end{columns} \pause \vspace{12pt}
	
		After 100 years, there will be about 9.97 pounds of plutonium, and after 125,000 years there will be about 0.27 pounds of plutonium.
	\end{frame}

	\begin{frame}{Next Steps}
		\begin{itemize}
			\item Ask questions in Lesson 17 Forum, if you have any
			\item Complete Assignment \#8
			\item Begin Module \#10
			\begin{itemize}
				\item Read 5.2
				\item Watch Video Lesson \#18
			\end{itemize}
		\end{itemize}
	
		\vfill
		
		Thanks for watching!
	\end{frame}
	
\end{document}