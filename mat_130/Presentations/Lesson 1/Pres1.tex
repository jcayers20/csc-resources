\documentclass{beamer}
\usepackage[utf8]{inputenc}
\usepackage{amsmath,amsfonts,amsthm,amstext,amssymb,xcolor,tikz,pgf}

% ----------------------------------------------------------
% Theme Setup

% Use Metropolis Theme
\usetheme[numbering=fraction]{metropolis}
\setbeamertemplate{blocks}[rounded][shadow=false]
\makeatletter
\setlength{\metropolis@titleseparator@linewidth}{1pt}
\makeatother

% Define Colors
\definecolor{chargerblue}{HTML}{002764}
\definecolor{chargerred}{HTML}{e02034}
\definecolor{bggray}{HTML}{d0d3d4}

% Set Colors
\setbeamercolor{title}{fg=chargerblue}
\setbeamercolor{background canvas}{bg=white}
\setbeamercolor{title separator}{fg=chargerred}
\setbeamercolor{structure}{fg=chargerblue}
\setbeamercolor{frametitle}{fg=white, bg=chargerblue}
\setbeamercolor*{normal text}{fg=chargerblue}
\setbeamercolor*{block body}{bg=bggray}
\setbeamercolor*{block title}{bg=chargerblue, fg=white}
% ----------------------------------------------------------

% ----------------------------------------------------------
% Custom Definitions, Commands, Environments, etc.

% Sets of numbers
\def\R{\mathbb{R}} % The reals
\def\N{\mathbb{N}} % The naturals
\def\Z{\mathbb{Z}} % The integers
\def\Q{\mathbb{Q}} % The rationals

% Blank space
\newcommand{\blank}[1]{\underline{\hspace{#1}}} % Blank space

% Fitted inclusion symbols
\newcommand{\fp}[1]{\left({#1}\right)} % Fitted parentheses around content
\newcommand{\fb}[1]{\left[{#1}\right]} % Fitted brackets
\newcommand{\set}[1]{\left\{{#1}\right\}} % Fitted braces (useful for sets)
\newcommand{\av}[1]{\left|{#1}\right|} % Fitted absolute value bars



% Coordinate Plane (Four-Quadrant)
\def\coordplane {
	\begin{tikzpicture}		\draw[step=0.25cm,black,very thin,opacity=0.25] (-2.5cm, -2.5cm) grid (2.5cm, 2.5cm);
		\draw[<->,thick,black] (-2.5cm, 0) -- (2.5cm, 0) node[anchor=north west,pos=0.94,font=\scriptsize]{$x$};
		\draw[<->,thick,black] (0,-2.5cm) -- (0, 2.5cm) node[anchor=south east,font=\scriptsize,pos=0.94]{$y$};
	\end{tikzpicture}
}

% Coordinate Plane (One-Quadrant)
\def\onequad {
	\begin{tikzpicture}
		\draw[step=0.25cm, black, very thin, opacity=0.25] (0,0) grid (7.5cm,5cm);
		\draw[->, thick, black] (0,0) -- (7.5cm, 0) node[anchor=north west,font=\scriptsize,pos=0.94]{$x$};
		\draw[->, black, thick] (0,0) -- (0,5cm) node[anchor=south east,font=\scriptsize,pos=0.94]{$y$};
	\end{tikzpicture}
}
% ----------------------------------------------------------

% ----------------------------------------------------------
% Presentation Information 
\title[P.1 and P.2]{Review of Real Numbers; Exponents and Radicals}
\subtitle{Sections P.1 and P.2}
\author{Jacob Ayers}
\institute{Lesson \#1}
\date{MAT 130}
% ----------------------------------------------------------

\begin{document}

% Slide 1 (Title Slide)
\begin{frame}
\titlepage
\end{frame}

% Slide 2 (Objectives)
\begin{frame}[t]{Objectives}
\begin{itemize}
	\item Classify Real Numbers
	\item Interpret Inequalities and Intervals
	\item Evaluate Algebraic Expressions
	\item Use Properties of Exponents
	\item Evaluate Exponential Equations
	\item Use Properties of Radicals
	\item Rationalize Denominators and Numerators
	\item Evaluate Rational Exponents
\end{itemize}
\end{frame}

\begin{frame}[t]{Subsets of the Real Numbers}
The \textit{real numbers} are all numbers that are not complex (i.e. do not have an imaginary part)

There are several important subsets of the real numbers: \vspace{-6pt} \begin{itemize}
\item<2-> The Natural Numbers: $\N = \set{1,2,3,\dots}$
\item<3-> The Whole Numbers: $\set{0,1,2,3,\dots}$
\item<4-> The Integers: $\Z = \set{\dots, -3, -2, -1, 0, 1, 2, 3, \dots}$
\item<5-> The Rational Numbers: $\Q = \set{\dfrac{a}{b} : a,b\in\Z \text{ and } b\neq 0}$
\item<6> The Irrational Numbers: any number that can't be written as a fraction of integers
\end{itemize}
\end{frame}

\begin{frame}[t]{Classifying Real Numbers}
Determine which numbers in the set $$\set{-\pi, -\dfrac14, \dfrac63, \dfrac12\sqrt{2}, -7.5, -1, 8, -22}$$ are (a) Natural Numbers, (b) Whole Numbers, (c) Integers, (d) Rational Numbers, and (e) Irrational Numbers.
\begin{enumerate}[(a)]
\item<2-> $\set{\dfrac63,8}$
\item<3-> $\set{\dfrac63,8}$
\item<4-> $\set{\dfrac63,8,-1,-22}$
\item<5-> $\set{\dfrac63,8,-1,-22,-\dfrac14,-7.5}$
\item<6> $\set{-\pi,\dfrac12\sqrt{2}}$
\end{enumerate}
\end{frame}

\begin{frame}[t]{Inequalities}
\begin{block}{Definition}
\emph{Order:} If $a,b\in\R$, then $a$ is less than $b$ when $b - a$ is positive. The \textit{inequality} $a < b$ denotes the \textit{order} of $a$ and $b$. Other inequality symbols include $a > b$ (greater than), $a \leq b$ (less than/equal to), and $a \geq b$ (greater than/equal to).
\end{block}

Describe the subset of real numbers represented by $x > -3$ \\
\onslide<2->{This inequality represents all real numbers that are larger than $-3$.}

\onslide<3->{Describe the subset of real numbers represented by $0<x\leq 4$} \\
\onslide<4>{This inequality represents all real numbers that are larger than $0$, but at most $4$.}
\end{frame}

\begin{frame}[t]{Intervals}
Another way of representing subsets of the real numbers is to use \textit{interval notation}.

In interval notation, we use brackets and parentheses rather than inequality symbols. A bracket indicates the possibility of equality (like $\leq / \geq$) while a parenthesis represents strict inequality (like $< / >$)

An \textit{unbounded interval} is an interval that includes infinity.
\end{frame}

\begin{frame}[t]{Intervals}
Write as an interval: $-3\leq x < 4$ \\
\onslide<2->{$[-3,4)$} 

\vspace{24pt}

Write as an interval: $x \geq 12$ \\
\onslide<3->{$[12,\infty)$}

\vspace{24pt}

Verbally describe the interval $(-2,5]$. \\
\onslide<4>{This interval includes all numbers greater than $-2$ and at most $5$.}
\end{frame}

\begin{frame}[t]{Absolute Value}
The \textit{absolute value} of a number is its distance from zero on the number line.

Note that the absolute value of any real number is nonnegative.

Examples: \\
$\av{14} = 14$ \\
$\av{-14} = 14$ \\
$-\av{-14} = -14$
\end{frame}

\begin{frame}[t]{Absolute Value}
Evaluate $\dfrac{\av{x+3}}{x+3}$ for (a) $x > -3$ and (b) $x < -3$.

\begin{enumerate}[(a)]
\item<2-> If $x > -3$, then $x + 3 > 0$. Hence, $\dfrac{\av{x+3}}{x+3} = \dfrac{x+3}{x+3} = 1$.
\item<3> If $x < -3$, then $x + 3 < 0$. Hence, $\dfrac{\av{x+3}}{x+3} = \dfrac{-\fp{x+3}}{x+3} = -1$.
\end{enumerate}
\end{frame}

\begin{frame}[t]{Algebraic Expressions}
An \textit{algebraic expression} is an expression that includes one or more \textit{variables}.

A \textit{variable} is a letter that represents a number.

To evaluate an algebraic expression, substitute in the value(s) of the variables and use order of operations to simplify.
\end{frame}

\begin{frame}[t]{Algebraic Expressions}
Evaluate $4x - 5$ when $x = 6$. \vspace{-6pt}
\onslide<2->{
\begin{flalign*}
4x - 5 &= 4(6) - 5 &\\ &= 24 - 5 &\\ &= 19 &
\end{flalign*}
}

Evaluate $\dfrac{2x}{x+1}$ when $x = -2$
\onslide<3>{
\begin{flalign*}
\dfrac{2x}{x+1} &= \dfrac{2(-2)}{-2 + 1} & \\ &= \dfrac{-4}{-1} & \\ &= 4
\end{flalign*}
}
\end{frame}

\begin{frame}[t]{Evaluating Exponential Expressions}
\begin{block}{Definition}
If $a \in \R$ and $n$ is a positive integer, then $$a^n = a\cdot a \cdot a \cdots \cdot a \; \text{ ($n$ times)}$$ where $n$ is the \textit{exponent} and $a$ is the \textit{base}.
\end{block}
\end{frame}

\begin{frame}[t]{Evaluating Exponential Expressions}
Evaluate each expression: \begin{enumerate}[(a)]
\item $-3^4$
\item $(-3)^4$
\item $3^2\cdot 3$
\item $\dfrac{3^5}{3^8}$
\end{enumerate}

\pause
\begin{enumerate}[(a)]
\item $-(3\cdot 3\cdot 3\cdot 3) = -81$ \pause
\item $(-3)(-3)(-3)(-3) = 81$ \pause
\item $3^{2+1} = 27$ \pause
\item $\dfrac{1}{3^{8-5}} = \dfrac{1}{27}$
\end{enumerate}

\end{frame}

\begin{frame}[t]{Using Properties of Exponents}
Use properties of exponents to simplify each expression: \begin{enumerate}[(a)]
\item $\fp{2x^{-2}y^3}\fp{-x^4y}$
\item $\fp{\dfrac{3x^4}{x^2y^2}}^2$
\end{enumerate}

\vspace{12pt}

\pause
\begin{enumerate}[(a)]
\item $\fp{2x^{-2}y^3}\fp{-x^4y} = (2)(-1)(x^{-2})(x^4)(y^3)(y) = -2x^2y^4$ \pause
\item $\fp{\dfrac{3x^4}{x^2y^2}}^2 = \fp{\dfrac{3x^2}{y^2}}^2 = \dfrac{3^2(x^2)^2}{(y^2)^2} = \dfrac{9x^4}{y^4}$
\end{enumerate}

\end{frame}

\begin{frame}[t]{Radicals and Their Properties}
\begin{block}{Definition}
Let $a,b\in\R$ and $n\geq 2 \in \Z$. If $$a = b^n$$ then $b$ is an \textit{$n$th root of $a$}. If $n = 2$ then we call it a square root, and if $n =3$, we call it a cube root.
\end{block}

When evaluating roots, the fundamental question is: What number times itself $n$ times gives $a$?

\pause
Examples: \\
$\sqrt{9} = 3$ \\ \pause
$\sqrt[4]{-16}$ has no real solution. \\ \pause
$\sqrt[3]{-343} = -7$

\end{frame}

\begin{frame}[t]{Using Properties of Radicals}
Use properties of exponents to simplify each expression:
\begin{enumerate}[(a)]
\item $\dfrac{\sqrt{125}}{\sqrt{5}}$
\item $\sqrt[3]{125^2}$
\item $\sqrt{\sqrt{x}}$
\end{enumerate}

\pause
\begin{enumerate}[(a)]
\item $\dfrac{\sqrt{125}}{\sqrt{5}} = \sqrt{\dfrac{125}{5}} = \sqrt{25} = 5$ \pause
\item $\sqrt[3]{125^2} = \fp{\sqrt[3]{125}}^2 = 5^2 = 25$ \pause
\item $\sqrt{\sqrt{x}} = \sqrt[2\cdot 2]{x} = \sqrt[4]{x}$
\end{enumerate}

\end{frame}

\begin{frame}[t]{Simplifying Radical Expressions}
A radical is in \textit{simplest form} when: \begin{enumerate}
\item All possible factors are removed from the radical.
\item There are no radicals in the denominator of a fraction.
\item The index of the radical is reduced.
\end{enumerate}

To simplify a radical, we seek to split it into a perfect $n$th power and a ``leftover" term.
\end{frame}

\begin{frame}[t]{Simplifying Radical Expressions}
Simplify each expression: \begin{enumerate}[(a)]
\item $\sqrt{32}$
\item $\sqrt[3]{250}$
\item $\sqrt{24a^5}$
\end{enumerate}

\pause
\begin{enumerate}[(a)]
\item $\sqrt{32} = \sqrt{16}\cdot\sqrt{2} = 4\sqrt{2}$ \pause
\item $\sqrt[3]{250} = \sqrt[3]{125}\cdot \sqrt[3]{2} = 5\sqrt[3]{2}$ \pause
\item $\sqrt{24a^5} = \sqrt{4a^4}\cdot \sqrt{6a}=2a^2\sqrt{6a}$
\end{enumerate}

\end{frame}

\begin{frame}[t]{Rationalizing Denominators and Numerators}
To rationalize a denominator or a numerator means to remove radicals from it.

To do this, we multiply by the \textit{conjugate}: $a + b\sqrt{m}$ and $a - b\sqrt{m}$ are conjugates of each other. 
\end{frame}

\begin{frame}[t]{Rationalize the Denominator}
Rationalize the denominator: $\dfrac{8}{\sqrt{6}-\sqrt{2}}$

\begin{flalign*}
\onslide<2->{\dfrac{8}{\sqrt{6} - \sqrt{2}}\fp{\dfrac{\sqrt{6} + \sqrt{2}}{\sqrt{6}+\sqrt{2}}} &= \dfrac{8\fp{\sqrt{6} + \sqrt{2}}}{6 + \sqrt{12} - \sqrt{12} - 2}} & \\
\onslide<3->{&= \dfrac{8\fp{\sqrt{6} + \sqrt{2}}}{4}} & \\
\onslide<4>{&= 2\fp{\sqrt{6} + \sqrt{2}}} &
\end{flalign*}
\end{frame}

\begin{frame}[t]{Rationalize the Numerator}
Looking ahead to calculus, it will sometimes make sense to rationalize the numerator.

\vspace{12pt}

Rationalize the numerator: $\dfrac{\sqrt{5} + 3}{4}$

\begin{flalign*}
\onslide<2->{\dfrac{\sqrt{5} + 3}{4}\fp{\dfrac{\sqrt{5}-3}{\sqrt{5}-3}} &= \dfrac{5 - 3\sqrt{5} + 3\sqrt{5} - 9}{4\fp{\sqrt{5} - 3}}& \\}
\onslide<3->{&= \dfrac{-4}{4\fp{\sqrt{5} - 3}} & \\}
\onslide<4>{&= -\dfrac{1}{\sqrt{5} - 3} &}
\end{flalign*}
\end{frame}

\begin{frame}[t]{Rational Exponents}
\begin{block}{Definition}
A \textit{rational exponent} is an exponent that is a fraction (or a decimal, which we can rewrite as a fraction). If $m$ and $n$ are  positive integers, then $$a^{m/n} =\sqrt[n]{a^m} = \fp{\sqrt[n]{a}}^n$$
\end{block}

Generally, it will be easier to evaluate the exponent if we use the second option.
\end{frame}

\begin{frame}[t]{Evaluating Rational Exponents}
Simplify each expression:
\begin{enumerate}[(a)]
\item $(-125)^{-2/3}$
\item $64^{1/3}$
\item $16^{3/2}$
\end{enumerate}

\pause
\begin{enumerate}[(a)]
\item $(-125)^{-2/3} = \dfrac{1}{(-125)^{2/3}} = \dfrac{1}{\fp{\sqrt[3]{-125}}^2} = \dfrac{1}{(-5)^2} = \dfrac{1}{25}$ \pause
\item $64^{1/3} = \fp{\sqrt[3]{64}}^1 = 4^1 = 4$ \pause
\item $16^{3/2} = \fp{\sqrt{16}}^3 = 4^3 = 64$
\end{enumerate}

\end{frame}

\begin{frame}[t]{Next Steps}
\begin{itemize}
\item Post questions in the Lesson 1 Forum, if you have them.
\item Read Sections P.3 and P.4
\item Watch Video Lesson \#2
\item Complete Assignment \#1
\end{itemize}

\vfill

Thanks for watching!
\end{frame}

\end{document}