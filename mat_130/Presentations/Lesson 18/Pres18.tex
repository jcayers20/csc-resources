\documentclass[t, aspectratio=169]{beamer}
\usepackage{amsmath,amsfonts,amsthm,amstext,amssymb, xcolor, tikz, pgf}

% ----------------------------------------------------------
% Theme Setup

% Use Metropolis Theme
\usetheme[numbering=fraction]{metropolis}
\setbeamertemplate{blocks}[rounded][shadow=false]
\makeatletter
\setlength{\metropolis@titleseparator@linewidth}{1pt}
\makeatother

% Define Colors
\definecolor{chargerblue}{HTML}{002764}
\definecolor{chargerred}{HTML}{e02034}
\definecolor{bggray}{HTML}{d0d3d4}

% Set Colors
\setbeamercolor{title}{fg=chargerblue}
\setbeamercolor{background canvas}{bg=white}
\setbeamercolor{title separator}{fg=chargerred}
\setbeamercolor{structure}{fg=chargerblue}
\setbeamercolor{frametitle}{fg=white, bg=chargerblue}
\setbeamercolor*{normal text}{fg=chargerblue}
\setbeamercolor*{block body}{bg=bggray}
\setbeamercolor*{block title}{bg=chargerblue, fg=white}
% ----------------------------------------------------------

% ----------------------------------------------------------
% Custom Definitions, Commands, Environments, etc.

% Sets of numbers
\def\R{\mathbb{R}} % The reals
\def\N{\mathbb{N}} % The naturals
\def\Z{\mathbb{Z}} % The integers
\def\Q{\mathbb{Q}} % The rationals

% Blank space
\newcommand{\blank}[1]{\underline{\hspace{#1}}} % Blank space

% Fitted inclusion symbols
\newcommand{\fp}[1]{\left({#1}\right)} % Fitted parentheses around content
\newcommand{\fb}[1]{\left[{#1}\right]} % Fitted brackets
\newcommand{\set}[1]{\left\{{#1}\right\}} % Fitted braces (useful for sets)
\newcommand{\av}[1]{\left|{#1}\right|} % Fitted absolute value bars

% Miscellaneous
\def\then{\Rightarrow}



% Coordinate Plane (Four-Quadrant)
\def\coordplane {
	\begin{tikzpicture}		\draw[step=0.25cm,black,very thin,opacity=0.25] (-2.5cm, -2.5cm) grid (2.5cm, 2.5cm);
	\draw[<->,thick,black] (-2.5cm, 0) -- (2.5cm, 0) node[anchor=north west,pos=0.94,font=\scriptsize]{$x$};
	\draw[<->,thick,black] (0,-2.5cm) -- (0, 2.5cm) node[anchor=south east,font=\scriptsize,pos=0.94]{$y$};
	\end{tikzpicture}
}

% Coordinate Plane (One-Quadrant)
\def\onequad {
	\begin{tikzpicture}
	\draw[step=0.25cm, black, very thin, opacity=0.25] (0,0) grid (7.5cm,5cm);
	\draw[->, thick, black] (0,0) -- (7.5cm, 0) node[anchor=north west,font=\scriptsize,pos=0.94]{$x$};
	\draw[->, black, thick] (0,0) -- (0,5cm) node[anchor=south east,font=\scriptsize,pos=0.94]{$y$};
	\end{tikzpicture}
}
% ----------------------------------------------------------

% ----------------------------------------------------------
% Presentation Information 
\title[5.2]{Logarithmic Functions and their Graphs}
\subtitle{Section 5.2}
\author{Jacob Ayers}
\institute{Lesson \#18}
\date{MAT 130}
% ----------------------------------------------------------

\begin{document}
	
	% Slide 1 (Title Slide)
	\begin{frame}
		\titlepage
	\end{frame}
	
	% Slide 2 (Objectives)
	\begin{frame}{Objectives}
		\begin{itemize}
			\item Evaluate logarithmic functions, including natural logarithmic functions
			\item Graph logarithmic functions, including natural logarithmic functions
			\item Use logarithmic functions to model and solve real-life problems
		\end{itemize}
	\end{frame}

	\begin{frame}{Logarithmic Functions}
		\begin{block}{Definition}
			For $x > 0$, $a > 0$, and $a \neq 1$, $$y = \log_a x \text{ if and only if } x = a^y$$ The function $$f(x) = \log_a x$$ is the \textit{logarithmic function with base $a$}
		\end{block} \pause
	
		Key takeaway: $\log_a x = y$ and $a^y = x$ are equivalent.
	\end{frame}

	\begin{frame}{Converting between Exponential and Logarithmic Form}
		Write as an exponential equation: $\log_3 81 = 4$ \pause \\
		$3^4 = 81$ \pause \vspace{12pt}
		
		Write as an exponential equation: $\log_9 \dfrac{1}{81} = -2$ \pause \\
		$9^{-2} = \dfrac{1}{81}$ \pause \vspace{12pt}
		
		Write as a logarithmic equation: $16^{3/2} = 64$ \pause \\
		$\log_{16} 64 = \dfrac32$ \pause \vspace{12pt}
		
		Write as a logarithmic equation: $2^8 = 256$ \pause \\
		$\log_2 256 = 8$
	\end{frame}

	\begin{frame}{Evaluating Logarithms}
		When evaluating logarithms, it can be helpful to write the logarithm as an exponential equation.
		
		Evaluate $\log_2 32$ \pause \\
		$\log_2 32 = y \Rightarrow 2^y = 32 \Rightarrow y = 5$ \pause \vspace{12pt}
		
		Evaluate $\log_6 1$ \pause \\
		$\log_6 1 = y \then 6^y = 1 \Rightarrow y = 0$ \pause \vspace{12pt}
		
		Evaluate $\log_5 \dfrac{1}{125}$ \pause \\
		$\log_5 \dfrac{1}{125} = y \then 5^y = \dfrac{1}{125} \then y = -3$
	\end{frame}

	\begin{frame}{Special Logarithms}
		\begin{block}{Common and Natural Logarithms}
			\begin{enumerate}[1)]
				\item The \textit{common logarithm} is a logarithm with base $10$. It is denoted by $\log_{10}$ or (more commonly) $\log$.
				\item The \textit{natural logarithm} is a logarithm with base $e$. It is denoted by $\log_e$ or (more commonly) $\ln$.
			\end{enumerate}
		\end{block} \pause
	
		There are $\log$ and $\ln$ buttons on all calculators. Take a second and find them on yours.
	\end{frame}

	\begin{frame}{Evaluating Special Logarithms}
		Using your calculator, evaluate each of the following logarithms (round to four decimal places): \begin{enumerate}[a)]
			\item $\log 275$
			\item $\log \dfrac12$
			\item $\ln -5$
			\item $\ln 10$
		\end{enumerate} \pause
		\begin{enumerate}[a)]
			\item $\log 275 \approx 2.4393$
			\item $\log \dfrac12 \approx -0.3010$
			\item $\ln -5$ has no solution because $-5 \leq 0$
			\item $\ln 10 \approx 2.3026$
		\end{enumerate}
	\end{frame}

	\begin{frame}{Properties of Logarithms}
		\begin{block}{Properties of Logarithms}
			\begin{enumerate}[1)]
				\item $\log_a 1 = 0$ because $a^0 = 1$
				\item $\log_a a = 1$ because $a^1 = a$
				\item (Inverse Property) $\log_a a^x = x$ and $a^{\log_a x} = x$
				\item (One-to-One Property) If $\log_a x = \log_a y$ then $x = y$
			\end{enumerate}
		\end{block} \pause
	
		Use properties of logarithms to simplify: \\ a) $\log_4 1$ \hspace{0.5in} b) $20^{\log_{20} 3}$ \hspace{0.5in} c) $\log_8 8$ \\ \pause \vspace{12pt}
		
		a) $\log_4 1 = 0$ (Property 1) \\
		b) $20^{\log_{20} 3} = 3$ (Property 3) \\
		c) $\log_8 8 = 1$ (Property 2)
	\end{frame}

	\begin{frame}{Properties of Logarithms}
		Use the One-to-One Property of Logarithms to solve: $\log_5 \fp{x^2 + 3} = \log_5 12$ \begin{flalign*}
		\onslide<2->{x^2 + 3 &= 12 & \\}
		\onslide<3->{x^2 &= 9 & \\}
		\onslide<4->{x &= \pm 3}
		\end{flalign*}
		\onslide<5->{Use the One-to-One Property of Logarithms to solve: $\ln (7x - 3) = \ln (4x + 1)$} \begin{flalign*}
		\onslide<6->{7x - 3 &= 4x + 1 & \\}
		\onslide<7->{3x &= 4 & \\}
		\onslide<8>{x &= \dfrac43}
		\end{flalign*}
	\end{frame}

	\begin{frame}{Graphs of Logarithmic Function}
		Graph $f(x) = 3^x$ and $g(x) = \log_3 x$ in the same coordinate plane. \pause
		
		Since $3^x$ and $\log_3 x$ are inverse functions, we can plot points for $f(x)$ and then invert the order of the ordered pairs in order to graph $g(x)$ \pause
		
		\begin{tabular}{c|ccccccc}
			$x$ & $-3$ & $-2$ & $-1$ & $0$ & $1$ & $2$ & $3$ \\ \hline
			$f(x)$ & $\dfrac{1}{27}$ & $\dfrac19$ & $\dfrac13$ & $1$ & $3$ & $9$ & $27$
		\end{tabular} \pause
	
		\begin{tabular}{c|ccccccc}
			$x$ & $\dfrac{1}{27}$ & $\dfrac19$ & $\dfrac13$ & $1$ & $3$ & $9$ & $27$ \\ \hline
			$g(x)$ & $-3$ & $-2$ & $-1$ & $0$ & $1$ & $2$ & $3$
		\end{tabular}
	\end{frame}

	\begin{frame}{Graphs of Logarithmic Functions}
		\includegraphics[width=4in]{Log1.png}
	\end{frame}

	\begin{frame}{Graphs of Logarithmic Functions}
		Graph without using a calculator: $f(x) = \log_2 x$ \pause
		
		When graphing without using a calculator, it's often easier to pick values of $f(x)$ and find values of $x$ than vice-versa. \pause
		
		$f(x) = 3 \then \log_2 x = 3 \then 2^3 = x \then x = 8 \then (8,3)$ \pause \\
		$f(x) = 2 \then \log_2 x = 2 \then 2^2 = x \then x = 4 \then (4,2)$ \pause \\
		$f(x) = 1 \then \log_2 x = 1 \then 2^1 = x \then x = 2 \then (2,1)$ \pause \\
		$f(x) = 0 \then \log_2 x = 0 \then 2^0 = x \then x = 1 \then (1,0)$ \pause \\
		$f(x) = -1 \then \log_2 x = -1 \then 2^{-1} = x \then x = \dfrac12 \then \fp{\dfrac12, -1}$ \pause \\
		$f(x) = -2 \then \log_2 x = -2 \then 2^{-2} = x \then x = \dfrac14 \then \fp{\dfrac14, -2}$ \pause \\
		$f(x) = -3 \then \log_2 x = -3 \then 2^{-3} = x \then x = \dfrac18 \then \fp{\dfrac18, -3}$
	\end{frame}

	\begin{frame}{Graphs of Logarithmic Equations}
		\includegraphics[width=4in]{Log2.png}
	\end{frame}

	\begin{frame}{An Application}
		Students participating in a psychology experiment attended several lectures on a subject and took an exam. Every month for a year after the exam, the students took a retest to see how much of the material they remembered. The average scores for the group are given by the \textit{human memory model} $f(t) = 78 - 5\ln(t+1), 0\leq t \leq 12$, where $t$ is the time in months. Find the average score on the original exam ($t = 0$) and the average score after 8 months. \pause
		
		To find the score at $t = 0$, we evaluate $f(0)$. \pause \\ $f(0) = 78 - 5\ln(1) = 78 - 0 = 78$. \pause
		
		The average score on the original test was 78.
	\end{frame}

	\begin{frame}{An Application}
		$f(t) = 78 - 5\ln(t + 1)$
		
		To find the average score after 8 months, we evaluate $f(8)$. \pause \\
		$f(8) = 78 - 5\ln(9) = 78 - 10.9861 \approx 67.0139$ \pause
		
		The average score after 8 months was about 67.
	\end{frame}

	\begin{frame}{Next Steps}
		\begin{itemize}
			\item Ask questions in Lesson 18 Forum, if you have any
			\item Read 5.3
			\item Watch Video Lesson \#19
			\item Complete Assignment \#9
		\end{itemize}
	
		\vfill
		
		Thanks for watching!
	\end{frame}
	
\end{document}