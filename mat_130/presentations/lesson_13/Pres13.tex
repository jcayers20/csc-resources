\documentclass[t]{beamer}
\usepackage[utf8]{inputenc}
\usepackage{amsmath,amsfonts,amsthm,amstext,amssymb, xcolor, tikz, pgf, polynom}

% ----------------------------------------------------------
% Theme Setup

% Use Metropolis Theme
\usetheme[numbering=fraction]{metropolis}
\setbeamertemplate{blocks}[rounded][shadow=false]
\makeatletter
\setlength{\metropolis@titleseparator@linewidth}{1pt}
\makeatother

% Define Colors
\definecolor{chargerblue}{HTML}{002764}
\definecolor{chargerred}{HTML}{e02034}
\definecolor{bggray}{HTML}{d0d3d4}

% Set Colors
\setbeamercolor{title}{fg=chargerblue}
\setbeamercolor{background canvas}{bg=white}
\setbeamercolor{title separator}{fg=chargerred}
\setbeamercolor{structure}{fg=chargerblue}
\setbeamercolor{frametitle}{fg=white, bg=chargerblue}
\setbeamercolor*{normal text}{fg=chargerblue}
\setbeamercolor*{block body}{bg=bggray}
\setbeamercolor*{block title}{bg=chargerblue, fg=white}
% ----------------------------------------------------------

% ----------------------------------------------------------
% Custom Definitions, Commands, Environments, etc.

% Sets of numbers
\def\R{\mathbb{R}} % The reals
\def\N{\mathbb{N}} % The naturals
\def\Z{\mathbb{Z}} % The integers
\def\Q{\mathbb{Q}} % The rationals

% Blank space
\newcommand{\blank}[1]{\underline{\hspace{#1}}} % Blank space

% Fitted inclusion symbols
\newcommand{\fp}[1]{\left({#1}\right)} % Fitted parentheses around content
\newcommand{\fb}[1]{\left[{#1}\right]} % Fitted brackets
\newcommand{\set}[1]{\left\{{#1}\right\}} % Fitted braces (useful for sets)
\newcommand{\av}[1]{\left|{#1}\right|} % Fitted absolute value bars



% Coordinate Plane (Four-Quadrant)
\def\coordplane {
	\begin{tikzpicture}		\draw[step=0.25cm,black,very thin,opacity=0.25] (-2.5cm, -2.5cm) grid (2.5cm, 2.5cm);
	\draw[<->,thick,black] (-2.5cm, 0) -- (2.5cm, 0) node[anchor=north west,pos=0.94,font=\scriptsize]{$x$};
	\draw[<->,thick,black] (0,-2.5cm) -- (0, 2.5cm) node[anchor=south east,font=\scriptsize,pos=0.94]{$y$};
	\end{tikzpicture}
}

% Coordinate Plane (One-Quadrant)
\def\onequad {
	\begin{tikzpicture}
	\draw[step=0.25cm, black, very thin, opacity=0.25] (0,0) grid (7.5cm,5cm);
	\draw[->, thick, black] (0,0) -- (7.5cm, 0) node[anchor=north west,font=\scriptsize,pos=0.94]{$x$};
	\draw[->, black, thick] (0,0) -- (0,5cm) node[anchor=south east,font=\scriptsize,pos=0.94]{$y$};
	\end{tikzpicture}
}
% ----------------------------------------------------------

% ----------------------------------------------------------
% Presentation Information 
\title[3.4]{Zeros of Polynomial Functions}
\subtitle{Section 3.4}
\author{Jacob Ayers}
\institute{Lesson \#13}
\date{MAT 130}
% ----------------------------------------------------------

\begin{document}
	
	% Slide 1 (Title Slide)
	\begin{frame}
		\titlepage
	\end{frame}
	
	% Slide 2 (Objectives)
	\begin{frame}{Objectives}
		\begin{itemize}
			\item Use the Fundamental Theorem of Algebra to determine number of zeros of polynomial functions
			\item Find rational zeros of polynomial functions
			\item Find complex zeros using conjugate pairs
			\item Find zeros of polynomials by factoring
		\end{itemize}
	\end{frame}

	\begin{frame}{FTA and Linear Factorization Theorem}
		We start with the Fundamental Theorem of Algebra, proven by the famous mathematician Carl Friedrich Gauss.
		
		\begin{block}{Fundamental Theorem of Algebra (FTA)}
			If $f(x)$ is a polynomial function of degree $n$, where $n > 0$, then $f$ has at least one zero in the complex number system.
		\end{block}
	
		\pause
	
		We can actually go further than that, though.
		
		\pause
		
		\begin{block}{Linear Factorization Theorem}
			If $f(x)$ is a polynomial function of degree $n$, where $n > 0$, then $f(x)$ has exactly $n$ linear factors $$f(x) = a_n\fp{x-c_1}\fp{x-c_2}\cdots\fp{x-c_n}$$ where $c_1,c_2,\dots,c_n$ are complex numbers.
		\end{block}
	\end{frame}

	\begin{frame}{FTA and Linear Factorization Theorem}
		Example: Determine the number of zeros that each of the following functions has:
		
		$f(x) = x^4 - 1$ \\ \pause $4$ \pause
		
		$f(x) = 3x^3 - 7x + 11$ \\ \pause $3$
		
		\pause $f(x) = 5x^9$ \\ \pause $9$
	\end{frame}

	\begin{frame}{The Rational Zero Test}
		The \textit{Rational Zero Test} relates the possible rational zeros of a function to its leading coefficient and its constant term.
		
		\pause
		
		\begin{block}{The Rational Zero Test}
			If the polynomial $$f(x) = a_nx^n + a_{n-1}x^{n-1} + \cdots + a_2x^2 + a_1x + a_0$$ has integer coefficients, then every rational zero of $f$ has the form $$\dfrac{p}{q}$$ where $p$ is a factor of $a_0$ and $q$ is a factor of $a_n$.
		\end{block}
	\end{frame}

	\begin{frame}{The Rational Zero Test}
		List all \textit{possible} rational zeros of $f(x) = x^3 + 7x^2 - 13x - 8$. \pause
		
		$\pm 1, \pm 2, \pm 4, \pm 8$ \vspace{18pt}
		
		List all \textit{possible} rational zeros of $f(x) = 2x^4 - 5x^3 + x^2 - 7x - 36$ \pause
		
		$\pm 1, \pm 2, \pm 3, \pm \dfrac32, \pm 4, \pm 6, \pm 9, \pm \dfrac92, \pm 12, \pm 18$
	\end{frame}

	\begin{frame}{The Rational Zero Test}
		Find all rational zeros of $f(x) = x^4 - x^3 + x^2 - 3x - 6$. \pause
		
		The possible rational zeros (using RZT) are: $\pm 1, \pm 2, \pm 3, \pm 6$
		
		We can use repeated division to find all the rational zeros.
		
		First, we try $x = 1$:
		
		\pause
		
		\polyhornerscheme[x=1]{x^4 - x^3 + x^2 - 3x - 6}
		
		\pause
		
		The remainder isn't zero, so we know that $x = 1$ is \textit{not} a zero.		
	\end{frame}

	\begin{frame}{The Rational Zero Test}
		Moving on to $x = -1$: \pause
		
		\polyhornerscheme[x=-1]{x^4 - x^3 + x^2 - 3x - 6}
		
		\pause
		
		Since the remainder is zero, $x = -1$ is a zero. Specifically, $x^4 - x^3 + x^2 - 3x - 6 = (x+1)(x^3-2x^2+3x-6)$. \pause
		
		We could keep using the original function, but let's make it easier on ourselves by testing out rational zeros on $x^3 - 2x^2 + 3x - 6$.
	\end{frame}

	\begin{frame}{The Rational Zero Test}
		$f(x) = (x+1)(x^3 - 2x^2 + 3x - 6)$
		
		Working through the divisions until we find another one that works:
		
		\pause
		
		\polyhornerscheme[x=1]{x^3 - 2x^2 + 3x - 6} \pause \polyhornerscheme[x=-1]{x^3 - 2x^2 + 3x - 6} \\
		\polyhornerscheme[x=2]{x^3 - 2x^2 + 3x - 6} \pause \\ $x = 2$ works; $f(x) = (x+1)(x-2)(x^2 + 3)$
	\end{frame}

	\begin{frame}{The Rational Zero Test}
		We know that $x^2 + 3$ isn't going to produce any real zeros, let alone rational zeros, so $x = -1$ and $x = 2$ are the only rational zeros of $f(x)$.
	\end{frame}
	
	\begin{frame}{The Rational Zero Test}
		Find all rational zeros of $x^3 - 15x^2 + 75x - 125$.
		
		\only<2->{First, list all possible rational zeros: $\pm 1, \pm 5, \pm 25, \pm 125$ }
		
		\only<3->{Next, test these possible zeros until one is found:} 
		
		\only<3-4>{\polyhornerscheme[x=1]{x^3 - 15x^2 + 75x - 125}}
		
		 \only<4>{\polyhornerscheme[x=-1]{x^3 - 15x^2 + 75x - 125}}
		 
		 \only<5->{\polyhornerscheme[x=5]{x^3 - 15x^2 + 75x - 125}}
		 
		 \only<6->{So $x = 5$ is a rational zero, and $f(x) = (x-5)(x^2 - 10x + 25)$}
		 
		 \only<7->{We don't need to continue dividing; we know that $x^2 - 10x + 25 = (x-5)^2$.}
		 
		 \only<8>{So $f(x) = (x-5)^3$ and $x = 5$ is the only rational zero of the function.}
	\end{frame}

	\begin{frame}{The Rational Zero Test}
		Find all real solutions of the function $-10x^3 + 15x^2 + 16x - 12 = 0$.
		
		\pause
		
		First, list all possible rational zeros: $\dfrac{\pm 1, \pm 2, \pm 3, \pm 4, \pm 6, \pm 12}{\pm 1, \pm 2, \pm 5, \pm 10}$ \pause
		
		In list form: $\pm 1, \pm 2, \pm 3, \pm 4, \pm 6, \pm 12, \pm \dfrac12, \pm \dfrac32, \newline \pm \dfrac15, \pm \dfrac25, \pm\dfrac35, \pm\dfrac45, \pm\dfrac65, \pm\dfrac{12}{5}, \pm\dfrac{1}{10}, \pm\dfrac{3}{10}$ \vspace{18pt} \pause
		
		That's a lot of possibilities!!! We're not testing all of them; let's use a graph to try to narrow it down.
	\end{frame}

	\begin{frame}{The Rational Zero Test}
		Looking at the graph, there are three zeros: \pause \\
		One appears to be at $x = 2$. \pause \\
		Another is between $0.5$ and $1$. \pause \\
		Finally, one is between $-1$ and $-1.5$. \pause
		
		Since $2$ looks like a surefire solution, let's test it and see where that takes us. \pause
		
		\polyhornerscheme[x=2]{-10x^3 + 15x^2 + 16x - 12} \pause
	\end{frame}

	\begin{frame}{The Rational Zero Test}
		So $-10x^3 + 15x^2 + 16x - 12 = (x-2)(-10x^2 - 5x + 6)$. You can use the quadratic formula to show that the other two real zeros are $x = \dfrac{5 + \sqrt{265}}{-20} \approx -1.0639$ and $x = \dfrac{5 - \sqrt{265}}{20} \approx 0.5639$.
	\end{frame}

	\begin{frame}{Conjugate Pairs}
		\begin{block}{Complex Zeros Occur in Conjugate Pairs}
			Let $f$ be a polynomial function with real coefficients. If $a + bi, b\neq 0$ is a zero of $f$, then $a - bi$ is also a zero of $f$.
		\end{block}
	
		This fact will prove useful to us as we try to find zeros of polynomials.
	\end{frame}

	\begin{frame}{Conjugate Pairs}
		Find a fourth-degree polynomial $f$ with real coefficients that has $-1$, $-1$, and $3i$ as zeros. \pause
		
		Since $3i$ is a zero of the function, $-3i$ is also a zero of the function. So our function is of the form $$f(x) = a(x+1)(x+1)(x-3i)(x+3i)$$ by the Linear Factorization Theorem. \pause
		
		Let $a = 1$ (because that's easy to work with; you can let it be whatever you want). Multiplying this out (I used Wolfram|Alpha), we get $f(x) = x^4 + 2x^3 + 10x^2 + 18x + 9$.
	\end{frame}

	\begin{frame}{Conjugate Pairs}
		Find the \textit{quartic} (fourth-degree) polynomial $f$ with real coefficients that has $1$, $-2$, and $2i$ as zeros, and $f(-1) = 10$. \pause
		
		This problem is similar to the last one. The only difference is we don't get to choose $a$. \pause
		
		First, we know that $-2i$ is our fourth zero. So $$f(x) = a\fb{(x-1)(x+2)(x-2i)(x+2i)} = a\fp{x^4 + x^3 + 2x^2 + 4x - 8}$$
	\end{frame}

	\begin{frame}{Conjugate Pairs}
		Now, we need to find a value of $a$ so that when we plug $-1$ into the function, we get $10$. That can be accomplished using basic algebra. \pause
		\begin{flalign*}
		a\fb{(-1)^4 + (-1)^3 + 2(-1)^2 + 4(-1) - 8} &= 10 & \\
		a\fb{1 - 1 + 2(1) -4 - 8} &= 10 & \\
		a\fp{-10} &= 10 & \\
		a &= -1
		\end{flalign*} \pause
		So our function is $f(x)=-\fp{x^4 + x^3 + 2x^2 + 4x - 8}$.
	\end{frame}

	\begin{frame}{Factoring Polynomials}
		Find all the zeros of $f(x) = 3x^3 - 2x^2 + 48x - 32$ given that $4i$ is a zero of $f$. \pause
		
		There are a total of $3$ zeros, and we are given two of them (since $4i$ is a zero, so is $-4i$). We just need to find the third one. \pause
		
		First, note that $(x-4i)(x+4i) = x^2 + 16$. So if we divide $3x^3 - 2x^2 + 48x - 32$ by $x^2 + 16$, we will find our third factor. \pause
		\polylongdiv{3x^3 - 2x^2 + 48x - 32}{x^2 + 16}
	\end{frame}

	\begin{frame}{Factoring Polynomials}
		Write $f(x) = x^4 + 8x^2 - 9$ as the product of linear factors and list all the zeros of the function. \pause
		
		Take a look at this function. Can you think of a way that we could do this without resorting to using the rational zero test and repeated synthetic division? \pause
		
		We could use substitution. Let $u = x^2$ so that $f(x) = u^2 + 8u - 9$. Finding the zeros of this function is easy: $u^2 + 8u - 9 = 0 \Rightarrow (u+9)(u-1) = 0 \Rightarrow u = \set{-9, 1}$ \pause
		
		So $x^2 = -9 \Rightarrow x = \pm 3i$ or $x^2 = 1 \Rightarrow x = \pm 1$. We have found all four zeros.
		
		As a product of linear factors: $f(x) = (x + 1)(x-1)(x+3i)(x-3i)$.
	\end{frame}

	\begin{frame}{Next Steps}
		\begin{itemize}
			\item Post questions in Lesson 13 Forum, if you have any
			\item Complete Assignment 6
			\item Begin Module 8
			\begin{itemize}
				\item Read 3.5
				\item Watch Video Lesson \#14
			\end{itemize}
		\end{itemize}
	
		\vfill
		
		Thanks for watching!
	\end{frame}
\end{document}