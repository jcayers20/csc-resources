\documentclass{beamer}
\usepackage[utf8]{inputenc}
\usepackage{amsmath,amsfonts,amsthm,amstext,amssymb, xcolor, tikz, pgf}

% ----------------------------------------------------------
% Theme Setup

% Use Metropolis Theme
\usetheme[numbering=fraction]{metropolis}
\setbeamertemplate{blocks}[rounded][shadow=false]
\makeatletter
\setlength{\metropolis@titleseparator@linewidth}{1pt}
\makeatother

% Define Colors
\definecolor{chargerblue}{HTML}{002764}
\definecolor{chargerred}{HTML}{e02034}
\definecolor{bggray}{HTML}{d0d3d4}

% Set Colors
\setbeamercolor{title}{fg=chargerblue}
\setbeamercolor{background canvas}{bg=white}
\setbeamercolor{title separator}{fg=chargerred}
\setbeamercolor{structure}{fg=chargerblue}
\setbeamercolor{frametitle}{fg=white, bg=chargerblue}
\setbeamercolor*{normal text}{fg=chargerblue}
\setbeamercolor*{block body}{bg=bggray}
\setbeamercolor*{block title}{bg=chargerblue, fg=white}
% ----------------------------------------------------------

% ----------------------------------------------------------
% Custom Definitions, Commands, Environments, etc.

% Sets of numbers
\def\R{\mathbb{R}} % The reals
\def\N{\mathbb{N}} % The naturals
\def\Z{\mathbb{Z}} % The integers
\def\Q{\mathbb{Q}} % The rationals

% Blank space
\newcommand{\blank}[1]{\underline{\hspace{#1}}} % Blank space

% Fitted inclusion symbols
\newcommand{\fp}[1]{\left({#1}\right)} % Fitted parentheses around content
\newcommand{\fb}[1]{\left[{#1}\right]} % Fitted brackets
\newcommand{\set}[1]{\left\{{#1}\right\}} % Fitted braces (useful for sets)
\newcommand{\av}[1]{\left|{#1}\right|} % Fitted absolute value bars



% Coordinate Plane (Four-Quadrant)
\def\coordplane {
	\begin{tikzpicture}		\draw[step=0.25cm,black,very thin,opacity=0.25] (-2.5cm, -2.5cm) grid (2.5cm, 2.5cm);
		\draw[<->,thick,black] (-2.5cm, 0) -- (2.5cm, 0) node[anchor=north west,pos=0.94,font=\scriptsize]{$x$};
		\draw[<->,thick,black] (0,-2.5cm) -- (0, 2.5cm) node[anchor=south east,font=\scriptsize,pos=0.94]{$y$};
	\end{tikzpicture}
}

% Coordinate Plane (One-Quadrant)
\def\onequad {
	\begin{tikzpicture}
		\draw[step=0.25cm, black, very thin, opacity=0.25] (0,0) grid (7.5cm,5cm);
		\draw[->, thick, black] (0,0) -- (7.5cm, 0) node[anchor=north west,font=\scriptsize,pos=0.94]{$x$};
		\draw[->, black, thick] (0,0) -- (0,5cm) node[anchor=south east,font=\scriptsize,pos=0.94]{$y$};
	\end{tikzpicture}
}
% ----------------------------------------------------------

% ----------------------------------------------------------
% Presentation Information 
\title[1.5 and 1.6]{Complex Numbers; Other Types of Equation}
\subtitle{Sections 1.5 and 1.6}
\author{Jacob Ayers}
\institute{Lesson \#6}
\date{MAT 130}
% ----------------------------------------------------------

\begin{document}

% Slide 1 (Title Slide)
\begin{frame}
\titlepage
\end{frame}

% Slide 2 (Objectives)
\begin{frame}[t]{Objectives}
\begin{itemize}
	\item Use the imaginary unit $i$ to write complex numbers
	\item Perform operations on complex numbers
	\item Find complex solutions to quadratic equations
	\item Solve polynomial equations of degree three or higher
	\item Solve radical equation
	\item Solve rational equations
\end{itemize}
\end{frame}

\begin{frame}[t]{The Imaginary Unit $i$ and Complex Numbers}
Not all equations have \textit{real} solutions.

Example: $x^2 = -4$

\pause

\begin{block}{Definition}
We define the \textit{imaginary unit} $i$ by $i = \sqrt{-1}$
\end{block}

\pause

\begin{block}{Definition}
A \textit{complex number} is a number of the form $a + bi$. We call $a$ the \textit{real part} and $bi$ the \textit{imaginary part}.

If $b = 0$, then $a+bi \in \R$. If $a,b\neq 0$, then $a + bi$ is an \textit{imaginary number}. If $a = 0$ and $b\neq 0$, then $a + bi$ is a \textit{pure imaginary number}.
\end{block}
\end{frame}

\begin{frame}[t]{Operations with Complex Numbers}
To add or subtract complex numbers, combine like terms (real is like to real, imaginary is like to imaginary).

\pause

To multiply complex numbers, use FOIL (bear in mind that $i^2 = -1$).

\pause
To divide complex numbers, multiply top and bottom by the conjugate of the denominator.
\end{frame}

\begin{frame}[t]{Adding and Subtracting Complex Numbers}
Example: $(3 + 4i) + (7 - 6i)$\pause$=10 - 2i$ \vspace{12pt}

\pause

Example: $(-7 - 3i) + (-2 + 13i)$\pause$= -9 + 10i$ \vspace{12pt}

\pause

Example: $(3 + 4i) - (7 - 6i)$\pause$= -4 + 10i$ \vspace{12pt}

\pause

Example: $(-7 - 3i) - (-2 + 13i)$\pause$= -5 - 16i$
\end{frame}

\begin{frame}[t]{Multiplying Complex Numbers}
Example: Find the product of $6 - 2i$ and $3 + 5i$.
\begin{flalign*}
\onslide<2->{(6-2i)(3+5i) &= 18 + 30i - 6i - 10i^2 & \\}
\onslide<3->{&= 18 + 24i - 10(-1) & \\}
\onslide<4->{&= 28 + 24i}
\end{flalign*}

\onslide<5->{Example: (-6-i)(-4-9i)}
\begin{flalign*}
\onslide<6->{(-6-i)(-4-9i) &= 24 + 54i + 4i + 9i^2 & \\}
\onslide<7->{&= 24 - 9 + 58i & \\}
\onslide<8>{&= 15 + 58i}
\end{flalign*}
\end{frame}

\begin{frame}[t]{Dividing Complex Numbers}
Example: $\dfrac{5 + 8i}{2 - 3i}$
\begin{flalign*}
\onslide<2->{\dfrac{5+8i}{2-3i}\fp{\dfrac{2+3i}{2+3i}} &= \dfrac{10 + 15i + 16i + 24i^2}{4 + 6i - 6i - 9i^2} & \\}
\onslide<3->{&= \dfrac{-14 + 31i}{13} & \\}
\onslide<4>{&= -\dfrac{14}{13} + \dfrac{31}{13}i}
\end{flalign*}
\end{frame}

\begin{frame}[t]{Dividing Complex Numbers}
Example: $\dfrac{4-2i}{2+3i}$
\begin{flalign*}
\onslide<2->{\dfrac{4-2i}{2+3i}\fp{\dfrac{2-3i}{2-3i}} &= \dfrac{8 - 12i - 4i + 6i^2}{4 - 6i + 6i - 9i^2} & \\}
\onslide<3->{&= \dfrac{2 - 16i}{13} & \\}
\onslide<4>{&= \dfrac{2}{13} - \dfrac{14}{13}i}
\end{flalign*}
\end{frame}

\begin{frame}[t]{Complex Solutions of Quadratic Equations}
Whenever $b^2 - 4ac < 0$, a quadratic equation has complex solutions.

Example: Solve $8x^2 + 14x + 9 = 0$
\begin{flalign*}
\onslide<2->{x &= \dfrac{-b\pm\sqrt{b^2 - 4ac}}{2a} & \\}
\onslide<3->{&= \dfrac{-(14)\pm\sqrt{(14)^2 - 4(8)(9)}}{2(8)} & \\}
\onslide<4->{&= \dfrac{-14\pm\sqrt{-92}}{16} & \\}
\onslide<5->{&= \dfrac{-14\pm 2i\sqrt{23}}{16} & \\}
\onslide<6>{&= -\dfrac{7}{8}\pm \dfrac{\sqrt{23}}{8}i}
\end{flalign*}
\end{frame}

\begin{frame}[t]{Polynomial Equations}
We can solve certain polynomial equations whose degree is 3 or higher using strategies we've learned previously.

Example: Solve $9x^4 - 12x^2 = 0$

\pause

First, factor out the GCD. Now, we have $3x^2\fp{x^2 - 4} = 0$

\pause

Now we can use the zero product property to solve for $x$: \\
$3x^2 = 0 \Rightarrow x = 0$ \\
$x^2 - 4 = 0 \Rightarrow x = \pm 2$

\pause

So the solutions to the equation are $x = \set{-2, 0, 2}$
\end{frame}

\begin{frame}[t]{Polynomial Equations}
Solve $x^3 - 5x^2 - 2x + 10 = 0$

\pause

This time, we can factor by grouping. \\
$x^2(x - 5) - 2(x-5) = 0 \Rightarrow \fp{x^2 - 2}(x-5) = 0$

\pause

From here, we again use the zero product property: \\
$x^2 - 2 = 0 \Rightarrow x = \pm\sqrt{2}$ \\
$x - 5 = 0 \Rightarrow x = 5$

\pause

So the solutions to the equation are $x = \set{-\sqrt{2}, \sqrt{2}, 5}$

\end{frame}

\begin{frame}[t]{Polynomial Equations}
Solve: $9x^4 - 37x^2 + 4 = 0$

\pause

This involves the strategy of substitution, which we've not yet encountered. The idea is to write this equation in the form $au^2 + bu + c = 0$, because we can solve that.

\pause

Let $u = x^2$ so that $9u^2 - 37u + 4 = 0$.

Factoring yields $(9u - 1)(u-4) = 0$ so $u = \set{\dfrac19, 4}$

\pause

Now, substitute $x^2$ back in for $u$: \\ $x^2 = \dfrac19 \Rightarrow x = \pm\dfrac13$ \\ $x^2 = 4 \Rightarrow x = \pm 2$

So the solutions to the equation are $x = \set{-2, -\dfrac13, \dfrac13, 2}$
\end{frame}

\begin{frame}[t]{Radical Equations}
The inverse of taking an $n$th root is taking an $n$th power. We can use this fact to solve radical equations.

\onslide<2->{Example: Solve for $x$: $\sqrt{2x + 7} - x = 2$} 
\begin{flalign*}
\onslide<3->{\sqrt{2x + 7} &= x + 2 & \\}
\onslide<4->{2x + 7 &= (x+2)^2 & \\}
\onslide<5->{2x + 7 &= x^2 + 4x + 4 & \\}
\onslide<6->{x^2 + 2x - 3 &= 0 & \\}
\onslide<7>{(x + 3)(x - 1) &= 0 & \\
x &= \set{-3, 1}}
\end{flalign*}
\end{frame}

\begin{frame}[t]{Radical Equations}
Solve $(x-5)^{2/3} = 16$
\begin{flalign*}
\onslide<2->{\sqrt[3]{(x-5)^2} &= 16 & \\}
\onslide<3->{(x-5)^2 &= 4096 & \\}
\onslide<4->{x - 5 &= 64 & \\ x &= 69}
\end{flalign*}
\end{frame}

\begin{frame}[t]{Rational Equations}
We solved rational equations before, but now we've seen quadratic equations so we can look at that type of rational equation.

\onslide<2->{Example: Solve $\dfrac{4}{x} + \dfrac{2}{x+3} = -3$ and check your solution(s).}
\begin{flalign*}
\onslide<3->{x(x+3)\dfrac{4}{x} + x(x+3)\dfrac{2}{x+3} &= x(x+3)(-3) & \\}
\onslide<4->{4(x+3) + 2x &= -3x^2 - 9x & \\}
\onslide<5->{-3x^2 - 15x - 12 &= 0 & \\}
\onslide<6->{-3(x^2 + 5x + 4) = 0 & \\}
\onslide<7->{(x+4)(x+1) &= 0 & \\ x&= \set{-4, -1}}
\end{flalign*}
\end{frame}

\begin{frame}[t]{Rational Equations}
We found answers of $-4$ and $-1$. Looking at the original equation $$\dfrac{4}{x} + \dfrac{2}{x+3} = -3$$ both of these numbers are in the domain, so neither is an extraneous solution. I'll leave it to you to plug them in and verify that they yield true statements.
\end{frame}

\begin{frame}[t]{Next Steps}
\begin{itemize}
\item Post questions in the Lesson 6 Forum, if you have them
\item Complete Assignment \#3
\item Begin Module \#4
\begin{itemize}
\item Read 1.7 and 1.8
\item Watch Video Lesson \#7
\end{itemize}
\end{itemize}

\vfill

Thanks for watching!
\end{frame}

\end{document}