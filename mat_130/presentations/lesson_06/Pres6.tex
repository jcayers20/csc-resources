\documentclass{beamer}
\usepackage[utf8]{inputenc}
\usepackage{amsmath,amsfonts,amsthm,amstext,amssymb, xcolor, tikz, pgf}

% ----------------------------------------------------------
% Theme Setup

% Use Metropolis Theme
\usetheme[numbering=fraction]{metropolis}
\setbeamertemplate{blocks}[rounded][shadow=false]
\makeatletter
\setlength{\metropolis@titleseparator@linewidth}{1pt}
\makeatother

% Define Colors
\definecolor{chargerblue}{HTML}{002764}
\definecolor{chargerred}{HTML}{e02034}
\definecolor{bggray}{HTML}{d0d3d4}

% Set Colors
\setbeamercolor{title}{fg=chargerblue}
\setbeamercolor{background canvas}{bg=white}
\setbeamercolor{title separator}{fg=chargerred}
\setbeamercolor{structure}{fg=chargerblue}
\setbeamercolor{frametitle}{fg=white, bg=chargerblue}
\setbeamercolor*{normal text}{fg=chargerblue}
\setbeamercolor*{block body}{bg=bggray}
\setbeamercolor*{block title}{bg=chargerblue, fg=white}
% ----------------------------------------------------------

% ----------------------------------------------------------
% Custom Definitions, Commands, Environments, etc.

% Sets of numbers
\def\R{\mathbb{R}} % The reals
\def\N{\mathbb{N}} % The naturals
\def\Z{\mathbb{Z}} % The integers
\def\Q{\mathbb{Q}} % The rationals

% Blank space
\newcommand{\blank}[1]{\underline{\hspace{#1}}} % Blank space

% Fitted inclusion symbols
\newcommand{\fp}[1]{\left({#1}\right)} % Fitted parentheses around content
\newcommand{\fb}[1]{\left[{#1}\right]} % Fitted brackets
\newcommand{\set}[1]{\left\{{#1}\right\}} % Fitted braces (useful for sets)
\newcommand{\av}[1]{\left|{#1}\right|} % Fitted absolute value bars



% Coordinate Plane (Four-Quadrant)
\def\coordplane {
	\begin{tikzpicture}		\draw[step=0.25cm,black,very thin,opacity=0.25] (-2.5cm, -2.5cm) grid (2.5cm, 2.5cm);
		\draw[<->,thick,black] (-2.5cm, 0) -- (2.5cm, 0) node[anchor=north west,pos=0.94,font=\scriptsize]{$x$};
		\draw[<->,thick,black] (0,-2.5cm) -- (0, 2.5cm) node[anchor=south east,font=\scriptsize,pos=0.94]{$y$};
	\end{tikzpicture}
}

% Coordinate Plane (One-Quadrant)
\def\onequad {
	\begin{tikzpicture}
		\draw[step=0.25cm, black, very thin, opacity=0.25] (0,0) grid (7.5cm,5cm);
		\draw[->, thick, black] (0,0) -- (7.5cm, 0) node[anchor=north west,font=\scriptsize,pos=0.94]{$x$};
		\draw[->, black, thick] (0,0) -- (0,5cm) node[anchor=south east,font=\scriptsize,pos=0.94]{$y$};
	\end{tikzpicture}
}
% ----------------------------------------------------------

% ----------------------------------------------------------
% Presentation Information 
\title[1.5 and 1.6]{Complex Numbers; Other Types of Equation}
\subtitle{Sections 1.5 and 1.6}
\author{Jacob Ayers}
\institute{Lesson \#6}
\date{MAT 130}
% ----------------------------------------------------------

\begin{document}

% Slide 1 (Title Slide)
\begin{frame}
\titlepage
\end{frame}

% Slide 2 (Objectives)
\begin{frame}[t]{Objectives}
\begin{itemize}
	\item Use the imaginary unit $i$ to write complex numbers
	\item Perform operations on complex numbers
	\item Find complex solutions to quadratic equations
	\item Solve polynomial equations of degree three or higher
	\item Solve radical equation
	\item Solve rational equations
\end{itemize}
\end{frame}

\begin{frame}[t]{The Imaginary Unit $i$ and Complex Numbers}
Not all equations have \textit{real} solutions.

Example: $x^2 = -4$

\pause

\begin{block}{Definition}
We define the \textit{imaginary unit} $i$ by $i = \sqrt{-1}$
\end{block}

\pause

\begin{block}{Definition}
A \textit{complex number} is a number of the form $a + bi$. We call $a$ the \textit{real part} and $bi$ the \textit{imaginary part}.

If $b = 0$, then $a+bi \in \R$. If $a,b\neq 0$, then $a + bi$ is an \textit{imaginary number}. If $a = 0$ and $b\neq 0$, then $a + bi$ is a \textit{pure imaginary number}.
\end{block}
\end{frame}

\begin{frame}[t]{Operations with Complex Numbers}
To add or subtract complex numbers, combine like terms (real is like to real, imaginary is like to imaginary).

\pause

To multiply complex numbers, use FOIL (bear in mind that $i^2 = -1$).

\pause
To divide complex numbers, multiply top and bottom by the conjugate of the denominator.
\end{frame}

\begin{frame}[t]{Adding and Subtracting Complex Numbers}
Example: $(3 + 4i) + (7 - 6i)$\pause$=10 - 2i$ \vspace{12pt}

\pause

Example: $(-7 - 3i) + (-2 + 13i)$\pause$= -9 + 10i$ \vspace{12pt}

\pause

Example: $(3 + 4i) - (7 - 6i)$\pause$= -4 + 10i$ \vspace{12pt}

\pause

Example: $(-7 - 3i) - (-2 + 13i)$\pause$= -5 - 16i$
\end{frame}

\begin{frame}[t]{Multiplying Complex Numbers}
Example: Find the product of $6 - 2i$ and $3 + 5i$.

\vfill \pause

Example: $(-6-i)(-4-9i)$
\end{frame}

\begin{frame}[t]{Dividing Complex Numbers}
Example: $\dfrac{5 + 8i}{2 - 3i}$

\vfill \pause

Example: $\dfrac{4 - 2i}{2 + 3i}$
\end{frame}

\begin{frame}[t]{Complex Solutions of Quadratic Equations}
Whenever $b^2 - 4ac < 0$, a quadratic equation has complex solutions.

Example: Solve $8x^2 + 14x + 9 = 0$
\end{frame}

\begin{frame}[t]{Polynomial Equations}
We can solve certain polynomial equations whose degree is 3 or higher using strategies we've learned previously.

Example: Solve $9x^4 - 12x^2 = 0$

\vfill \pause

Solve $x^3 - 5x^2 - 2x + 10 = 0$
\end{frame}

\begin{frame}[t]{Polynomial Equations}
Solve: $9x^4 - 37x^2 + 4 = 0$
\end{frame}

\begin{frame}[t]{Radical Equations}
The inverse of taking an $n$th root is taking an $n$th power. We can use this fact to solve radical equations.

\pause

Example: Solve for $x$: $\sqrt{2x + 7} - x = 2$

\vfill \pause

Solve $(x-5)^{2/3} = 16$
\end{frame}

\begin{frame}[t]{Rational Equations}
We solved rational equations before, but now we've seen quadratic equations so we can look at that type of rational equation.

\pause

Example: Solve $\dfrac{4}{x} + \dfrac{2}{x+3} = -3$ and check your solution(s).
\end{frame}

\begin{frame}[t]{Rational Equations}
We found answers of $-4$ and $-1$. Looking at the original equation $$\dfrac{4}{x} + \dfrac{2}{x+3} = -3$$ both of these numbers are in the domain, so neither is an extraneous solution. I'll leave it to you to plug them in and verify that they yield true statements.
\end{frame}

\begin{frame}[t]{Next Steps}
\begin{itemize}
\item Complete Assignment \#3
\item Begin Module \#4
\begin{itemize}
\item Read 1.7 and 1.8
\item Watch Video Lesson \#7
\end{itemize}
\end{itemize}

\vfill

Thanks for watching!
\end{frame}

\end{document}