\documentclass[12pt]{letter}
\usepackage{amsmath,amsfonts,amsthm,amstext,amssymb,graphicx, multicol,fancyhdr,lastpage,fullpage,framed,fancybox,enumerate,tikz,color,mathrsfs, polynom, pifont, tabto}
\usepackage[margin=0.6in,headsep=3pt, headheight=15pt]{geometry}

% ----------------------------------------------------------
% Custom Definitions, Commands, Environments, etc.

% Sets of numbers
\def\R{\mathbb{R}} % The reals
\def\N{\mathbb{N}} % The naturals
\def\Z{\mathbb{Z}} % The integers
\def\Q{\mathbb{Q}} % The rationals

% Blank space
\newcommand{\blank}[1]{\underline{\hspace{#1}}} % Blank space

% Change font colors
\newcommand{\cyan}[1]{{\color{cyan}{#1}}} % Changes font to cyan
\newcommand{\red}[1]{{\color{red}{#1}}} % Changes font to red
\newcommand{\magenta}[1]{{\color{magenta}{#1}}} % Changes font to magenta
\newcommand{\orange}[1]{{\color{orange}{#1}}} % Changes font to orange
\newcommand{\yellow}[1]{{\color{yellow}{#1}}} % Changes font to yellow
\newcommand{\violet}[1]{{\color{violet}{#1}}} % Changes font to violet
\newcommand{\green}[1]{{\color{green}{#1}}} % Changes font to green
\newcommand{\blue}[1]{{\color{blue}{#1}}} % Changes font to blue
\newcommand{\white}[1]{{\color{white}{#1}}} % Changes font to white

% Fitted inclusion symbols
\newcommand{\fp}[1]{\left({#1}\right)} % Fitted parentheses around content
\newcommand{\fb}[1]{\left[{#1}\right]} % Fitted brackets
\newcommand{\lhoi}[1]{\left({#1}\right]} % Left half-open interval
\newcommand{\rhoi}[1]{\left[{#1}\right)} % Right half-open interval
\newcommand{\set}[1]{\left\{{#1}\right\}} % Fitted braces (useful for sets)
\newcommand{\av}[1]{\left|{#1}\right|} % Fitted absolute value bars

% Augmented Matrix Environment
\newenvironment{amatrix}[1]{%
	\left[\begin{array}{@{}*{#1}{c}|c@{}}
	}{%
	\end{array}\right]
}

% Miscellaneous
\def\then{\Rightarrow}
\def\to{\rightarrow}
\def\d{^{\circ}}
\newcommand{\?}{\stackrel{?}{=}}
\newcommand{\cmark}{\text{ \ding{51}}}
\newcommand{\xmark}{\text{ \ding{55}}}



% Coordinate Plane (Four-Quadrant)
\def\coordplane {
	\begin{tikzpicture}		\draw[step=0.25cm,black,very thin,opacity=0.25] (-2.5cm, -2.5cm) grid (2.5cm, 2.5cm);
	\draw[<->,thick,black] (-2.5cm, 0) -- (2.5cm, 0) node[anchor=north west,pos=0.94,font=\scriptsize]{$x$};
	\draw[<->,thick,black] (0,-2.5cm) -- (0, 2.5cm) node[anchor=south east,font=\scriptsize,pos=0.94]{$y$};
	\end{tikzpicture}
}

% Coordinate Plane (One-Quadrant)
\def\onequad {
	\begin{tikzpicture}
	\draw[step=0.25cm, black, very thin, opacity=0.25] (0,0) grid (7.5cm,5cm);
	\draw[->, thick, black] (0,0) -- (7.5cm, 0) node[anchor=north west,font=\scriptsize,pos=0.94]{$x$};
	\draw[->, black, thick] (0,0) -- (0,5cm) node[anchor=south east,font=\scriptsize,pos=0.94]{$y$};
	\end{tikzpicture}
}

% Counters
\newcounter{exercise}

% Exercise environment (auto-numbered)
\newenvironment{exercise}[1][]{\begin{framed}\refstepcounter{exercise}\textbf{Exercise~\theexercise:} #1}{\end{framed}}

% Book exercise environment
\newenvironment{bex}[2][] {
	\begin{framed}
		\textbf{Book Exercise {#2}}#1
	\end{framed}
}
% ----------------------------------------------------------

% ----------------------------------------------------------
% Header and Footer Information
% \pagestyle{fancy}
% \fancyhf{}
% \renewcommand{\headrulewidth}{0pt}
% \rhead{Name: \blank{2in}}
% \lhead{@}
% \rfoot{Page \thepage \, of \,\pageref{LastPage}}
% ----------------------------------------------------------
\author{Jacob Ayers}

\begin{document}
	\textbf{Assignment 13 Key \\ MAT 130}
	
	\begin{bex}{10.2.6}
		{
			
		}
	\end{bex} \vspace{-8pt}
	
	% My answer here
	$x = 13$ \\
	$3y = 12 \then y = 4$
	
	\vfill % \newpage
	
	\begin{bex}{10.2.12}
		{
			
		}
	\end{bex} \vspace{-8pt}
	
	% My answer here
	(a) Undefined (dimensions do not match)
	
	(b) Undefined (dimensions do not match)
	
	(c) $3\begin{bmatrix}
	3 \\ 2 \\ -1
	\end{bmatrix} = \begin{bmatrix}
	3\cdot 3 \\ 3 \cdot 2 \\ 3 \cdot -1
	\end{bmatrix} = \begin{bmatrix}
	9 \\ 6 \\ -3
	\end{bmatrix}$
	
	(d) Undefined (dimensions do not match)
	
	\vfill % \newpage
	
	\begin{bex}{10.2.14}
		{
			
		}
	\end{bex} \vspace{-8pt}
	
	% My answer here
	(a) $\begin{bmatrix}
	1 + (-2) & -1 + 0 & 3 + (-5) \\ 0 + (-3) & 6 + 4 & 9 + (-7)
	\end{bmatrix} = \begin{bmatrix}
	-1 & -1 & -2 \\ -3 & 10 & 2
	\end{bmatrix}$
	
	(b) $\begin{bmatrix}
	1 - (-2) & -1 - 0 & 3 - (-5) \\ 0 - (-3) & 6 - 4 & 9 - (-7)
	\end{bmatrix} = \begin{bmatrix}
	3 & -1 & 8 \\ 3 & 2 & 16
	\end{bmatrix}$
	
	(c) $\begin{bmatrix}
	3(1) & 3(-1) & 3(3) \\ 3(0) & 3(6) & 3(9)
	\end{bmatrix} = \begin{bmatrix}
	3 & -3 & 9 \\ 0 & 18 & 27
	\end{bmatrix}$
	
	(d) $\begin{bmatrix}
	3 & -3 & 9 \\ 0 & 18 & 27
	\end{bmatrix} - \begin{bmatrix}
	-4 & 0 & -10 \\ -6 & 8 & -14
	\end{bmatrix} = \begin{bmatrix}
	3 - (-4) & -3 - 0 & 9 - (-10) \\ 0 - (-6) & 18 - 8 & 27 - (-14)
	\end{bmatrix} = \begin{bmatrix}
	7 & -3 & 19 \\ 6 & 10 & 41
	\end{bmatrix}$
	
	\vfill % \newpage
	
	\begin{bex}{10.2.16}
		{
			
		}
	\end{bex} \vspace{-8pt}
	
	% My answer here
	(a) $\begin{bmatrix}
		-4 & 9 & 1 \\ 5 & -6 & -5 \\ 15 & -5 & -2 \\ 3 & 10 & -10 \\ -4 & 0 & -2
	\end{bmatrix}$ \tabto{0.5\textwidth} (b) $\begin{bmatrix}
	2 & -1 & -1 \\ 1 & 2 & 9 \\ -5 & 13 & 0 \\ -3 & 6 & -2 \\ -4 & -2 & 2
	\end{bmatrix}$
	
	(c) $\begin{bmatrix}
	-3 & 12 & 0 \\ 9 & -6 & 6 \\ 15 & 12 & -3 \\ 0 & 24 & -18 \\ -12 & -3 & 0
	\end{bmatrix}$ \tabto{0.5\textwidth} (d) $\begin{bmatrix}
	3 & 2 & -2 \\ 5 & 2 & 20 \\ -5 & 30 & -1 \\ -6 & 20 & -10 \\ -12 & -5 & 4
	\end{bmatrix}$
	
	\vfill \newpage
	
	\begin{bex}{10.2.20}
		{
			
		}
	\end{bex} \vspace{-8pt}
	
	% My answer here
	$\begin{bmatrix}
	\dfrac{19}{2} & 2 & -7 & \dfrac92
	\end{bmatrix}$
	
	\vfill % \newpage
	
	\begin{bex}{10.2.22}
		{
			
		}
	\end{bex} \vspace{-8pt}
	
	% My answer here
	$\begin{bmatrix}
	-\frac{11}{3} & -\frac{31}{3} \\ 1 & \frac32 \\ -8 & -1
	\end{bmatrix}$
	
	\vfill % \newpage
	
	\begin{bex}{10.2.28}
		{
			
		}
	\end{bex} \vspace{-8pt}
	
	% My answer here
	$\begin{bmatrix}
	-6 & -1 & 17 \\ -9 & 0 & 10
	\end{bmatrix}$
	
	\vfill % \newpage
	
	\begin{bex}{10.2.36}
		{
			
		}
	\end{bex} \vspace{-8pt}
	
	% My answer here
	The dimension of A is $3 \times 3$, and the dimension of B is $3 \times 2$. Since the number of columns in A is equal to the number of rows in B, we can compute the product $AB$. \begin{flalign*}
	\begin{bmatrix}
	0 & -1 & 2 \\ 6 & 0 & 3 \\ 7 & -1 & 8
	\end{bmatrix}\begin{bmatrix}
	2 & -1 \\ 4 & -5 \\ 1 & 6
	\end{bmatrix} &= \begin{bmatrix}
	0(2) + (-1)(4) + 2(1) & 0(-1) + (-1)(-5) + 2(6) \\ 6(2) + 0(4) + 3(1) & 6(-1) + 0(-5) + 3(6) \\ 7(2) + (-1)(4) + 8(1) & 7(-1) + (-1)(-5) + 8(6)
	\end{bmatrix} & \\
	&= \begin{bmatrix}
	-2 & 17 \\ 15 & 12 \\ 18 & 46
	\end{bmatrix}
	\end{flalign*}
	
	\vfill % \newpage
	
	\begin{bex}{10.2.38}
		{
			
		}
	\end{bex} \vspace{-8pt}
	
	% My answer here
	The dimension of A is $2 \times 4$ and the dimension of B is $2 \times 2$. Since the number of columns in A is not equal to the number of rows in B, it is not possible to compute the product $AB$.
	
	\vfill % \newpage
	
	\begin{bex}{10.2.40}
		{
			
		}
	\end{bex} \vspace{-8pt}
	
	% My answer here
	$\begin{bmatrix}
	0 & 0 & 0 \\ 0 & 0 & 0 \\ 0 & 0 & 0
	\end{bmatrix}$
	
	\vfill \newpage
	
	\begin{bex}{10.2.46}
		{
			
		}
	\end{bex} \vspace{-8pt}
	
	% My answer here
	(a) $AB = \begin{bmatrix}
	5 & -9 & 0 \\ 3 & 0 & -8 \\ -1 & 4 & 11
	\end{bmatrix}$
	
	(b) $BA = \begin{bmatrix}
	5 & -9 & 0 \\ 3 & 0 & -8 \\ -1 & 4 & 11
	\end{bmatrix}$
	
	(c) $A^2 = AA = \begin{bmatrix}
	-2 & -45 & 72 \\ 23 & -59 & -88 \\ -4 & 53 & 89
	\end{bmatrix}$
	
	\vfill % \newpage
	
	\begin{bex}{10.4.8}
		{
			
		}
	\end{bex} \vspace{-8pt}
	
	% My answer here
	$-9(-2) - 0(6) = 18$
	
	\vfill % \newpage
	
	\begin{bex}{10.4.12}
		{
			
		}
	\end{bex} \vspace{-8pt}
	
	% My answer here
	$4(0) - (-3)(0) = 0$
	
	\vfill % \newpage
	
	
	\begin{bex}{10.4.16}
		{
			
		}
	\end{bex} \vspace{-8pt}
	
	% My answer here
	$4(5) - 7(-2) = 34$
	
	\vfill % \newpage
	
	\begin{bex}{10.4.20}
		{
			
		}
	\end{bex} \vspace{-8pt}
	
	% My answer here
	$0(2) - 2.5(-3) = 7.5$
	
	\vfill \newpage
	
	\begin{bex}{10.2.32}
		{
			
		}
	\end{bex} \vspace{-8pt}
	
	% My answer here
	$M_{11} = 1\begin{vmatrix}
		2 & 5 \\ -6 & 4
		\end{vmatrix} = 38$ \tabto{0.33\textwidth}
	$M_{12} = \begin{vmatrix}
	3 & 5 \\ 4 & 4
	\end{vmatrix} = -8$ \tabto{0.66\textwidth} $M_{13} = \begin{vmatrix}
	3 & 2 \\ 4 & -6
	\end{vmatrix} = -28$
	
	$M_{21} = \begin{vmatrix}
	-1 & 0 \\ -6 & 4
	\end{vmatrix} = -4$ \tabto{0.33\textwidth} $M_{22} = \begin{vmatrix}
	1 & 0 \\ 4 & 4
	\end{vmatrix} = 4$ \tabto{0.66\textwidth} $M_{23} = \begin{vmatrix}
	1 & -1 \\ 4 & -6
	\end{vmatrix} = -2$
	
	$M_{31} = \begin{vmatrix}
	-1 & 0 \\ 2 & 5
	\end{vmatrix} = -5$ \tabto{0.33\textwidth} $M_{32} = \begin{vmatrix}
	1 & 0 \\ 3 & 5
	\end{vmatrix} = 5$ \tabto{0.66\textwidth} $M_{33} = \begin{vmatrix}
	1 & -1 \\ 3 & 2
	\end{vmatrix} = 5$
	
	To find cofactors, add up the digits of the subscripts. When the sum is odd, switch the sign. When the sum is even, do not switch the sign.
	
	\begin{tabular}{ccc}
		$C_{11} =  38$ & $C_{12} = 8$ & $C_{13} = -28$ \\ $C_{21} = 4$ & $C_{22} = 4$ & $C_{23} = 2$ \\	
		$C_{31} = -5$ & $C_{32} = -5$ & $C_{33} = 5$
	\end{tabular}
	
	\vfill % \newpage
	
	\begin{bex}{10.2.34}
		{
			
		}
	\end{bex} \vspace{-8pt}
	
	% My answer here
	$M_{11} = \begin{vmatrix}
	-6 & 0 \\ 7 & -6
	\end{vmatrix} = 36$ \tabto{0.33 \textwidth} $M_{12} = \begin{vmatrix}
	7 & 0 \\ 6 & -6
	\end{vmatrix} = -42$ \tabto{0.66 \textwidth} $M_{13} = \begin{vmatrix}
	7 & -6 \\ 6 & 7
	\end{vmatrix} = 85$
	
	$M_{21} = \begin{vmatrix}
	9 & 4 \\ 7 & -6
	\end{vmatrix} = -82$ \tabto{0.33 \textwidth} $M_{22} = \begin{vmatrix}
	-2 & 4 \\ 6 & -6
\end{vmatrix} = -12$ \tabto{0.66 \textwidth} $M_{23} = \begin{vmatrix}
	-2 & 9 \\ 6 & 7
\end{vmatrix} = -68$

	$M_{31} = \begin{vmatrix}
	9 & 4 \\ -6 & 0
\end{vmatrix} = 24$ \tabto{0.33 \textwidth} $M_{32} = \begin{vmatrix}
	-2 & 4 \\ 7 & 0
\end{vmatrix} = -28$ \tabto{0.66 \textwidth} $M_{33} = \begin{vmatrix}
	-2 & 9 \\ 7 & -6
\end{vmatrix} = -51$

	\begin{tabular}{ccc}
		$C_{11} =  36$ & $C_{12} = 42$ & $C_{13} = 85$ \\ $C_{21} = 82$ & $C_{22} = -12$ & $C_{23} = 68$ \\	
		$C_{31} = 24$ & $C_{32} = 28$ & $C_{33} = -51$
	\end{tabular}
	
	\vfill % \newpage
	
	\begin{bex}{10.4.46}
		{
			
		}
	\end{bex} \vspace{-8pt}
	
	% My answer here
	$-5$
	
	\vfill % \newpage
	
	\begin{bex}{10.4.50}
		{
			
		}
	\end{bex} \vspace{-8pt}
	
	% My answer here
	$3$
	
	\vfill % \newpage
	
	\begin{bex}{10.4.54}
		{
			
		}
	\end{bex} \vspace{-8pt}
	
	% My answer here
	$0$
	
	\vfill % \newpage
	
	\begin{bex}{10.4.58}
		{
			
		}
	\end{bex} \vspace{-8pt}
	
	% My answer here
	$-150$
	
	\vfill
		
\end{document}