\documentclass[t]{beamer}
\usepackage{amsmath,amsfonts,amsthm,amstext,amssymb, xcolor, tikz, pgf, mathrsfs, polynom, pifont, tabto}

% ----------------------------------------------------------
% Theme Setup

% Use Metropolis Theme
\usetheme[numbering=fraction]{metropolis}
\setbeamertemplate{blocks}[rounded][shadow=false]
\makeatletter
\setlength{\metropolis@titleseparator@linewidth}{1pt}
\makeatother

% Define Colors
\definecolor{chargerblue}{HTML}{002764}
\definecolor{chargerred}{HTML}{e02034}
\definecolor{bggray}{HTML}{d0d3d4}

% Set Colors
\setbeamercolor{title}{fg=chargerblue}
\setbeamercolor{background canvas}{bg=white}
\setbeamercolor{title separator}{fg=chargerred}
\setbeamercolor{structure}{fg=chargerblue}
\setbeamercolor{frametitle}{fg=white, bg=chargerblue}
\setbeamercolor*{normal text}{fg=chargerblue}
\setbeamercolor*{block body}{bg=bggray}
\setbeamercolor*{block title}{bg=chargerblue, fg=white}
% ----------------------------------------------------------

% ----------------------------------------------------------
% Custom Definitions, Commands, Environments, etc.

% Sets of numbers
\def\R{\mathbb{R}} % The reals
\def\N{\mathbb{N}} % The naturals
\def\Z{\mathbb{Z}} % The integers
\def\Q{\mathbb{Q}} % The rationals

% Blank space
\newcommand{\blank}[1]{\underline{\hspace{#1}}} % Blank space

% Change font colors
\newcommand{\cyan}[1]{{\color{cyan}{#1}}} % Changes font to cyan
\newcommand{\red}[1]{{\color{red}{#1}}} % Changes font to red
\newcommand{\magenta}[1]{{\color{magenta}{#1}}} % Changes font to magenta
\newcommand{\orange}[1]{{\color{orange}{#1}}} % Changes font to orange
\newcommand{\yellow}[1]{{\color{yellow}{#1}}} % Changes font to yellow
\newcommand{\violet}[1]{{\color{violet}{#1}}} % Changes font to violet
\newcommand{\green}[1]{{\color{green}{#1}}} % Changes font to green
\newcommand{\blue}[1]{{\color{blue}{#1}}} % Changes font to blue
\newcommand{\white}[1]{{\color{white}{#1}}} % Changes font to white

% Fitted inclusion symbols
\newcommand{\fp}[1]{\left({#1}\right)} % Fitted parentheses around content
\newcommand{\fb}[1]{\left[{#1}\right]} % Fitted brackets
\newcommand{\lhoi}[1]{\left({#1}\right]} % Left half-open interval
\newcommand{\rhoi}[1]{\left[{#1}\right)} % Right half-open interval
\newcommand{\set}[1]{\left\{{#1}\right\}} % Fitted braces (useful for sets)
\newcommand{\av}[1]{\left|{#1}\right|} % Fitted absolute value bars

% Augmented Matrix Environment
\newenvironment{amatrix}[1]{%
	\left[\begin{array}{@{}*{#1}{c}|c@{}}
	}{%
	\end{array}\right]
}

% Miscellaneous
\def\then{\Rightarrow}
\def\to{\rightarrow}
\def\d{^{\circ}}
\newcommand{\?}{\stackrel{?}{=}}
\newcommand{\cmark}{\text{ \ding{51}}}
\newcommand{\xmark}{\text{ \ding{55}}}

% Coordinate Plane (Four-Quadrant)
\def\coordplane {
	\begin{tikzpicture}		\draw[step=0.25cm,black,very thin,opacity=0.25] (-2.5cm, -2.5cm) grid (2.5cm, 2.5cm);
	\draw[<->,thick,black] (-2.5cm, 0) -- (2.5cm, 0) node[anchor=north west,pos=0.94,font=\scriptsize]{$x$};
	\draw[<->,thick,black] (0,-2.5cm) -- (0, 2.5cm) node[anchor=south east,font=\scriptsize,pos=0.94]{$y$};
	\end{tikzpicture}
}

% Coordinate Plane (One-Quadrant)
\def\onequad {
	\begin{tikzpicture}
	\draw[step=0.25cm, black, very thin, opacity=0.25] (0,0) grid (7.5cm,5cm);
	\draw[->, thick, black] (0,0) -- (7.5cm, 0) node[anchor=north west,font=\scriptsize,pos=0.94]{$x$};
	\draw[->, black, thick] (0,0) -- (0,5cm) node[anchor=south east,font=\scriptsize,pos=0.94]{$y$};
	\end{tikzpicture}
}
% ----------------------------------------------------------

% ----------------------------------------------------------
% Presentation Information 
\title[Abbr]{Data Collection and Sampling Techniques; Experimental Design}
\subtitle{1.3 and 1.4}
\author{Jacob Ayers}
\institute{Lesson \#2}
\date{MAT 110}
% ----------------------------------------------------------

\begin{document}
	
	% Slide 1 (Title Slide)
	\begin{frame}
		\titlepage
	\end{frame}
	
	% Slide 2 (Objectives)
	\begin{frame}{Objectives}
		\begin{itemize}
			\item Identify sampling techniques
			\item Explain the difference between observational studies and experimental studies
			\item Expalin how statistics can be used or misused
		\end{itemize}
	\end{frame}

	\begin{frame}{Data Collection}
		We use data in many ways: \begin{itemize}
			\item determine whether we're meeting goals
			\item determine needs of a population
			\item learn about consumers of a product
			\item gain insight into people's perception of political events
		\end{itemize} \pause
	
		In order to use data, we must first collect it.
		
		There are a variety of ways of collecting data.
	\end{frame}

	\begin{frame}{Data Collection}
		Often, data is collected using a survey.
		
		\textit{Telephone surveys} are surveys that are conducted over the phone. \pause
		
		\underline{Pros:} \begin{itemize}
			\item Relatively cheap
			\item More candid responses (not face-to-face)
		\end{itemize} \pause
	
		\underline{Cons:} \begin{itemize}
			\item Not everyone can/will be reached by phone
			\item Easy for interviewer to influence person being surveyed with tone of voice
		\end{itemize}
	\end{frame}

	\begin{frame}{Data Collection}
		\textit{Mailed questionnaires} are surveys that are sent through the mail. \pause
		
		\underline{Pros:} \begin{itemize}
			\item Even cheaper than telephone surveys
			\item Easily cover wide geographic areas
			\item Respondents can remain anonymous, which leads to more candid responses
		\end{itemize} \pause
	
		\underline{Cons:} \begin{itemize}
			\item Low response rate
			\item Inappropriate answers to questions (anonymity)
			\item Not everyone will be able to read/understand the questions
		\end{itemize}
	\end{frame}

	\begin{frame}{Data Collection}
		\textit{Personal interviews} are surveys in which people are interviewed. \pause
		
		\underline{Pros:} \begin{itemize}
			\item In-depth responses
		\end{itemize}
	
		\underline{Cons:} \begin{itemize}
			\item Costly
			\item Difficult to cover wide geographic area
			\item Potential for bias in selection of respondents
		\end{itemize}
	\end{frame}

	\begin{frame}{Data Collection}
		Which method is best? \pause
		
		Researchers must weigh cost and effectiveness when they choose how to collect data - there are no hard-and-fast rules. \pause
		
		If you need to reach a lot of people, you might choose either a mailed questionnaire or a telephone survey because they are cheaper than interviewing people personally.
		
		If you want in-depth answers, a personal interview may be the best choice.
	\end{frame}

	\begin{frame}{Sampling Techniques}
		How we select our sample is very important - we want to ensure that everyone in the population has an equal chance of being selected.
		
		Example: conducting a survey on a street corner from 8-noon would exclude people who work during that time. The information we obtain is likely to be biased. \pause
		
		There are four main ways that researchers obtain unbiased samples.
	\end{frame}

	\begin{frame}{Sampling Techniques}
		\begin{block}{Random Sampling}
			A \textit{random sample} is a sample in which the members are selected either by chance methods or using random numbers
		\end{block}
	
		One way to do a random sample: put everyone's name in a hat, then draw $x$ names from the hat. \pause
		
		This method has a tendency to be biased, since it's hard to thoroughly mix up the names. \pause
		
		Better way: Random number generator on a computer or a calculator.
	\end{frame}

	\begin{frame}{Sampling Techniques}
		\begin{block}{Systematic Sampling}
			A \textit{systematic sample} is a sample obtained by selecting every $k$th member of the population.
		\end{block}
	
		Example: say there are 1000 members of the population and we want a sample size of 50. \pause
		
		We would select every $\dfrac{1000}{50} = 20$ members. The first member (between 1 and 20) would be selected at random, and then every 20th member after would be selected.
	\end{frame}

	\begin{frame}{Sampling Techniques}
		\begin{block}{Stratfied Sampling}
			A \textit{stratified sample} is a sample obtained by dividing the population into subgroups based on some characteristic, then selecting subjects at random from each subgroup.
		\end{block}
	
		Samples within each subgroup are selected using random sampling; there can be many subgroups.
		
		Example: Say I own ten restaurants and I want to get a feel for the difference in service at each location. \pause
		
		I could randomly select 25 customers from each restaurant to generate a stratified sample.
	\end{frame}

	\begin{frame}{Sampling Techniques}
		\begin{block}{Cluster Sampling}
			A \textit{cluster sample} is a sample obtained by dividing the population into subgroups (as in stratified sampling), then selecting all members of one or more of the clusters as the members of the sample.
		\end{block}
	
		Example: Say I own ten restaurants and I want to know people's thoughts on the menu. \pause
		
		I could randomly select two of the restaurants and survey all the customers who walk in to generate a cluster sample.
	\end{frame}

	\begin{frame}{Sampling Techniques}
		Determine the type of sampling that was used in each case:
		
		\textbf{To check the accuracy of a machine filling coffee cups, every fifth cup is selected and weighed.}
		
		\onslide<2->{Since every fifth cup is measured, this is a systematic sample.}
		
		\textbf{To determine how long people exercise, a researcher interviews 5 people from a yoga class, 5 people from a weightlifting class, 5 people from an aerobics class, and 5 people from a swimming class.}
		
		\onslide<3>{Since members of the sample are selected from various subgroups, this is a stratified sample.}
	\end{frame}

	\begin{frame}{Sampling Techniques}
		Determine the type of sampling that was used in each case:
		
		\textbf{In a large school district, a researcher numbers all of the full-time teachers and randomly selects 30 to be interviewed.}
		
		\onslide<2->{Since the sample is obtained at random, it is a random sample.}
		
		\textbf{For 15 minutes, all customers entering a specific Walmart store on a specific day are asked how many miles from the store they live.}
		
		\onslide<3>{Since all members of the sample come from the same subgroup, this is a cluster sample.}
	\end{frame}

	\begin{frame}{Sampling Techniques}
		Which method is best? \pause
		
		It depends on the population and the researcher's needs.
		
		Random sample: can be used any time the population can be numbered \pause
		
		Systematic Sample: selects subjects in ordered population; fast and convenient when population is easily numbered \pause
		
		Stratified Sample: useful when you want to determine differences between subgroups and on assembly lines \pause
		
		Cluster Sample: can be done quickly and cheaply, but one cluster may not represent the entire population
	\end{frame}

	\begin{frame}{Sampling Techniques}
		No sample is a perfect representation of the population from which it was chosen.
		
		There will always be some degree of error in the results. \pause
		
		\begin{block}{Definition}
			\textit{Sampling error} is the difference obtained from a sample and the results obtained from the population.
		\end{block} \pause
	
		Example: Say you are president of a college whose student population is 60\% female. If I obtain a sample that is 63\% female, then the difference of 3\% is due to sampling error. \pause
		
		Usually, the degree of sampling error is unknown.
	\end{frame}

	\begin{frame}{Sampling Techniques}
		In addition to sampling error, there is nonsampling error.
		
		\begin{block}{Definition}
			A \textit{nonsampling error} occurs when the data are obtained erroneously or the sample is biased.
		\end{block} \pause
	
		Examples of nonsampling error: \begin{itemize}
			\item measuring temperature on a defective thermometer
			\item researcher makes an error in recording a value
		\end{itemize}
	\end{frame}

	\begin{frame}{Experimental Design}
		There are two types of study: \pause
		
		\textit{Observational studies} are studies in which the researcher simply watches what is happening or what has happened in the past and tries to draw conclusions based on these observations. \pause
		
		\textit{Experimental studies} are studies in which the researcher manipulates one variable and tries to determine how that manipulation affects other variables.
	\end{frame}

	\begin{frame}{Observational Studies}
		There are three main types of observational study: \begin{itemize}[<+->]
			\item Cross-sectional studies - data all gathered at one time
			\item Retrospective studies - data gathered from past records
			\item Longitudinal studies - data gathered over time (past and present)
		\end{itemize}
	\end{frame}

	\begin{frame}{Observational Studies}
		Advantages of observational studies: \begin{itemize}
			\item occurs in natural setting - people won't be influenced by the researcher
			\item can be done in situations where performing an experiment would be unethical/dangerous (e.g. studying suicide rate)
			\item can be done using variables that the researcher is unable to manipulate (e.g. smokers vs. non-smokers)
		\end{itemize}
	\end{frame}

	\begin{frame}{Observational Studies}
		Disadvantages of observational studies: \begin{itemize}
			\item unable to determine direct cause-effect relationship between variables
			\item can be costly and time-consuming
			\item when gathering data from outside sources, the accuracy of your results is dependent upon the accuracy of their data
		\end{itemize}
	\end{frame}

	\begin{frame}{Experimental Studies}
		Example: divide students into two groups (randomly assign to each group). Each person does as many push-ups as possible. \pause
		
		One group is given instructions to try to increase the number of push-ups they can do by 10\% each day, while the other is simply told to do their best. \pause
		
		After four days, the test was performed again. \pause
		
		Here, the researchers manipulated one variable (instructions given) and measured the change in another variable (improvement at doing push-ups).
	\end{frame}

	\begin{frame}{Experimental Studies}
		The \textit{independent variable} is the variable that the researcher manipulates. \pause
		
		The \textit{dependent variable} is the variable that changes as a result of the manipulation. \pause
		
		In our previous example, the instructions given were the independent variable and the number of push-ups was the dependent variable.
	\end{frame}

	\begin{frame}{Experimental Studies}
		In our example, the students were divided into two groups - this is something that happens in all experiments \pause
		
		The group that received special instructions is called the \textit{treatment group}, while the group that did not is called the \textit{control group}.
	\end{frame}

	\begin{frame}{Experimental Studies}
		Advantages of experimental studies: \begin{itemize}
			\item researcher gets to decide who is selected and who goes in what group
			\item researcher gets to control the independent variable
		\end{itemize} \pause
	
		Example: Researcher can determine what dosage of a medicine the treatment group receives (control group receives a placebo)
	\end{frame}

	\begin{frame}{Experimental Studies}
		Disadvantages of experimental studies: \begin{itemize}
			\item may occur in unnatural setting - results don't translate
			\item Hawthorne effect - subjects behave differently when they know they're part of an experiment
			\item Confounding variables - variables other than the independent variable may impact results
			\item Placebo effect - even members of control group may respond positively to ``treatment"
		\end{itemize}
	\end{frame}

	\begin{frame}{Experimental Design}
		Researchers randomly assigned 10 people to each of three different groups. Group 1 was instructed to write an essay about the hassles in their lives. Group 2 was instructed to write an essay about circumstances that made them feel thankful. Group 3 was asked to write an essay about events they felt neutral about. After the exercise, they were given a questionnaire on their outlook on life. The researchers found that those who wrote about circumstances that made them feel thankful had a more optimistic outlook on life. The conclusion is that focusing on the positive makes you more optimistic about life in general.
	\end{frame}

	\begin{frame}{Experimental Design}
		\textbf{Was this an observational study or an experimental study?} \\
		\onslide<2->{The researcher had control over which type of essay each participant wrote, so this is an experimental study.}
		
		\textbf{What is the independent variable?} \\
		\onslide<3->{The independent variable is the type of essay written.}
		
		\textbf{What is the dependent variable?} \\
		\onslide<4->{The dependent variable is the result of the questionnaire.}
		
		\textbf{What may be a confounding variable?} \\
		\onslide<5->{There are many possible confounding variables. For example, income is not controlled for, and it could certainly have an impact on one's outlook on life.}
	\end{frame}

	\begin{frame}{Experimental Design}
		\textbf{What can you say about the sample size?} \\
		\onslide<2->{The sample size is $10 \times 3 = 30$.}
		
		\textbf{Do you agree with the conclusion?} \\
		\onslide<3->{This answer is subjective.}
	\end{frame}

	\begin{frame}{Next Steps}
		\begin{itemize}
			\item Complete Assignment \#1
			\item Begin Module \#2
			\begin{itemize}
				\item Read 2.1 and 2.2
				\item Watch Video Lesson \#3
			\end{itemize}
		\end{itemize}
	
		\vfill
		
		Thanks for watching!
	\end{frame}
	
\end{document}