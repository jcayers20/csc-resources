\documentclass[t, aspectratio=169]{beamer}
\usepackage{amsmath,amsfonts,amsthm,amstext,amssymb, xcolor, tikz, pgf, mathrsfs, polynom, pifont, tabto}

% ----------------------------------------------------------
% Theme Setup

% Use Metropolis Theme
\usetheme[numbering=fraction]{metropolis}
\setbeamertemplate{blocks}[rounded][shadow=false]
\makeatletter
\setlength{\metropolis@titleseparator@linewidth}{1pt}
\makeatother

% Define Colors
\definecolor{chargerblue}{HTML}{002764}
\definecolor{chargerred}{HTML}{e02034}
\definecolor{bggray}{HTML}{d0d3d4}

% Set Colors
\setbeamercolor{title}{fg=chargerblue}
\setbeamercolor{background canvas}{bg=white}
\setbeamercolor{title separator}{fg=chargerred}
\setbeamercolor{structure}{fg=chargerblue}
\setbeamercolor{frametitle}{fg=white, bg=chargerblue}
\setbeamercolor*{normal text}{fg=chargerblue}
\setbeamercolor*{block body}{bg=bggray}
\setbeamercolor*{block title}{bg=chargerblue, fg=white}
% ----------------------------------------------------------

% ----------------------------------------------------------
% Custom Definitions, Commands, Environments, etc.

% Sets of numbers
\def\R{\mathbb{R}} % The reals
\def\N{\mathbb{N}} % The naturals
\def\Z{\mathbb{Z}} % The integers
\def\Q{\mathbb{Q}} % The rationals

% Blank space
\newcommand{\blank}[1]{\underline{\hspace{#1}}} % Blank space

% Change font colors
\newcommand{\cyan}[1]{{\color{cyan}{#1}}} % Changes font to cyan
\newcommand{\red}[1]{{\color{red}{#1}}} % Changes font to red
\newcommand{\magenta}[1]{{\color{magenta}{#1}}} % Changes font to magenta
\newcommand{\orange}[1]{{\color{orange}{#1}}} % Changes font to orange
\newcommand{\yellow}[1]{{\color{yellow}{#1}}} % Changes font to yellow
\newcommand{\violet}[1]{{\color{violet}{#1}}} % Changes font to violet
\newcommand{\green}[1]{{\color{green}{#1}}} % Changes font to green
\newcommand{\blue}[1]{{\color{blue}{#1}}} % Changes font to blue
\newcommand{\white}[1]{{\color{white}{#1}}} % Changes font to white

% Fitted inclusion symbols
\newcommand{\fp}[1]{\left({#1}\right)} % Fitted parentheses around content
\newcommand{\fb}[1]{\left[{#1}\right]} % Fitted brackets
\newcommand{\lhoi}[1]{\left({#1}\right]} % Left half-open interval
\newcommand{\rhoi}[1]{\left[{#1}\right)} % Right half-open interval
\newcommand{\set}[1]{\left\{{#1}\right\}} % Fitted braces (useful for sets)
\newcommand{\av}[1]{\left|{#1}\right|} % Fitted absolute value bars

% Augmented Matrix Environment
\newenvironment{amatrix}[1]{%
	\left[\begin{array}{@{}*{#1}{c}|c@{}}
	}{%
	\end{array}\right]
}

% Miscellaneous
\def\then{\Rightarrow}
\def\to{\rightarrow}
\def\d{^{\circ}}
\newcommand{\?}{\stackrel{?}{=}}
\newcommand{\cmark}{\text{ \ding{51}}}
\newcommand{\xmark}{\text{ \ding{55}}}

% Coordinate Plane (Four-Quadrant)
\def\coordplane {
	\begin{tikzpicture}        \draw[step=0.25cm,black,very thin,opacity=0.25] (-2.5cm, -2.5cm) grid (2.5cm, 2.5cm);
		\draw[<->,thick,black] (-2.5cm, 0) -- (2.5cm, 0) node[anchor=north west,pos=0.94,font=\scriptsize]{$x$};
		\draw[<->,thick,black] (0,-2.5cm) -- (0, 2.5cm) node[anchor=south east,font=\scriptsize,pos=0.94]{$y$};
	\end{tikzpicture}
}

% Coordinate Plane (One-Quadrant)
\def\onequad {
	\begin{tikzpicture}
		\draw[step=0.25cm, black, very thin, opacity=0.25] (0,0) grid (7.5cm,5cm);
		\draw[->, thick, black] (0,0) -- (7.5cm, 0) node[anchor=north west,font=\scriptsize,pos=0.94]{$x$};
		\draw[->, black, thick] (0,0) -- (0,5cm) node[anchor=south east,font=\scriptsize,pos=0.94]{$y$};
	\end{tikzpicture}
}
% ----------------------------------------------------------

% ----------------------------------------------------------
% Presentation Information
\title[7-1b]{Confidence Intervals when $\sigma$ is Known}
\subtitle{Section 7-1}
\author{Jacob Ayers}
\institute{Lesson \#21}
\date{MAT 110}
% ----------------------------------------------------------

\begin{document}
	
	% Slide 1 (Title Slide)
	\begin{frame}
		\titlepage
	\end{frame}
	
	% Slide 2 (Objectives)
	\begin{frame}{Objectives}
		\begin{itemize}
			\item Construct confidence intervals when $\sigma$ is known
			\item Determine the minimum sample size for finding a confidence interval for the mean
		\end{itemize}
	\end{frame}

	\begin{frame}{Constructing Confidence Intervals}
		For a random sample of 60 overweight men, the mean of the number of pounds they were overweight was 30. The standard deviation of the population is 4.2 pounds. Find the 93\% confidence interval of the mean of the excess pounds they weighed. \pause
		
		1) The assumptions hold. This is a random sample, and $60 \geq 30$. \pause \\
		2) Using a calculator, $\dfrac{\sigma}{\sqrt{n}} = \dfrac{4.2}{\sqrt{60}} \approx 0.5422176685$; store this value \pause \\
		3) Look up $1 - 0.07/2 = 0.9650$\pause; $z_{\alpha/2} = 1.81$ \pause
		
		4) Lower Limit for CI: $30 - 1.81(0.5422) \approx 29.02$ \pause \\
		Upper Limit for CI: $30 + 1.81(0.5422) \approx 30.98$ \pause
		
		So the 93\% confidence interval is: $29.02 < \mu < 30.98$
	\end{frame}
	
	\begin{frame}{Determining Sample Size}
		A commonly asked question in statistics: How big a sample do I need to collect in order to make an accurate estimate? \pause
		
		The answer depends on three things: \begin{enumerate}[1)]
			\item How big of a margin of error are you willing to accept? \pause
			\item What is the population standard deviation? \pause
			\item How confident do you want to be? \pause
		\end{enumerate}
		
		We can use the margin of error formula $E = z_{\alpha / 2}\fp{\dfrac{\sigma}{\sqrt{n}}}$ to derive a formula for the sample size needed for a given confidence level and margin of error.
	\end{frame}
	
	\begin{frame}{Determining Sample Size}
		\begin{flalign*}
			\onslide<1->{E &= z_{\alpha / 2}\fp{\dfrac{\sigma}{\sqrt{n}}} &  \text{(margin of error formula)}\\}
			\onslide<2->{E(\sqrt{n}) &= z_{\alpha / 2}\cdot \sigma & \text{(multiply by $\sqrt{n}$)} & \\}
			\onslide<3->{\sqrt{n} &= \dfrac{z_{\alpha / 2} \cdot \sigma}{E} & \text{(divide by $E$)} & \\}
			\onslide<4->{n &= \fp{\dfrac{z_{\alpha / 2 \cdot \sigma}}{E}}^2 & \text{(square each side)}}
		\end{flalign*}
		
		\onslide<5->{Rounding Rule: Always round up to nearest whole number (e.g. $65.03 \uparrow 66$)}
	\end{frame}
	
	\begin{frame}{Determining Sample Size}
		A sociologist wishes to estimate the average number of automobile thefts in a large city per day within 2 automobiles. He wishes to be 99\% confident, and from a previous study the standard deviation was found to be 4.2. How many days should he select to survey?
		
		\onslide<2->{We have $\sigma = 4.2$ and $E = 2$.}
		\onslide<3->{Looking up $1 - 0.01/2 = 0.9950$, we find $z_{\alpha/2} = 2.58$}
		\begin{flalign*}
			\onslide<4->{n &= \fp{\dfrac{z_{\alpha / 2} \cdot \sigma}{E}}^2 & \\}
			\onslide<5->{&= \dfrac{2.58 \cdot 4.2}{2}^2} & \\
			\onslide<6->{&= 5.418^2} & \\
			\onslide<7->{&\approx 29.35}
		\end{flalign*}
		\onslide<8->{So he should sample 30 days.}
	\end{frame}
	
	\begin{frame}{Determining Sample Size}
		A researcher wishes to estimate the average number of minutes per day a person spends on the Internet. How large a sample must she select if she wishes to be 90\% confident that the population mean is within 10 minutes of the sample mean? Assume the population standard deviation is 42 minutes.
		
		\onslide<2->{$\sigma = 42$, $E = 10$; $z_{\alpha/2} = 1.65$}
		\begin{flalign*}
			\onslide<3->{n &= \fp{\dfrac{z_{\alpha/2}\cdot\sigma}{E}}^2 & \\}
			\onslide<4->{&= \fp{\dfrac{1.65(42)}{10}}^2 & \\}
			\onslide<5->{&= 6.93^2} & \\
			\onslide<6->{&\approx 48.02}
		\end{flalign*}
		\onslide<7->{So the sample size should be 49.}
	\end{frame}
	
	\begin{frame}{Next Steps}
		\begin{itemize}
			\item Complete Assignment \#10
			\item Begin Module \#11 \begin{itemize}
				\item Read 7-2
				\item Watch Video Lesson \#22
			\end{itemize}
		\end{itemize}
		
		\vfill
		
		Thanks for watching!
	\end{frame}
	
\end{document}