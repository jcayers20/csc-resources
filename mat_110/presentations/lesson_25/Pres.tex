\documentclass[t, aspectratio=169]{beamer}
\usepackage{amsmath,amsfonts,amsthm,amstext,amssymb, xcolor, tikz, pgf, mathrsfs, polynom, pifont, tabto}

% ----------------------------------------------------------
% Theme Setup

% Use Metropolis Theme
\usetheme[numbering=fraction]{metropolis}
\setbeamertemplate{blocks}[rounded][shadow=false]
\makeatletter
\setlength{\metropolis@titleseparator@linewidth}{1pt}
\makeatother

% Define Colors
\definecolor{chargerblue}{HTML}{002764}
\definecolor{chargerred}{HTML}{e02034}
\definecolor{bggray}{HTML}{d0d3d4}

% Set Colors
\setbeamercolor{title}{fg=chargerblue}
\setbeamercolor{background canvas}{bg=white}
\setbeamercolor{title separator}{fg=chargerred}
\setbeamercolor{structure}{fg=chargerblue}
\setbeamercolor{frametitle}{fg=white, bg=chargerblue}
\setbeamercolor*{normal text}{fg=chargerblue}
\setbeamercolor*{block body}{bg=bggray}
\setbeamercolor*{block title}{bg=chargerblue, fg=white}
% ----------------------------------------------------------

% ----------------------------------------------------------
% Custom Definitions, Commands, Environments, etc.

% Sets of numbers
\def\R{\mathbb{R}} % The reals
\def\N{\mathbb{N}} % The naturals
\def\Z{\mathbb{Z}} % The integers
\def\Q{\mathbb{Q}} % The rationals

% Blank space
\newcommand{\blank}[1]{\underline{\hspace{#1}}} % Blank space

% Change font colors
\newcommand{\cyan}[1]{{\color{cyan}{#1}}} % Changes font to cyan
\newcommand{\red}[1]{{\color{red}{#1}}} % Changes font to red
\newcommand{\magenta}[1]{{\color{magenta}{#1}}} % Changes font to magenta
\newcommand{\orange}[1]{{\color{orange}{#1}}} % Changes font to orange
\newcommand{\yellow}[1]{{\color{yellow}{#1}}} % Changes font to yellow
\newcommand{\violet}[1]{{\color{violet}{#1}}} % Changes font to violet
\newcommand{\green}[1]{{\color{green}{#1}}} % Changes font to green
\newcommand{\blue}[1]{{\color{blue}{#1}}} % Changes font to blue
\newcommand{\white}[1]{{\color{white}{#1}}} % Changes font to white

% Fitted inclusion symbols
\newcommand{\fp}[1]{\left({#1}\right)} % Fitted parentheses around content
\newcommand{\fb}[1]{\left[{#1}\right]} % Fitted brackets
\newcommand{\lhoi}[1]{\left({#1}\right]} % Left half-open interval
\newcommand{\rhoi}[1]{\left[{#1}\right)} % Right half-open interval
\newcommand{\set}[1]{\left\{{#1}\right\}} % Fitted braces (useful for sets)
\newcommand{\av}[1]{\left|{#1}\right|} % Fitted absolute value bars

% Augmented Matrix Environment
\newenvironment{amatrix}[1]{%
	\left[\begin{array}{@{}*{#1}{c}|c@{}}
	}{%
	\end{array}\right]
}

% Miscellaneous
\def\then{\Rightarrow}
\def\to{\rightarrow}
\def\d{^{\circ}}
\newcommand{\?}{\stackrel{?}{=}}
\newcommand{\cmark}{\text{ \ding{51}}}
\newcommand{\xmark}{\text{ \ding{55}}}

% Coordinate Plane (Four-Quadrant)
\def\coordplane {
	\begin{tikzpicture}        \draw[step=0.25cm,black,very thin,opacity=0.25] (-2.5cm, -2.5cm) grid (2.5cm, 2.5cm);
		\draw[<->,thick,black] (-2.5cm, 0) -- (2.5cm, 0) node[anchor=north west,pos=0.94,font=\scriptsize]{$x$};
		\draw[<->,thick,black] (0,-2.5cm) -- (0, 2.5cm) node[anchor=south east,font=\scriptsize,pos=0.94]{$y$};
	\end{tikzpicture}
}

% Coordinate Plane (One-Quadrant)
\def\onequad {
	\begin{tikzpicture}
		\draw[step=0.25cm, black, very thin, opacity=0.25] (0,0) grid (7.5cm,5cm);
		\draw[->, thick, black] (0,0) -- (7.5cm, 0) node[anchor=north west,font=\scriptsize,pos=0.94]{$x$};
		\draw[->, black, thick] (0,0) -- (0,5cm) node[anchor=south east,font=\scriptsize,pos=0.94]{$y$};
	\end{tikzpicture}
}
% ----------------------------------------------------------

% ----------------------------------------------------------
% Presentation Information
\title[8-2]{$z$ Test for a Mean}
\subtitle{Section 8-2}
\author{Jacob Ayers}
\institute{Lesson \#25}
\date{MAT 110}
% ----------------------------------------------------------

\begin{document}
	
	% Slide 1 (Title Slide)
	\begin{frame}
		\titlepage
	\end{frame}
	
	% Slide 2 (Objectives)
	\begin{frame}{Objectives}
		\begin{itemize}
			\item Test means when $\sigma$ is known
		\end{itemize}
	\end{frame}

	\begin{frame}{Steps in Hypothesis Testing}
		\includegraphics[width=\textwidth]{hyp-process.png} \pause
		
		Today, we'll be testing means when we know the population standard deviation. \pause
		
		In this situation, we use the $z$ test to find our critical value(s) and test value.
	\end{frame}

	\begin{frame}{The $z$ Test - Traditional Method}
		Many hypotheses are tested using this general formula: $$\text{Test Value} = \dfrac{\text{(observed)} - \text{(expected)}}{\text{(standard error)}}$$ \pause
		
		Recall: The standard error for the $z$ test is $\sigma / \sqrt{n}$. \pause
		
		\includegraphics[width=0.8\textwidth]{z-formula.png}
	\end{frame}

	\begin{frame}{The $z$ Test - Traditional Method}
		Let's take a look at a few examples.
	\end{frame}

	\begin{frame}{The $z$ Test - $P$-Value Method}
		Computer software often reports a $P$-value when reporting results of hypothesis tests. \pause
		
		\includegraphics[width=\textwidth]{p-val.png} \pause
		
		The $P$-value is the actual area under the curve representing the probability that a value more extreme than the observed value will occur. \pause
		
		If the $P$-value is less than or equal to $\alpha$, we reject $H_0$. If $P > \alpha$, we do not reject $H_0$. \pause
		
		Important: If using a two-tailed test, double the area in one tail. The area found in the $z$ table is the area in only one of the two tails.
	\end{frame}

	\begin{frame}{The $z$ Test - $P$-Value Method}
		Some researchers do not choose an $\alpha$ value, but report a $P$-value and ask the reader to make a decision for themselves whether to reject $H_0$. \pause
		
		Here are the guidelines the textbook recommends using: \pause
		
		\includegraphics[width=\textwidth]{p-rules.png}
	\end{frame}

	\begin{frame}{The $z$ Test - $P$-Value Method}
		Let's take a look at a few examples.
	\end{frame}

	\begin{frame}{Next Steps}
		\begin{itemize}
			\item Complete Assignment 12
			\item Begin Module \#15 \begin{itemize}
				\item Read 8-3
				\item Watch Video Lesson \#26
			\end{itemize}
		\end{itemize}
	
		\vfill
		
		Thanks for watching!
	\end{frame}
	
\end{document}