\documentclass[t, aspectratio=169]{beamer}
\usepackage{amsmath,amsfonts,amsthm,amstext,amssymb, xcolor, tikz, pgf, mathrsfs, polynom, pifont, tabto}

% ----------------------------------------------------------
% Theme Setup

% Use Metropolis Theme
\usetheme[numbering=fraction]{metropolis}
\setbeamertemplate{blocks}[rounded][shadow=false]
\makeatletter
\setlength{\metropolis@titleseparator@linewidth}{1pt}
\makeatother

% Define Colors
\definecolor{chargerblue}{HTML}{002764}
\definecolor{chargerred}{HTML}{e02034}
\definecolor{bggray}{HTML}{d0d3d4}

% Set Colors
\setbeamercolor{title}{fg=chargerblue}
\setbeamercolor{background canvas}{bg=white}
\setbeamercolor{title separator}{fg=chargerred}
\setbeamercolor{structure}{fg=chargerblue}
\setbeamercolor{frametitle}{fg=white, bg=chargerblue}
\setbeamercolor*{normal text}{fg=chargerblue}
\setbeamercolor*{block body}{bg=bggray}
\setbeamercolor*{block title}{bg=chargerblue, fg=white}
% ----------------------------------------------------------

% ----------------------------------------------------------
% Custom Definitions, Commands, Environments, etc.

% Sets of numbers
\def\R{\mathbb{R}} % The reals
\def\N{\mathbb{N}} % The naturals
\def\Z{\mathbb{Z}} % The integers
\def\Q{\mathbb{Q}} % The rationals

% Blank space
\newcommand{\blank}[1]{\underline{\hspace{#1}}} % Blank space

% Change font colors
\newcommand{\cyan}[1]{{\color{cyan}{#1}}} % Changes font to cyan
\newcommand{\red}[1]{{\color{red}{#1}}} % Changes font to red
\newcommand{\magenta}[1]{{\color{magenta}{#1}}} % Changes font to magenta
\newcommand{\orange}[1]{{\color{orange}{#1}}} % Changes font to orange
\newcommand{\yellow}[1]{{\color{yellow}{#1}}} % Changes font to yellow
\newcommand{\violet}[1]{{\color{violet}{#1}}} % Changes font to violet
\newcommand{\green}[1]{{\color{green}{#1}}} % Changes font to green
\newcommand{\blue}[1]{{\color{blue}{#1}}} % Changes font to blue
\newcommand{\white}[1]{{\color{white}{#1}}} % Changes font to white

% Fitted inclusion symbols
\newcommand{\fp}[1]{\left({#1}\right)} % Fitted parentheses around content
\newcommand{\fb}[1]{\left[{#1}\right]} % Fitted brackets
\newcommand{\lhoi}[1]{\left({#1}\right]} % Left half-open interval
\newcommand{\rhoi}[1]{\left[{#1}\right)} % Right half-open interval
\newcommand{\set}[1]{\left\{{#1}\right\}} % Fitted braces (useful for sets)
\newcommand{\av}[1]{\left|{#1}\right|} % Fitted absolute value bars

% Augmented Matrix Environment
\newenvironment{amatrix}[1]{%
	\left[\begin{array}{@{}*{#1}{c}|c@{}}
	}{%
	\end{array}\right]
}

% Miscellaneous
\def\then{\Rightarrow}
\def\to{\rightarrow}
\def\d{^{\circ}}
\newcommand{\?}{\stackrel{?}{=}}
\newcommand{\cmark}{\text{ \ding{51}}}
\newcommand{\xmark}{\text{ \ding{55}}}

% Coordinate Plane (Four-Quadrant)
\def\coordplane {
	\begin{tikzpicture}        \draw[step=0.25cm,black,very thin,opacity=0.25] (-2.5cm, -2.5cm) grid (2.5cm, 2.5cm);
		\draw[<->,thick,black] (-2.5cm, 0) -- (2.5cm, 0) node[anchor=north west,pos=0.94,font=\scriptsize]{$x$};
		\draw[<->,thick,black] (0,-2.5cm) -- (0, 2.5cm) node[anchor=south east,font=\scriptsize,pos=0.94]{$y$};
	\end{tikzpicture}
}

% Coordinate Plane (One-Quadrant)
\def\onequad {
	\begin{tikzpicture}
		\draw[step=0.25cm, black, very thin, opacity=0.25] (0,0) grid (7.5cm,5cm);
		\draw[->, thick, black] (0,0) -- (7.5cm, 0) node[anchor=north west,font=\scriptsize,pos=0.94]{$x$};
		\draw[->, black, thick] (0,0) -- (0,5cm) node[anchor=south east,font=\scriptsize,pos=0.94]{$y$};
	\end{tikzpicture}
}
% ----------------------------------------------------------

% ----------------------------------------------------------
% Presentation Information
\title[4-5]{Probability and Counting Rules}
\subtitle{Section 4-5}
\author{Jacob Ayers}
\institute{Lesson \#13}
\date{MAT 110}
% ----------------------------------------------------------

\begin{document}
	
	% Slide 1 (Title Slide)
	\begin{frame}
		\titlepage
	\end{frame}
	
	% Slide 2 (Objectives)
	\begin{frame}{Objectives}
		\begin{itemize}
			\item Solve probability problems using counting rules
		\end{itemize}
	\end{frame}

	\begin{frame}{Probability and Counting Rules}
		In Video Lessons 9-11, we looked at probability rules. \\
		In Video Lesson 12, we looked at counting rules. \pause
		
		In this lesson, we'll put them together. \pause
		
		Recall: $P(E) = \dfrac{\text{number events in $E$}}{\text{number events in sample space}}$
	\end{frame}

	\begin{frame}{Probability and Counting Rules}
		Find the probability of getting 4 aces when 5 cards are drawn from an ordinary deck of cards. \pause
		
		Note that the order in which the cards are drawn does not matter, so we'll use combinations when necessary. \pause
		
		The number of possibilities in the sample space is $_{52} C _5 = 2598960$, since we're taking 5 cards from a 52-card deck.
		
		There are 4 aces in the deck - we want to get all four of them. There is $_4 C _4 = 1$ way of doing so. \pause
		
		There are 48 non-aces in the deck - we want to get 1 of them. There are $_{48} C _1 = 48$ ways of doing so. \pause
		
		So $P(\text{4 aces}) = \dfrac{1 \cdot 48}{2598960} \approx 0.0000185$
	\end{frame}

	\begin{frame}{Probability and Counting Rules}
		A box contains 24 integrated circuits, 4 of which are defective. If 4 are sold at random, find the following probabilities. \begin{enumerate}[a)]
			\item Exactly 2 are defective.
			\item None is defective.
			\item All are defective.
			\item At least 1 is defective.
		\end{enumerate}
	\end{frame}

	\begin{frame}{Probability and Counting Rules}
		a) Exactly two are defective. \pause
		
		The number of outcomes in the sample space is $_{24} C _4 = 10626$. \pause \\
		There are 4 defective circuits; we want 2. There are $_4 C _2 = 6$ ways of getting them. \pause \\
		There are 20 working circuits; we want 2. There are $_{20} C _2 = 190$ ways of getting them. \pause
		
		So, $P(\text{exactly 2 defective circuits}) = \dfrac{6 \cdot 190}{10626} \approx 0.107$. \pause
		
		b) None are defective. \pause
		
		The number of outcomes in the sample space is still 10626. \pause \\
		Now, we want no defective circuits. There is $_4 C _0 = 1$ way of getting this outcome. \pause \\
		We want 4 good circuits. There are $_{20} C _4 = 4845$ ways of getting this outcome. \pause
		
		So, $P(\text{no defective circuits}) = \dfrac{4845}{10626} \approx 0.456$.
	\end{frame}

	\begin{frame}{Probability and Counting Rules}
		c) All are defective. \pause
		
		The number of outcomes in the sample space is still 10626. \pause \\
		We want 4 defective circuits this time. There is $_4 C _4 = 1$ way of getting this outcome. \\ \pause
		We want no good circuits. There is $_{20} C _0 = 1$ way of getting this outcome. \pause
		
		So, $P(\text{all circuits defective}) = \dfrac{1}{10626} \approx 0.0000941$. \pause
		
		d) At least one is defective.
		
		We could do this using one of two methods: \pause \\
		$P(\text{at least one defective}) = P(1) + P(2) + P(3) + P(4)$ or \pause \\
		$P(\text{at least one defective}) = 1 - P(0)$. \pause
		
		I'll take option 2. $P(\text{at least one defective}) = 1 - 0.456 \approx 0.544$
	\end{frame}

	\begin{frame}{Probability and Counting Rules}
		A student needs to select two topics to write two term papers for a course. There are 8 topics in economics and 11 topics in science. Find the probability that she selects one topic in economics and one topic in science to complete her assignment. \pause
		
		The number of outcomes in the sample space is $_{19} C _2 = 171$. \pause \\
		There are 8 econ topics; we want to choose 1. There are $_8 C _1 = 8$ ways of doing so. \pause \\
		There are 11 science topics; we want to choose 1. There are $_{11} C _1 = 11$ ways of doing so. \pause
		
		So $P(\text{one econ and one science}) = \dfrac{88}{171} \approx 0.515$.
	\end{frame}

	\begin{frame}{Probability and Counting Rules}
		There are 8 married couples in a tennis club. If 1 man and 1 woman are selected at random to plan the summer tournament, find the probability that they are married to each other. \pause
		
		The number of outcomes in the sample space is $8 \cdot 8 = 64$. \pause \\
		The number of outcomes in which the two selected are married is 8, since there are 8 couples. \pause
		
		So $P(\text{married}) = \dfrac{8}{64} = 0.125$.
	\end{frame}

	\begin{frame}{Probability and Counting Rules}
		All holly plants are dioecious --- a male plant must be planted within 30 to 40 feet of a female plant in order to yield berries. A home improvement store has 12 unmarked holly plants for sale, 8 of which are female. If a homeowner buys 3 plants at random, what is the probability that berries will be produced? \pause
		
		In order to get berries, we need at least one male plant and one female plant. There are two ways of achieving this, given that 3 plants were purchased: \pause \begin{itemize}
			\item One female and two males
			\item Two females and one male
		\end{itemize} \pause
		So if we find these two probabilities and add them, we'll have our answer.
	\end{frame}

	\begin{frame}{Probability and Counting Rules}
		$P(\text{one female, two males}) = \dfrac{_8 C _1 \cdot _4 C _2}{_{12} C _3} = \dfrac{12}{55}$ \pause
		
		$P(\text{two females, one male}) = \dfrac{_8 C _2 \cdot _4 C _1}{_{12} C _3} = \dfrac{28}{55}$ \pause
		
		So $P(\text{berries}) = \dfrac{12}{55} + \dfrac{28}{55} = \dfrac{40}{55} \approx 0.727$
	\end{frame}

	\begin{frame}{Next Steps}
		\begin{itemize}
			\item Complete Assignment \#6
			\item Begin Module \#8 \begin{itemize}
				\item Read 5-1
				\item Watch Video Lesson \#14
			\end{itemize}
		\end{itemize}
	
		\vfill
		
		Thanks for watching!
	\end{frame}
	
\end{document}