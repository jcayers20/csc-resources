\documentclass[t, aspectratio=169]{beamer}
\usepackage{amsmath,amsfonts,amsthm,amstext,amssymb, xcolor, tikz, pgf, mathrsfs, polynom, pifont, tabto}

% ----------------------------------------------------------
% Theme Setup

% Use Metropolis Theme
\usetheme[numbering=fraction]{metropolis}
\setbeamertemplate{blocks}[rounded][shadow=false]
\makeatletter
\setlength{\metropolis@titleseparator@linewidth}{1pt}
\makeatother

% Define Colors
\definecolor{chargerblue}{HTML}{002764}
\definecolor{chargerred}{HTML}{e02034}
\definecolor{bggray}{HTML}{d0d3d4}

% Set Colors
\setbeamercolor{title}{fg=chargerblue}
\setbeamercolor{background canvas}{bg=white}
\setbeamercolor{title separator}{fg=chargerred}
\setbeamercolor{structure}{fg=chargerblue}
\setbeamercolor{frametitle}{fg=white, bg=chargerblue}
\setbeamercolor*{normal text}{fg=chargerblue}
\setbeamercolor*{block body}{bg=bggray}
\setbeamercolor*{block title}{bg=chargerblue, fg=white}
% ----------------------------------------------------------

% ----------------------------------------------------------
% Custom Definitions, Commands, Environments, etc.

% Sets of numbers
\def\R{\mathbb{R}} % The reals
\def\N{\mathbb{N}} % The naturals
\def\Z{\mathbb{Z}} % The integers
\def\Q{\mathbb{Q}} % The rationals

% Blank space
\newcommand{\blank}[1]{\underline{\hspace{#1}}} % Blank space

% Change font colors
\newcommand{\cyan}[1]{{\color{cyan}{#1}}} % Changes font to cyan
\newcommand{\red}[1]{{\color{red}{#1}}} % Changes font to red
\newcommand{\magenta}[1]{{\color{magenta}{#1}}} % Changes font to magenta
\newcommand{\orange}[1]{{\color{orange}{#1}}} % Changes font to orange
\newcommand{\yellow}[1]{{\color{yellow}{#1}}} % Changes font to yellow
\newcommand{\violet}[1]{{\color{violet}{#1}}} % Changes font to violet
\newcommand{\green}[1]{{\color{green}{#1}}} % Changes font to green
\newcommand{\blue}[1]{{\color{blue}{#1}}} % Changes font to blue
\newcommand{\white}[1]{{\color{white}{#1}}} % Changes font to white

% Fitted inclusion symbols
\newcommand{\fp}[1]{\left({#1}\right)} % Fitted parentheses around content
\newcommand{\fb}[1]{\left[{#1}\right]} % Fitted brackets
\newcommand{\lhoi}[1]{\left({#1}\right]} % Left half-open interval
\newcommand{\rhoi}[1]{\left[{#1}\right)} % Right half-open interval
\newcommand{\set}[1]{\left\{{#1}\right\}} % Fitted braces (useful for sets)
\newcommand{\av}[1]{\left|{#1}\right|} % Fitted absolute value bars

% Augmented Matrix Environment
\newenvironment{amatrix}[1]{%
	\left[\begin{array}{@{}*{#1}{c}|c@{}}
	}{%
	\end{array}\right]
}

% Miscellaneous
\def\then{\Rightarrow}
\def\to{\rightarrow}
\def\d{^{\circ}}
\newcommand{\?}{\stackrel{?}{=}}
\newcommand{\cmark}{\text{ \ding{51}}}
\newcommand{\xmark}{\text{ \ding{55}}}

% Coordinate Plane (Four-Quadrant)
\def\coordplane {
	\begin{tikzpicture}        \draw[step=0.25cm,black,very thin,opacity=0.25] (-2.5cm, -2.5cm) grid (2.5cm, 2.5cm);
		\draw[<->,thick,black] (-2.5cm, 0) -- (2.5cm, 0) node[anchor=north west,pos=0.94,font=\scriptsize]{$x$};
		\draw[<->,thick,black] (0,-2.5cm) -- (0, 2.5cm) node[anchor=south east,font=\scriptsize,pos=0.94]{$y$};
	\end{tikzpicture}
}

% Coordinate Plane (One-Quadrant)
\def\onequad {
	\begin{tikzpicture}
		\draw[step=0.25cm, black, very thin, opacity=0.25] (0,0) grid (7.5cm,5cm);
		\draw[->, thick, black] (0,0) -- (7.5cm, 0) node[anchor=north west,font=\scriptsize,pos=0.94]{$x$};
		\draw[->, black, thick] (0,0) -- (0,5cm) node[anchor=south east,font=\scriptsize,pos=0.94]{$y$};
	\end{tikzpicture}
}
% ----------------------------------------------------------

% ----------------------------------------------------------
% Presentation Information
\title[2-3]{Other Types of Graph}
\subtitle{Section 2-3}
\author{Jacob Ayers}
\institute{Lesson \#4}
\date{MAT 110}
% ----------------------------------------------------------

\begin{document}
	
	% Slide 1 (Title Slide)
	\begin{frame}
		\titlepage
	\end{frame}
	
	% Slide 2 (Objectives)
	\begin{frame}{Objectives}
		\begin{itemize}
			\item Construct horizontal and vertical bar charts
			\item Construct Pareto charts
			\item Construct time series graphs
			\item Construct pie graphs
			\item Construct stem and leaf plots
		\end{itemize}
	\end{frame}

	\begin{frame}{Other Types of Graph}
		In this lesson, we will study other commonly-used types of graph.
		
		We can construct most of these in Google Sheets, but we will do stem-and-leaf plots by hand.
	\end{frame}

	\begin{frame}{Bar Charts}
		Bar graphs represent data by using vertical or horizontal bars whose heights or lengths represent the frequencies of the data. \pause
		
		They are used when data is qualitative or categorical. \pause
		
		Let's take a look at a couple examples in Google Sheets (see lesson data on Moodle).
	\end{frame}

	\begin{frame}{Pareto Charts}
		
		A Pareto chart is essentially a vertical bar graph where the bars are arranged from highest to lowest. \pause
		
		As with bar charts, we use Pareto charts when data is qualitative or categorical. \pause
		
		Let's take a look at a couple examples in Google Sheets (see lesson data on Moodle).
	\end{frame}

	\begin{frame}{Time Series Graphs}
		A time series graph is used to represent data that occur over a specified graph. \pause
		
		It is a line graph where the horizontal axis is the year, and the vertical axis is the data value. \pause
		
		Let's take a look at a couple examples in Google Sheets (see lesson data on Moodle).
	\end{frame}

	\begin{frame}{Pie Charts}
		Pie charts show the relationship between the parts and the whole by visually comparing the size of each part. \pause
		
		It is a circle that is divided into sections/wedges according to the percentage of frequencies in each category. \pause
		
		We use pie charts when the variable is nominal or categorical. \pause
		
		Let's take a look at a couple examples in Google Sheets (see lesson data on Moodle).
	\end{frame}

	\begin{frame}{Stem and Leaf Plots}
		A stem and leaf plot is a data plot that uses part of the data as the stem and part as the leaf to form groups or classes. \pause
		
		Example: 17, 31, 9, 18, 36, 24 \pause
		
		We'll let the tens place be the stem, and the ones place be the leaf. \pause
		
		\begin{tabular}{c|l}
			Stem & Leaf \\ \hline
			0 & 9 \\
			1 & 7 \; 8 \\
			2 & 4 \\
			3 & 1 \; 6
		\end{tabular}
	\end{frame}

	\begin{frame}{Stem and Leaf Plots}
		A listing of calories per ounce of selected salad dressings (not fat-free) is given below. Construct a stem and leaf plot for the data (use leaves of 10, 11, etc.).
		
		100 130 130 130 110 110 120 130 140 100 \\
		140 170 160 130 160 120 150 100 145 145 \\
		145 115 120 100 120 160 140 120 180 100 \\
		160 120 140 150 190 150 180 160 \pause
		
		First, sort the data from least to greatest so that we can easily construct the plot: \pause
		
		100 100 100 100 100 110 110 115 120 120 \\
		120 120 120 120 130 130 130 130 130 140 \\
		140 140 140 145 145 145 150 150 150 160 \\
		160 160 160 160 170 180 180 190 \pause
	\end{frame}

	\begin{frame}{Stem and Leaf Plots}
		Now, we can construct the plot. \pause
		
		\begin{tabular}{c|l}
			Stem & Leaf \\ \hline
			10 & 0 \; 0 \; 0 \; 0 \; 0 \\
			11 & 0 \; 0 \; 5 \\
			12 & 0 \; 0 \; 0 \; 0 \; 0 \; 0 \\
			13 & 0 \; 0 \; 0 \; 0 \; 0 \\
			14 & 0 \; 0 \; 0 \; 0 \; 5 \; 5 \; 5 \\
			15 & 0 \; 0 \; 0 \\
			16 & 0 \; 0 \; 0 \; 0 \; 0 \\
			17 & 0 \\
			18 & 0 \; 0 \\
			19 & 0
		\end{tabular}
	\end{frame}

	\begin{frame}{Next Steps}
		\begin{itemize}
			\item Complete Assignment 2
			\item Begin Module 3 \begin{itemize}
				\item Read 3-1
				\item Watch Video Lesson \#5
			\end{itemize}
		\end{itemize}
	
		\vfill
		
		Thanks for watching!
	\end{frame}
	
\end{document}